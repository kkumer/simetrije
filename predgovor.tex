% Correcting the title chapter page
\fancypagestyle{plain}{%
    \fancyhf{}
    \fancyhead[RO,LE]{\bfseries \thepage}
    \fancyhead[CO]{\rightmark}
    \fancyhead[CE]{\leftmark}
    \renewcommand{\headrulewidth}{0.4pt}}


%\thispagestyle{empty}

\hspace*{10ex}
\section*{Predgovor}
\addtocontents{toc}{\protect\vspace*{\protect\baselineskip}}
\addcontentsline{toc}{section}{\slshape Predgovor}

Ovo su nekompletne bilješke za predavanja iz kolegija \emph{Teorija grupa}
i \emph{Simetrije u fizici} koje
držim na Fizičkom odsjeku Prirodoslovno-matematičkog fakulteta u Zagrebu
studentima istraživačkog smjera studija fizike. One su na mjestima samo u obliku
natuknica za predavača pa ih stoga ne
treba doživljavati kao kompletan udžbenik namijenjen samostalnom učenju.
Nadam se da svejedno mogu dobro poslužiti za pripremu ispita.

Dijelovi označeni zvjezdicom se mogu izostaviti pri prvom čitanju.

Povijesno, simetrije su često bile vodilja u otkrićima (Einstein, ...).
Njihova potentnost ide do te mjere da je na kraju naše najapstraktnije
razumijevanje što je to Demokritov atom, tj. elementarna čestica, 
sasvim izraženo kroz simetrije tog objekta. Riječima dobitnika Nobelove
nagrade za važeću teoriju elementarnih čestica, tzv. standardni model,
Stevena Weinberga, \emph{Čestice su grudice energije i impulsa. Što su
    energija i impuls, nego ....Svemir je enorman direktni produkt 
reprezentacija grupa simetrija.} \cite[187]{Crease:1996}.

\begin{flushright}
\begin{minipage}{38ex}
Krešimir Kumerički\\
U Zagrebu, 20. siječnja 2009.
\end{minipage}
\end{flushright}

