% Correcting the title chapter page
\fancypagestyle{plain}{%
    \fancyhf{}
    \fancyhead[RO,LE]{\bfseries \thepage}
    \fancyhead[CO]{\rightmark}
    \fancyhead[CE]{\leftmark}
    \renewcommand{\headrulewidth}{0.4pt}}


\thispagestyle{empty}

\hspace*{10ex}
\section*{Predgovor}
\addtocontents{toc}{\protect\vspace*{\protect\baselineskip}}
\addcontentsline{toc}{section}{\slshape Predgovor}

Ovo je udžbenik koji je nastao za potrebe nastave iz 
kolegija \emph{Teorija grupa} i \emph{Simetrije u fizici} koje
sam niz godina držao na Prirodoslovno-matematičkom fakultetu
Sveučilišta u Zagrebu studentima istraživačkog smjera studija fizike.
Kolegiji se drže na trećoj godini studija, kad
su studenti već ovladali osnovama linearne algebre, jer poznavanje
vektorskih prostora i algebre matrica je temeljni preduvjet za
gradivo ove knjige. Drugi preduvjet je poznavanje osnova kvantne
mehanike, jer su upravo u kontekstu kvantne mehanike načela simetrije
primjenjivana u fizici s najviše uspjeha.


Povijesno, simetrije su često bile vodilja fizičarima, a posebno
je Albert Einstein u svojim teorijama relativnosti pokazao do je koje
mjere moguće ostvariti nove spoznaje o prirodi konzistentnom primjenom
načela simetrija.
Njihova sveobuhvatnost ide do te mjere da je danas naše najapstraktnije
razumijevanje pojma elementarne čestice, moderne realizacije antičke ideje atoma, 
sasvim izraženo kroz simetrije tog objekta. Riječima 
Stevena Weinberga, "\emph{Čestice su grudice energije i impulsa. No što su
energija i impuls nego brojevi definirani translacijama u
vremenu i prostoru. Čestica nije ništa drugo nego
reprezentacija njene grupe simetrija. Svemir je enorman direktni produkt 
reprezentacija grupa simetrija.}" \cite{Crease:1996}.

Osnovni cilj ove knjige je dati u ruke studentu alat široke
upotrebljivosti. Prva četiri poglavlja, koja daju matematičke
osnove teorije konačnih grupa s primjenama na fizikalne situacije
su cjelina od koje bi koristi mogao imati svaki fizičar. 
Zadnja četiri poglavlja su više od koristi teorijskim fizičarima;
zadnja dva štoviše teorijskim fizičarima posebno zainteresiranima za
fiziku visokih energija.

\begin{flushleft}
  \makebox[\textwidth][r]{%
    \begin{tabular}{@{}l@{}}
	Krešimir Kumerički\\
	U Zagrebu, 31. ožujka 2025.
    \end{tabular}%
  }
\end{flushleft}
