\newpage

\begin{tabular}{c|c}
  & Klase konjugacije \\ \hline
\begin{turn}{90} IRREPs \end{turn} & {\Huge $\chi$}
\end{tabular}

\textbf{\Large Pravila za konstrukciju tablice karaktera}

\begin{enumerate}

\item Broj ireducibilnih reprezentacija jednak je broju klasa konjugacije 
  grupe. Iz ovoga slijedi da tablica ima jednak broj redova i stupaca. 
  Broj klasa pronalazimo "pješke," analizom grupe.

\item $\sum_{\alpha} d_{\alpha}^2 = n$ često ima jedinstveno rješenje koje
  određuje dimenzionalnosti $d_{\alpha}$ ireducibilnih reprezentacija.

\item Jedinični element grupe je klasa za sebe, a reprezentiran je uvijek
  jediničnom matricom $D^{(\alpha)}(e)=\Eins$ čiji je 
   karakter $\chi^{(\alpha)}(e)=d_{\alpha}$.
  Ovo određuje jedan (konvencionalno prvi) stupac tablice.

\item Uvijek postoji trivijalna jednodimenzionalna ireducibilna
   reprezentacija  $D(g)=\chi(g)=1  \;
    \forall g\in G$.
  Dakle jedan red (konvencionalno prvi) se sastoji od samih jedinica.

\item Za jednodimenzionalne reprezentacije vrijedi $D(g)=\chi(g)$ pa
  sami karakteri reprezentiraju grupu i njihovo množenje mora odražavati
  množenje odgovarajućih elemenata grupe.

\item  \[\sum_{i=1}^{k} k_{i} \chi^{(\alpha)}_{i} \chi^{(\beta) *}_{i} =
    n \delta^{\alpha\beta}\] 
(\emph{Redovi} tablice su ortogonalni i, kad se uzmu u obzir težinski
  faktori $k_i$, normirani su na $n$)

\item  \[\sum_{\alpha=1}^{k} k_{i} \chi^{(\alpha)}_{i} \chi^{(\alpha) *}_{j} =
    n \delta_{ij} \] 
(\emph{Stupci} tablice su ortogonalni i, kad se uzmu u obzir težinski
  faktori $k_i$, normirani su na $n$)

\end{enumerate}

Pravila je najbolje primjenjivati po redu jer su ona s većim rednim
brojem teža za primjenu i rjeđe nužna za kompletiranje tablice.

\newpage

\textbf{\Large Kristalografske oznake}

\textbf{Ireducibilne reprezentacije} $\Gamma^{(\alpha)}$ se obično označavaju
velikim slovima i to tako da se 1D reprezentacije označavaju slovima
$A$ i $B$, 2D reprezentacije slovom $E$, 3D reprezentacije slovom $T$
itd. Par kompleksno konjugiranih 1D reprezentacija se smatra jednom
2D reprezentacijom (jer ih povezuje vremenska inverzija) tako da se
one udružuju vitičastom zagradom i označavaju s $E$.

\textbf{Klase konjugacije} se obično označavaju simbolom $mC_n$ gdje je $m$
broj elemenata klase, a $C_n$ tipični predstavnik klase označen
Sch\"{o}nfliesovim simbolom:

\begin{tabular}{rcl}
$E$ & = & identiteta \\
$C_n$ & = & rotacija za $2\pi/n$ \\
$\sigma$ & = & refleksija preko ravnine \\
$\sigma_{h}$ & = & refleksija preko ``horizontalne'' ravnine tj. ravnine
  okomite da os najveće  \\ && rotacijske simetrije \\
$\sigma_{v}$ & = & refleksija preko ``vertikalne'' ravnine tj. ravnine
  koja sadrži os najveće \\ &&rotacijske simetrije \\
$\sigma_{d}$ & = & refleksija preko ``dijagonalne'' ravnine tj. ravnine
  koja sadrži os najveće \\ && rotacijske simetrije i raspolavlja kut između
  dvije $C_2$ osi okomite na tu os. \\ &&(Specijalni slučaj $\sigma_{v}$.) \\
$S_n$  & = & rotacija za $2\pi/n$ kombinirana s refleksijom preko ravnine
   okomite na \\ && os te rotacije (Ove dvije operacije komutiraju.) \\
$i$ & = &  $S_2 \;\, = \;\,$  inverzija $\vec{r} \to -\vec{r}$
\end{tabular}

\textbf{Točkaste grupe} kristala se označavaju slijedećim
Sch\"{o}nfliesovim oznakama:

\begin{tabular}{rcl}
$C_n$ & = & grupe s jednom $C_n$ osi simetrije \\
$C_{nv}$ & = & grupe s jednom $C_n$ osi i $n$ $\sigma_v$
   refleksijskih ravnina  \\
$C_{nh}$ & = & $C_n$ os,  $\sigma_h$ refleksija $+$ dodaci \\
$S_{n}$ & = & $S_n$ os \\
$D_{n}$ & = & $C_n$ os i $n$ $C_2$ osi okomitih na nju \\
$D_{nd}$ & = & elementi od $D_{n}$ i $\sigma_d$ ravnine refleksije \\
$D_{nh}$ & = & elementi od $D_{n}$ i $\sigma_h$ ravnina refleksije \\
$T$  & = & tetrahedralna grupa \\
$O$  & = & oktahedralna grupa \\
$\cdots$ && $\cdots \quad$ itd. vidi literaturu \\
\end{tabular}
