%Fancy RCS footer:
     \fancyfoot[C]{\texttt{6\_rotacije.tex}}
     \fancyfoot[RO,LE]{\mbox{$$Revision: 1.10 $$}}
     \fancyfoot[LO,RE]{\mbox{$$Date: 2012-02-24 $$}}
% Correcting the title chapter page
\fancypagestyle{plain}{%
    \fancyhf{}
    \fancyhead[RO,LE]{\bfseries \thepage}
    \fancyhead[CO]{\rightmark}
    \fancyhead[CE]{\leftmark}
     \fancyfoot[C]{\texttt{6\_rotacije.tex}}
     \fancyfoot[RO,LE]{\mbox{$$Revision: 1.10 $$}}
     \fancyfoot[LO,RE]{\mbox{$$Date: 2012-02-24  $$}}
    \renewcommand{\headrulewidth}{0.4pt}
    \renewcommand{\footrulewidth}{0.4pt}}

\chapter{Rotacije i moment impulsa u kvantnoj mehanici}
\label{rotacije}

\section{Ireducibilne reprezentacije grupe SO(2)}

Teorija reprezentacija za kompaktne grupe je drastično različita
 od one za nekompaktne grupe

 \emph{Kompaktne} Lieve grupe su one čiji parametri poprimaju
   vrijednosti iz zatvorenih konačnih intervala:\footnote{
  Pojam kompaktnosti općenito zna biti vrlo netrivijalan, ali za
   naše Lieve grupe koje su \emph{linearne} tj. posjeduju barem
   jednu vjernu konačnodimenzionalnu reprezentaciju vrijedi ova
 pojednostavljena definicija.}
\begin{displaymath}
         p \in [a, b]\;,  \quad a,b\neq \pm \infty
\end{displaymath}

\textbf{Primjeri:}
\begin{itemize}
\item grupa translacija nije kompaktna jer parametri translacije
      nisu ograničeni: $p \in (-\infty, \infty)$
\item Lorentzova grupa nije kompaktna jer parametar Lorentzovog
      potiska dolazi iz otvorenog intervala: $v \in [0,c)$
\item grupa SO(2) je kompaktna: $\phi\in[0, 2\pi=0]\equiv[0,2\pi)$
\item grupa SO(1,1) nije kompaktna
\end{itemize}

Ključna prednost kompaktnosti je činjenica da su kontinuirane funkcije
na kompaktnom skupu integrabilne pa 
velik broj teorema o konačnim grupama vrijedi i za kompaktne Lieve
grupe, uz zamjenu (u dokazima i iskazima):
\begin{displaymath}
    \frac{1}{n} \sum_{g} \longrightarrow \int \rmd g \;,
\end{displaymath}
gdje integral, zahvaljujući kompaktnosti, konvergira.

- Mjeru integrala treba izabrati tako da vrijedi
\begin{displaymath}
  \int dg f(g) = \int dg f(hg) = \int dg f(gh) \quad \forall h \in G \;.
\end{displaymath}
Integraciju koja ima to svojstvo zovemo \emph{invarijantna}.  
Takvo nam je svojstvo
trebalo za dokaze teorema o reprezentacijama. Uvijek je moguće izabrati
mjeru tako da integracija bude invarijantna.

- Tako možemo ``preuzeti'' iz teorije reprezentacije konačnih grupa
  teorem da su sve reprezentacije ekvivalentne unitarnima. (To nam
 je važno jer je unitarnost željeno svojstvo transformacija kvantnomehaničkih
 stanja.)

- Specijalno, svi IRREPSi od SO(2) su ekvivalentni unitarnima, pa se,
  kao što smo to radili i kod konačnih grupa, smijemo ograničiti na
  unitarne IRREPSe bez gubitka općenitosti. 

- SO(2) je Abelova $\imp$ svi IRREPsi su 1D, što uz unitarnost 
($uu^\dagger = uu^* = |u|^2=1$) znači:
\begin{displaymath}
D(\phi)=e ^{-i f(\phi)} \;,  f(\phi) \in \mathbb{R} \;.
\end{displaymath}

Zahtjev homomorfnosti s grupom rotacija, koja je aditivna, dalje
povlači
\begin{displaymath}
   f(\phi_1)+f(\phi_2) = f(\phi_1+\phi_2)
\end{displaymath}
što znači da je $f(\phi)=m\phi$, $m\in\mathbb{R}$.
Na kraju, zahtjev periodičnosti daje:

- $D^{(m)}(\phi)=D^{(m)}(\phi + 2\pi) \imp m \in \mathbb{Z}$

- Dakle, imamo prebrojivo beskonačno 1D IRREPsa, i označavamo
  ih cijelim brojem $m$.

- Karakteri: $\chi^{(m)}(\phi)=D^{(m)}(\phi)=e^{-\rmis m \phi}$

- Ortogonalnost:
\begin{displaymath}
(\chi^{(m)}, \chi^{(m')})=\int_{0}^{2\pi}\frac{\rmd \phi}{2\pi}\,
 e^{\rmis m \phi}e^{-\rmis m' \phi} = \delta_{mm'}
\end{displaymath}

- (D.Z. Uvjerite se da je ovakva integracija invarijantna u gornjem smislu.)


- Koeficijenti u razvoju direktnog produkta dvaju IRREPsa:
\begin{displaymath}
  \Gamma^{(m)}\otimes\Gamma^{(n)} =
  \sum \oplus\: a_{k}\Gamma^{(k)} 
\end{displaymath}
\begin{eqnarray*}
a_{k} & = & (\chi^{(k)}, \chi^{(m)}\chi^{(n)}) \\
      & = & \int_{0}^{2\pi}\frac{\rmd \phi}{2\pi}\,
    e^{\rmis k \phi}e^{-\rmis m \phi}e^{-\rmis n \phi} = \delta_{k,m+n}
\end{eqnarray*}
tj.
\begin{displaymath}
  \Gamma^{(m)}\otimes\Gamma^{(n)} =
  \sum_{k} \oplus\: \delta_{k,m+n}\Gamma^{(k)} = \Gamma^{(m+n)}
\end{displaymath}

- Rastav reducibilne reprezentacije:
\begin{displaymath}        
 D_{\rm 2D} (\phi) = \left( 
\begin{array}{cc}        
\cos\phi & -\sin\phi \\
\sin\phi & \cos\phi
\end{array}                
\right) \qquad \chi_{\rm 2D}=2\cos\phi
\end{displaymath}
\begin{eqnarray*}
a_k & = & (\chi^{(k)}, \chi_{\rm 2D}) \\
    & = & \int_{0}^{2\pi}\frac{\rmd \phi}{2\pi}\,
   e^{\rmis k \phi} 2\cos\phi  \\
 & = & \int_{0}^{2\pi}\frac{\rmd \phi}{2\pi}\,
  e^{\rmis k \phi} ( e^{\rmis  \phi} + e^{-\rmis  \phi}) \\
 & = & \delta_{k,-1} + \delta_{k, 1}
\end{eqnarray*}
Znači postoji $S$ [pronađite za D.Z.] takva da
\begin{displaymath}
  S D_{\rm 2D} (\phi) S^{-1} = \left(
\begin{array}{cc}        
e^{\rmis  \phi} &  0 \\
0 & e^{- \rmis  \phi}
\end{array}                
\right) 
\end{displaymath}

\subsubsection*{Digresija: Projektivne reprezentacije$^*$}

- U QM postoje sustavi za koje $m = 1/2, 3/2, 5/2, \ldots$ tj.
  $D^{(m)}(\phi)=-D^{(m)}(\phi + 2\pi)$


Po definiciji, reprezentacija $\Gamma=\{D(\phi)\}$ grupe G mora zadovoljavati:
\begin{displaymath}
    D(\phi_1)D(\phi_2)=D(\phi_1\circ_{\textrm{\tiny G}}\phi_2)
        =D(\phi_1 + \phi_2)
\end{displaymath}
dok za $m = 1/2, 3/2, 5/2, \ldots$ imamo npr.
\begin{displaymath}
 D(\pi) D(\pi) = e^{i m \pi}e^{i m \pi} = e^{2 i m \pi}=-1
\neq D(2\pi)=1 \;.
\end{displaymath}
Riječ je o tome da se u QM $D(\phi_1)D(\phi_2)\ket{\alpha}$ i
$D(\phi_1 + \phi_2)\ket{\alpha}$ smiju razlikovati za fazu
a da i dalje opisuju isto fizikalno stanje.

Tako na QM stanjima dopuštamo reprezentiranje grupe i tzv.
\emph{projektivnim reprezentacijama} s ``olabavljenim'' uvjetom homomorfizma:
\begin{displaymath}
   D(g_1)D(g_2)=e^{i \phi(g_1,g_2)}D(g_1 g_2) \;.
\end{displaymath}

- O grupi ovisi hoće li ona imati i projektivnih REP, a o fizikalnom
sustavu hoće li se transformirati obzirom na projektivne REP ili ne.

- Jedan od načina da grupa ima i projektivne reprezentacije je da
  ne bude jednostavno povezana.

SO(3) je dvostruko povezana i dopušta dvije faze: $\pm 1$. Fermioni
su čestice koje se transformiraju projektivno.

SO(2) je beskonačno povezana i ima beskonačno mogućih faza. No, sustavi
koje razmatramo su 3D i čak i pod 2D rotacijama mogu samo promijeniti
predznak tj. SO(2) je ovdje samo podgrupa od SO(3).

\section{Ireducibilne reprezentacije grupe SO(3) tj. SU(2)}

- Grupa SO(3) nije Abelova i IRREPSi više nisu 1D pa konstrukcija nije
tako laka kao za SO(2).

- Lakše je konstruirati reprezentacije SO(3) \emph{algebre}, a onda
  eksponencijacijom dobiti reprezentacije grupe.

- Algebru čine tri generatora grupe, $X_1, X_2, X_3$, eksplicitno navedena u
(\ref{eq:SO3generators}), sa komutacijskim relacijama
\begin{equation}
          [X_i, X_j] = \epsilon_{ijk} X_k
\end{equation}
ali mi ćemo raditi s malo drugačije izraženom algebrom, preko kvantnomehaničkih operatora
momenta impulsa, $J_1, J_2, J_3$, $J_i=\rmi \hbar X_i$ koji zadovoljavaju
\begin{equation}
          [J_i, J_j] = i \hbar \epsilon_{ijk} J_k
\label{eq:SU2algebra}
\end{equation}
 
Elementi grupe su $e^{\phi\unitn\cdot\vec{X}}=e^{(-i/\hbar)\phi\unitn\cdot
 \vec{J}}$ što se slaže sa odjeljkom \ref{sec:lievealgebre}.

- Iz unitarnosti reprezentacije ondmah slijedi da su $J_i$ hermitski, kao
  što i treba biti. Znači da su njihove svojstvene vrijednosti realne.

- Operator $\vec{J}^2 = J_{x}^2 + J_{y}^2 + J_{z}^2$ je Casimirov
  (cf. odjeljak 4.2) tj. komutira sa svim elementima algebre:
\begin{equation}
   [\jsq, J_{i}^2]=0\;, \quad i=1,2,3
\end{equation}

- Kako nas zanimaju IRREPSi, iz druge Schurove leme slijedi da $\jsq$
mora biti proporcionalan jediničnom operatoru: $\jsq \propto \Eins$.
Tako se svojstvena vrijednost od $\jsq$ ne mijenja kroz cijelu IRREP i
zgodna je za njeno označavanje. Usporedimo sad IRREPse grupe rotacija
na običnom 3D prostoru i na prostoru kvantnomehaničkih stanja.

\begin{itemize}
\item  IRREP na 3D euklidskom prostoru:
\begin{displaymath}
   D^{(3D)}_{ij}(\unitn, \phi) r_j = \text{lin. komb. od}\: \hat{x}, \hat{y}, \hat{z}
\end{displaymath}
   $\unitn, \phi$ --- parametri \\
  $3D$ --- oznaka IRREPsa \\
  $r_j$ --- koordinate vektora $r_x, r_y, r_z$

\item IRREPsi rotacija na Hilbertovom prostoru:
\begin{displaymath}
   D^{(\beta)}(\unitn,\phi) \ket{\beta, m} =
 e^{(-i/\hbar)\vec{J}\cdot\unitn\phi}\ket{\beta, m} =
\text{lin. komb. od}\: \ket{\beta, m_1}, \ket{\beta, m_2}, \dots
\end{displaymath}
$\beta$ --- indeks IRREPsa (na $\vec{J}$ često implicitan, no
 ponekad se piše $\vec{J}^{(\beta)}$) \\
$ m $ --- ``koordinata'' koja razlikuje vektore unutar istog IRREPsa.\\
 Vektori baze prostora na koji djeluje IRREP $\beta$ su definirani slijedećim
relacijama:
\begin{eqnarray}
  \jsq \ket{\beta, m} &=& \beta \hbar^2 \ket{\beta, m} \;.  \\
  J_z \ket{\beta, m} &=& m \hbar \ket{\beta, m} \;.
\end{eqnarray}
\end{itemize}

Sada ćemo eksplicitno konstruirati IRREPse od $SU(2)$, koristeći
isključivo komutacijske relacije (\ref{eq:SU2algebra}).
\secret{Ova konstrukcija se radi na kvantnoj pa se može
preskočiti i samo rekapitulirati sažetak na kraju ovog odjeljka.}
Prvo definiramo operatore \emph{podizanja i spuštanja}
\begin{displaymath}
    J_{\pm} = J_x \pm i J_y \;, \qquad [\jsq, J_\pm] =0
\end{displaymath}
 
\begin{eqnarray*}
 [J_z, J_\pm]& = & [J_z, J_x] \pm i [J_z, J_y] = i\hbar J_y \pm i (-i\hbar J_x)
 \\ & = & \pm \hbar (J_x \pm i J_y) = \pm \hbar J_\pm
\end{eqnarray*}

Slično (pokažite):
\begin{displaymath}
 [J_+, J_-] = 2 \hbar J_z
\end{displaymath}

 Djelovanjem ovih operatora dizanja i spuštanja na vektore neke IRREP,
ostajemo naravno u toj istoj IRREP:
\begin{displaymath}
   \jsq(J_{\pm}\ket{\beta, m})=J_{\pm}\jsq\ket{\beta, m}=\beta\hbar^2
    (J_{\pm}\ket{\beta, m}) \;,
\end{displaymath}
ali svojstvena vrijednost od $J_z$ se mijenja:
\begin{equation}
J_z(J_{\pm}\ket{\beta, m})=([J_z,J_{\pm}]+J_{\pm}J_z)\ket{\beta, m}=
\hbar(m\pm 1)J_{\pm}\ket{\beta, m} \;,
\label{jzjpm}
\end{equation}
tj.
\begin{displaymath}
    J_{\pm}\ket{\beta, m}\propto \ket{\beta, m\pm 1} \;.
\end{displaymath}


Do kuda  može ići to dizanje i spuštanje tj. koliko različitih vrijednosti
može poprimiti $m$?

Tvrdnja: $m^2 \leq \beta$

Dokaz: $J_{\pm}^{\dagger}=J_{\mp}$, jer $J^{\dagger}_{x,y,z}=J_{x,y,z}$.
\begin{eqnarray*}
 \frac{1}{2}(J_+ J_{+}^{\dagger} + J_{+}^{\dagger}J_+)& = &
 \frac{1}{2}(J_+ J_{-}+ J_- J_+) \\
& = & \frac{1}{2} \left[ (J_x + i J_y)(J_x - i J_y) + \textrm{h. c.} \right]
\\ & = & \frac{1}{2} \left[ J_{x}^2 + i J_y J_x - i J_x J_y + J_{y}^2
  + J_{x}^2 - i J_y J_x + i J_x J_y + J_{x}^2 \right] \\
& = & J_{x}^2 + J_{y}^2 = \jsq - J_{z}^2
\end{eqnarray*}
No,
\begin{displaymath}
\bra{\beta, m}J_{+}^{\dagger}J_+ \ket{\beta, m} =
\bra{J_{+}\beta, m}J_{+}\beta, m\rangle  \geq 0 \quad \textrm{i isto za}
J_+ J_{+}^{\dagger}
\end{displaymath}
Slijedi da je
\begin{displaymath}
\bra{\beta, m}(\jsq - J_{z}^2)\ket{\beta, m}=
\bra{\beta, m}(\beta\hbar^2 - m^2\hbar^2)\ket{\beta, m} =
(\beta - m^2)\hbar^2 \geq 0
\end{displaymath}
Q.E.D.

Dakle, za svaki $\beta$ postoji $m_{\textrm{\scriptsize max}}$ tako da 
\begin{displaymath}
 J_+ \ket{\beta, m_{\textrm{\scriptsize max}}} = 0
\end{displaymath}
(To je jedini način da (\ref{jzjpm}) ostane vrijediti.)

$m_{\textrm{\scriptsize max}}(\beta)$=?

\begin{displaymath}
J_- J_+ \ket{\beta, m_{\textrm{\scriptsize max}}} = 0
\end{displaymath}
\begin{eqnarray*}
J_- J_+ =(J_x - i J_y)(J_x - i J_y)
   & = &J_{x}^2 + J_{y}^2 +i (J_x J_y - J_y J_x) \\
 & = & J_{x}^2 + J_{y}^2 - \hbar J_z \\
 & = & \jsq - J_{z}^2 - \hbar J_z
\end{eqnarray*}
Pa je
\begin{displaymath}
 (\jsq - J_{z}^2 - \hbar J_z)\ket{\beta, m_{\textrm{\scriptsize max}}}
 = (\beta\hbar^2 - m_{\textrm{\scriptsize max}}^2 \hbar^2 
                 - m_{\textrm{\scriptsize max}}   \hbar^2 )
\ket{\beta, m_{\textrm{\scriptsize max}}} = 0 \;,
\end{displaymath}
pa kako $\ket{\beta, m_{\textrm{\scriptsize max}}}$ nije nul-vektor
slijedi da je
\begin{displaymath}
    \beta = m_{\textrm{\scriptsize max}} ( m_{\textrm{\scriptsize max}} +1)\;.
\end{displaymath}

Iz maločas dokazane činjenice da je $m^2 \le \beta$ slijedi također
da ni spuštanje vrijednosti $m$ operatorom spuštanja ne može ići
unedogled već da mora postojati $m_{\textrm{\scriptsize min}}$
sa svojstvom
\begin{displaymath}
    J_- \ket{\beta, m_{\textrm{\scriptsize min}}} = 0 \;,
\end{displaymath}
iz čega postupkom analognim ovom gore dolazimo do
\begin{displaymath}
    \beta = m_{\textrm{\scriptsize min}} ( m_{\textrm{\scriptsize min}} -1)\;.
\end{displaymath}

Izjednačivši ove dvije vrijednosti za $\beta$ 
\begin{displaymath}
            m_{\textrm{\scriptsize max}} ( m_{\textrm{\scriptsize max}} +1)
= m_{\textrm{\scriptsize min}} ( m_{\textrm{\scriptsize min}} -1)
\end{displaymath}
dobijemo kvadratnu jednadžbu za $m_{\textrm{\scriptsize min}}$ od čija
dva rješenja $m_{\textrm{\scriptsize min}}=m_{\textrm{\scriptsize max}}+1$
i $m_{\textrm{\scriptsize min}} = - m_{\textrm{\scriptsize max}}$ ovo
prvo ne dolazi u obzir jer je $m_{\textrm{\scriptsize max}}$ po pretpostavci
najveća moguća vrijednost za $m$. Tako imamo, uvodeći oznaku
$j=m_{\textrm{\scriptsize max}}$,
\begin{displaymath}
               -j = m_{\textrm{\scriptsize min}} \le m \le 
 m_{\textrm{\scriptsize max}} = j   \;.
\end{displaymath}

Nadalje, kako operatori $J_\pm$ dižu ili spuštaju $m$ točno za 1, mora
biti $m_{\textrm{\scriptsize max}} = m_{\textrm{\scriptsize min}} +n $,
gdje je $n\in \mathbb{N}_{0}$, tj. $j = -j +n  \imp j=n/2$
\begin{displaymath}
  j= 0, \frac{1}{2}, 1, \frac{3}{2}, 2, \ldots \;.
\end{displaymath}

  Vrijednost $\beta = j(j+1)$ je ista za cijelu IRREP i uobičajenije
je upravo $j$ koristiti za označavanje IRREPa:
\begin{displaymath}
  \ket{\beta, m} \longrightarrow \ket{j, m} \;,
\end{displaymath}
gdje $m$ poprima vrijednosti
\begin{displaymath}
   m= -j, -j+1, \ldots, j-1, j
\end{displaymath}
i gdje vektori baze IRREPa $\ket{j, m}$ zadovoljavaju
\begin{eqnarray*}
 \jsq \ket{j, m}& = & \hbar^2 j(j+1)\ket{j, m} \quad \mbox{i} \\
 J_z \ket{j, m} & =  &\hbar m \ket{j, m} \;.
\end{eqnarray*}
Tako smo koristeći isključivo komutacijske relacije algebre grupe rotacija
konstruirali sve njene ireducibilne reprezentacije!

Kako je ta algebra zajednička i grupi $SO(3)$ i grupi $SU(2)$ dobili
smo i reprezentacije s polucjelobrojnim momentom impulsa $j$.

Još nas zanima i točno djelovanje operatora $J_\pm$ na vektore baze:
\begin{displaymath}
     J_+ \ket{j, m} = c_{jm} \ket{j, m+1} \;.
\end{displaymath}
Zahvaljujući ortonormiranosti:
\begin{eqnarray*}
 |c_{jm}|^2 & = & \bra{j, m} J_- J_+ \ket{j, m} \\
            & = & \bra{j, m}(\jsq - J_{z}^2 - \hbar J_z \ket{j, m} \\
& = & \bra{j, m}(\hbar^2 j(j+1) - \hbar^2 m^2 - \hbar^2 m \ket{j, m}  \\
            & = & \hbar^2[j(j+1)-m(m+1)] \\
            & = & \hbar^2 (j-m)(j+m+1)
\end{eqnarray*}
Faza od $c_{jm}$ je neodređena i prema tzv. Condon-Shortley konvenciji
odabire se realan i pozitivan $c_{jm}$:
\begin{displaymath}
  J_+ \ket{j, m} = \hbar \sqrt{(j-m)(j+m+1)} \ket{j, m+1} \;.
\label{jplus}
\end{displaymath}
Slično, [pokažite]
\begin{displaymath}
  J_- \ket{j, m} = \hbar \sqrt{(j+m)(j-m+1)} \ket{j, m-1} \;.
\label{jminus}
\end{displaymath}

Ove relacije nam omogućuju da odredimo matrične elemente operatora
momenta impulsa između proizvoljnih stanja. Također, eksponencijacijom
možemo odrediti matrične elemente samih operatora rotacija:
\begin{displaymath}
    \bra{j,m'} D^{(j)}(\hat{\vec{n}},\phi) \ket{j, m} \quad - \quad
\text{Wignerove D-funkcije}
\end{displaymath}
Za konkretne primjere vidi vježbe.

\textbf{Sažetak:}

\begin{itemize}
\item  Operatori momenta impulsa $J_x, J_y$ i $J_z$ te njihove kombinacije
$\jsq, J_\pm, \ldots$ za dani $j$, tj. za danu ireducibilnu reprezentaciju
grupe rotacija, djeluju na ($2j+1$)-dimenzionalnom vektorskom prostoru
razapetom vektorima $\ket{j, m}$, $m=-j, -j+1, \ldots, j-1, j$.

\item operatori $D^{(j)}=e^{-i \vec{J}\cdot\unitn\phi/\hbar}$ čine
  reprezentaciju grupe rotacija na ($2j+1$)-dimenzionalnom  Hilbertovom
  vektorskom prostoru kvantnomehaničkih stanja

\item $J_{\pm}$ povezuju svih $2j+1$ vektora $\ket{j, m}$ tj.
  reprezentacija je stvarno ireducibilna

\item baza vektorskog prostora $\ket{j, m}$, $m=-j, -j+1, \ldots, j-1, j$ 
 se obično naziva \emph{multiplet}. (Neki autori tako zovu cijeli
 vektorski prostor.)
  \begin{itemize}
\item  $j=0$ : \emph{singlet}
\item  $j=\frac{1}{2}$ : \emph{dublet}
\item  $j=1$ : \emph{triplet}
\item  $\cdots$ 
\end{itemize}
\end{itemize}

\textbf{Orbitalni moment impulsa:}

- Specijalna vrsta momenta impulsa koja je oblika $\vec{L} = \vec{r}
  \times \vec{p}$

- Kao posljedica toga (ne sasvim trivijalna \secret{cf. Kaplan and Wu, 
Chinese J. of Phys. \textbf{9} (1971) 31; Gatland, Am. J. Phys.
\textbf{74} (2006) 191.}), svojstvene vrijednosti su 
 isključivo  cjelobrojne:
\begin{displaymath}
     \vec{L}^2 \ket{l, m} = \hbar^2 l(l+1) \ket{l,m} \qquad l=0,1,2,\ldots
\end{displaymath}
i označavaju se s $l$, a ne $j$.

- Iz povijesnih razloga vezanih uz atomsku fiziku, stanja s $l=0,1,2,\ldots$
se često nazivaju $S, P, D,\ldots$ stanja, a $m$ se ponekad naziva
\emph{magnetski} kvantni broj (zbog svoje uloge u Zeemanovom efektu).

\section{Zbrajanje momenata impulsa i Clebsch-Gordanovi koeficijenti}

Sustav od dvije čestice, sa spinovima $j_1$ i $j_2$ u stanju 
$\ket{j_1, m_1; j_2, m_2}$ ima svojstva:
\begin{align*}
\vec{J}_{1}^2 \ket{j_1, m_1; j_2, m_2} &= j_1(j_1+1)\hbar^2 
  \ket{j_1, m_1; j_2, m_2} \\
\vec{J}_{2}^2 \ket{j_1, m_1; j_2, m_2} &= j_2(j_2+1)\hbar^2 
  \ket{j_1, m_1; j_2, m_2} \\
J_{1z}\ket{j_1, m_1; j_2, m_2} &= m_1 \hbar \ket{j_1, m_1; j_2, m_2}\\
J_{2z}\ket{j_1, m_1; j_2, m_2} &= m_2 \hbar \ket{j_1, m_1; j_2, m_2}
\end{align*}
\begin{align*}
\ket{j_1, m_1} &\in V_1\\
\ket{j_2, m_2} &\in V_2\\
\ket{j_1, m_1; j_2, m_2}&\equiv\ket{j_1, m_1}\ket{j_2, m_2} \in V_1\otimes V_2
\end{align*}
Na vektorskom
prostoru $V_1$ grupa rotacija reprezentirana je operatorima $D^{(j_1)}=
\exp(-i\vec{J}_1\cdot\hat{\vec{n}}\theta/\hbar)$ (i slično za $V_2$),
a na $V_1\otimes V_2$ operatorima
$D^{(j_1)}\otimes D^{(j_2)}$.

Međutim, $D^{(j_1)}\otimes D^{(j_2)}$ je općenito reducibilna reprezentacija
koja nema dobro definiran spin. $\ket{j_1, m_1; j_2, m_2}$ nije nužno svojstveno
stanje od $\vec{J}^2 = (\vec{J}_1 + \vec{J}_2)^2$ --- operatora ukupnog spina
sustava.

Rastav na IRREPSE:
\begin{displaymath}
  D^{(j_1)}\otimes D^{(j_2)} =
  \sum_J \oplus\: a_{J}D^{(J)} \quad
   \textrm{ - Clebsch-Gordanov razvoj }
\end{displaymath}

\begin{displaymath}
a_J = \left( \chi^{(J)}, \chi^{(j_1)} \chi^{(j_2)} \right)
\end{displaymath}

Za općeniti račun skalarnih umnožaka karaktera trebali bismo
znati invarijantno integrirati u grupnom prostoru (cf. Jones,
Appendix C ili Hamermesh 9-2), ali ovdje ćemo potrebni rezulatat
dobiti i bez toga, uz par matematičkih trikova:

Kao reprezentante klasa biramo rotacije oko $z$-osi.
\begin{equation}
\begin{split}
 \chi^{(j)}(\phi) &= \Tr D^{(j)}(\phi) = \Tr e^{(-i/\hbar)J_3 \phi} \\
 &= \Tr \textrm{diagonal}\left( e^{-i j\phi }, e^{-i (j-1)\phi}, \ldots,
e^{-i(-j)\phi}\right)\\
&= e^{-ij\phi } + e^{-i (j-1)\phi} + \cdots + e^{-i(-j)\phi}\\
&= \text{geom. red s omjerom članova $e^{i j\phi}$} \\
&= e^{-i j\phi} \frac{1- (e^{i j\phi})^{2j+1}}{1-e^{i\phi}}
= \frac{e^{-i(j+1/2)\phi} - e^{i(j+1/2)\phi}}{e^{-i\phi/2}-e^{i\phi/2}}\\
&= \frac{\sin(j+1/2)\phi}{\sin\phi/2}
\end{split}
\end{equation}


Odredimo sada koeficijente $a_J$ Clebsch-Gordanovog razvoja. Pretpostavimo
prvo da je $j_1 \ge j_2$.
\begin{equation}
\begin{split}
\chi^{(j_1)}(\phi)\chi^{(j_2)}(\phi)&=
 \frac{e^{+i(j_1+1/2)\phi} - e^{-i(j_1+1/2)\phi}}{2i\sin\phi/2}
\sum_{m=-j_2}^{j_2}e^{i m \phi}\\
&= \frac{1}{2 i \sin\phi/2} \sum_{m=-j_2}^{j_2}
\left[ e^{i(j_1+m+1/2)\phi} - e^{-i(j_1-m+1/2)\phi}\right]\\
&= (\text{zamjena $m\to -m$ u drugom članu}) \\
&=\sum_{m=-j_2}^{j_2} \frac{\sin(j_1+m+1/2)\phi}{\sin\phi/2}\\
&=\sum_{m=-j_2}^{j_2} \chi^{(j_1 + m)}(\phi) \qquad (J\equiv j_1 +m) \\
&=\sum_{J=j_1-j_2}^{J=j_1+j_2} \chi^{(J)}(\phi) \;.
\end{split}
\end{equation}
($J$ mora biti pozitivan da bi bio labela neke IRREPS. Zato smo
tražili $j_1\ge j_2$.)
Za slučaj $j_2 \ge j_1$ imali bi sve isto, uz zamjenu $j_1 \leftrightarrow
j_2$. Slijedi da općenito možemo pisati:
\begin{displaymath}
 \chi^{(j_1)}(\phi)\chi^{(j_2)}(\phi)=
\sum_{J=|j_1-j_2|}^{J=j_1+j_2} \chi^{(J)}(\phi)
\end{displaymath}
Dakle,
\begin{displaymath}
 a_J = \left(\chi^{(J)}, \chi^{(j_1)}\chi^{(j_2)}\right) =
 \sum_{J'=|j_1-j_2|}^{J'=j_1+j_2} 
\underbrace{\left(\chi^{(J)}, \chi^{(J')}\right)}_{\delta_{JJ'}}
\end{displaymath}
odnosno,
\begin{equation}
D^{(j_1)} \otimes D^{(j_2)} = \sum_{J=|j_1-j_2|}^{J=j_1+j_2} 
\oplus D^{(J)}
\end{equation}

Npr. $D^{(1/2)}\otimes D^{(1)} = D^{(1/2)} \oplus D^{(3/2)}$.

- Ovo da se svaka ireducibilna reprezentacija pojavljuje najviše jednom
je veliko pojednostavljenje svojstveno grupi rotacija.

- D.Z. Uvjerite se da se dimenzije dobro zbrajaju tj. da je 
\begin{displaymath}
    \sum_{J=|j_1-j_2|}^{J=j_1+j_2} (2J+1) = (2j_1+1)(2j_2+1)
\end{displaymath}

No, za primjene je zanimljivije zbrajanje stanja!

$\ket{j_1, m_1; j_2, m_2}$ su vektori koji su baza od $V_1 \otimes V_2$ i
koji su svojstveni vektori operatora $\{\jsqj, \jsqd, J_{1z}, J_{2z}\}$. No,
postoji i baza $\ket{j_1, j_2; J, M}\equiv\ket{J, M}$
tog istog prostora koju čine svojstveni vektori operatora
$\{\jsqj, \jsqd, \jsq, J_z\}$.

- D.Z. Uvjerite se da je
\begin{displaymath}
   [\jsq, \jsqj] = [\jsq, \jsqd] = 0
\end{displaymath}

- Dvije baze su naravno povezane:
\begin{equation*}
\ket{J, M} = \underbrace{\sum_{m_1,m_2}\ket{j_1, m_1; j_2, m_2}
\bra{j_1, m_1; j_2, m_2}}_{=1}
\ket{J, M}
\end{equation*}
\begin{equation}
 \bra{j_1, m_1; j_2, m_2} J, M\rangle \equiv C^{JM}_{j_1m_1j_2m_2}
\quad \text{Clebsch-Gordanovi koeficijenti}
\end{equation}

\subsubsection{Svojstva Clebsch-Gordanovih koeficijenata$^*$}
\begin{itemize}
\item $C^{JM}_{j_1m_1j_2m_2}=0$ ako nije $|j_1-j_2|\le J\le j_1 + j_2$. \\
Dokaz: očito iz rastava $D^{(j_1)}\otimes D^{(j_2)}$.

\item $C^{JM}_{j_1m_1j_2m_2}=0$ ako nije $M=m_1+m_2$.\\
Dokaz: 
\begin{gather}
\vec{J}=\vec{J}_{1}+\vec{J}_{2} \imp J_z-J_{1z}-J_{2z}=0 \\
\bra{j_1, m_1; j_2, m_2}(J_z-J_{1z}-J_{2z})\ket{J, M} = 0 \\
(M-m_1-m_2)\bra{j_1, m_1; j_2, m_2} J, M\rangle = 0
\end{gather}

\item CG-koeficijenti imaju neodređenu fazu. Standardni izbor je
da se uzme $C^{JJ}_{j_1 j_1 j_2 (J-j_1)}$ realan i pozitivan.
Kao (netrivijalna) posljedica toga, svi CG-koeficijenti ispadaju realni.

\item
\begin{displaymath}
 \sum_{JM} C^{JM}_{j_1m_1j_2m_2} C^{JM}_{j_1m'_1j_2m'_2} =
\delta_{m_1 m'_1} \delta_{m_2 m'_2}
\end{displaymath}

\item
\begin{displaymath}
  \sum_{m_1, m_2} C^{JM}_{j_1m_1j_2m_2}C^{J'M'}_{j_1m_1j_2m_2} =
 \delta_{JJ'}\delta_{MM'}
\end{displaymath}

\item
\begin{displaymath}
 \ket{J, M} = \sum_{m_1, m_2} C^{JM}_{j_1m_1j_2m_2}\ket{j_1, m_1; j_2, m_2}
\end{displaymath}
(Isti koeficijenti pretvaraju baze u oba smjera.). Za dokaz, 
pomnožiti s $\sum_{J,M}C^{JM}_{j_1m'_1j_2m'_2}$ i koristiti svojstva
ortogonalnosti.

\item
\begin{displaymath}
 C^{JM}_{j_1m_1j_2m_2} = (-1)^{J-j_1 -j_2} C^{JM}_{j_2 m_2 j_1 m_1}
\end{displaymath}
\secret{Netrivijalno, cf. Cornwell}
\end{itemize}

\subsubsection{Izračunavanje Clebsch-Gordanovih koeficijenata$^*$}
\label{tripletsinglet}

Primjer: 
 $D^{(1/2)}\otimes D^{(1/2)} = D^{(1)} \oplus D^{(0)}$.

Prva baza: $\ket{1/2, m_1; 1/2, m_2}\equiv\ket{m_1, m_2}$.
$m_{1,2}=\pm 1/2$ $\imp$ 4 stanja

Druga baza: $\ket{J, M}$. $M=-1, 0, 1$ za $J=1$ i $M=0$ za $J=0$.
$\imp 3+1=4$ stanja.

\begin{equation*}
\begin{split}
 \ket{1,1} &= \sum_{m_1, m_2}C^{1 1}_{\fhalf m_1 \fhalf m_2}\ket{m_1, m_2}\\
& (M=m_1+m_2=1 \imp m_1=m_2=\fhalf )\\
&= C^{1 1}_{\fhalf \fhalf \fhalf \fhalf}\ket{\fhalf, \fhalf}
\end{split}
\end{equation*}

To što su oba stanja normirana povlači da je $|C^{1 1}_{\fhalf \fhalf
\fhalf \fhalf}|=1$. Već smo odabrali da je $C\in\mathbb{R}$, a sada
još biramo i da je pozitivan: 
$C^{1 1}_{\fhalf \fhalf \fhalf \fhalf}=1$.
\begin{displaymath}
    \ket{1, 1} = \ket{\fhalf, \fhalf}
\end{displaymath}

Sada, da bismo dobili ostale CG-koeficijente,
djelujemo s $J_- = J_{1-}+J_{2-}$ na obje strane ove jednadžbe:

\begin{align*}
J_- \ket{1, 1}&=\hbar \sqrt{(1+1)(1-1+1)} \ket{1,0} =\hbar\sqrt{2}\ket{1, 0}\\
J_{1-} \ket{\fhalf, \fhalf}&=\hbar\ket{-\fhalf, \fhalf}\\
J_{2-} \ket{\fhalf, \fhalf}&=\hbar\ket{\fhalf, -\fhalf}\\
\end{align*}

Slijedi
\begin{displaymath}
 \ket{1, 0} = \frac{1}{\sqrt{2}}\left(\ket{\fhalf, -\fhalf}+
\ket{-\fhalf, \fhalf}\right)
\end{displaymath}
tj.
\begin{displaymath}
C^{1 0}_{\fhalf -\fhalf \fhalf \fhalf}=
C^{1 0}_{\fhalf  \fhalf \fhalf -\fhalf} = \frac{1}{\sqrt{2}}
\end{displaymath}

Nadalje,
\begin{displaymath}
    \ket{1, -1} = \ket{-\fhalf, -\fhalf} \imp
C^{1 -1}_{\fhalf -\fhalf \fhalf -\fhalf}= 1
\end{displaymath}
D.Z. - Dobiti ovo s $J_-$.

Na kraju, napišimo, imajući u vidu da $M=m_1+m_2$
\begin{displaymath}
 \ket{0,0} = \alpha\ket{\fhalf, -\fhalf}+\beta\ket{-\fhalf, \fhalf} \;,
\end{displaymath}
gdje su $\alpha$ i $\beta$  koeficijenti koje treba odrediti.
Djelovanjem s $J_-$ na obje strane imamo:
\begin{equation}
\begin{split}
0 &= \alpha\ket{-\fhalf, -\fhalf} + \beta\ket{-\fhalf, -\fhalf} + 0 + 0\\
&= (\alpha+\beta)\ket{-\fhalf, -\fhalf} \imp \beta=-\alpha
\end{split}
\end{equation}
$\imp$
\begin{displaymath}
   \ket{0,0}=\alpha\left(\ket{\fhalf, -\fhalf}-\ket{-\fhalf, \fhalf}
\right)
\end{displaymath}
Normalizacija i izbor faze daju $\alpha=1/\sqrt{2}$ tj.
\begin{displaymath}
 C^{0 0}_{\fhalf \fhalf \fhalf -\fhalf}=
 -C^{0 0}_{\fhalf -\fhalf \fhalf \fhalf}= \frac{1}{\sqrt{2}}.
\end{displaymath}

- Svi ostali koeficijenti su nula.

- Za CG-koeficijente postoje tablice i računalni programi. Za
  generalnu formulu vidi Hamermesh 9-8.

Dosad rečeno nam opisuje ponašanje proizvoljnog kvantnomehaničkog
stanja pri rotaciji:
\begin{itemize}
\item $\ket{j, m}$ stanja se transformiraju pod IRREPsima - $D$-funkcije
\item složenija stanja reduciramo pomoću Clebsch-Gordanovih koeficijenata
\end{itemize}


\section{Tenzorski operatori i Wigner-Eckartov teorem}
\label{sect:tenzorskioperatori}

U prošla dva odjeljka smo naučili rotirati kvantnomehanička stanja.
Da bismo potpuno ovladali primjenama rotacijske simetrije u
kvantnoj mehanici
potrebno je znati i kako se pri rotacijama ponašaju kvantnomehanički
operatori.

\subsubsection{Primjer: Skalarni operator $S$}

 - $D(R) S D(R)^{-1} = S$ --- invarijantan na rotacije (po definiciji)

\begin{equation}
\begin{split}
 e^{-i\vec{J}\cdot\unitn \phi/\hbar} S e^{-i\vec{J}\cdot\unitn \phi/\hbar}
  &= \left(1-\frac{i}{\hbar}\vec{J}\cdot\unitn \phi\right)S
\left(1+\frac{i}{\hbar}\vec{J}\cdot\unitn \phi\right) + O(\phi^2) \\
&= S - \frac{i}{\hbar}(\vec{J}S - S\vec{J})\cdot\unitn\phi \\
&= S \qquad \imp \qquad [\vec{J}, S] = 0
\end{split}
\end{equation}
(Analogno kao što očuvani operatori tj. operatori invarijantni na
translacije u vremenu komutiraju s hamiltonijanom --- generatorom translacija
u vremenu.)

Matrični elementi od $S$? (Obično je najzanimljivije znati neke informacije
o matričnim elementima operatora.)
\begin{displaymath}
   \bra{j' m'}S\ket{j m} = \text{?}
\end{displaymath}
\begin{align*}
\bra{j' m'}S\vec{J}^2 \ket{jm} &= \bra{j' m'}\vec{J}^2 S\ket{jm} \\
j(j+1)\hbar^2 \bra{j' m'}S \ket{jm} &= j'(j'+1)\hbar^2 \bra{j' m'}S \ket{jm}\\
(j-j')(\underbrace{j+j'+1}_{\neq 0})\bra{j' m'}S \ket{jm} &= 0 \\
    &\imp \bra{j' m'}S \ket{jm} \propto \delta_{jj'}
\end{align*}
Slično (D.Z), 
\begin{displaymath}
            \bra{j' m'}S \ket{jm} \propto \delta_{mm'} \;.
\end{displaymath}
Znači, samo
\begin{displaymath}
               \bra{j m}S \ket{jm} \neq 0 \qquad \qquad \text{Izborno pravilo}
\end{displaymath}
Promotrimo li sada matrične elemente od $SJ_+$ možemo dobiti dodatne
informacije:
\begin{align*}
\bra{j m} S J_+ \ket{j m-1} = \underbrace{\bra{j m} J_+}_{\bra{J_jm}}
S \ket{j m-1} \\
\hbar \sqrt{(j-m+1)(j+m)} \bra{j m}S \ket{jm} &=
\hbar \sqrt{(j+m)(j-m+1)} \bra{j m-1}S \ket{jm-1} \\
 \bra{j m}S \ket{jm} &= \bra{j m-1}S \ket{jm-1}
\end{align*}
Dakle, matrični element $\bra{j m}S \ket{jm}$ ne ovisi o $m$!
Običaj je u tom slučaju pisati
\begin{displaymath}
 \bra{j' m'}S \ket{jm} = \delta_{jj'} \delta_{mm'}
 \underbrace{\bra{j' }|S| \ket{j}}_{\text{tzv. \emph{reducirani
 matrični element}}}
\end{displaymath}
Za dati $j$, umjesto $(2j+1)^2$ matričnih elemenata, dovoljno je
izračunati jedan!

Za poopćenje ovog na složenije operatore treba promatrati operatore koji
se dobro transformiraju na rotacije \secret{tj. za koje znamo komutacijske
relacije s $\vec{J}$} --- \emph{tenzorski operatori}.
Podsjetimo se: Stanja koja se ``dobro'' transformiraju pri rotacijama
su $\ket{jm}$ sa svojstvom
\begin{displaymath}
 D(r) \ket{jm} = \sum_{m'} \ket{jm'}\bra{jm'}D(R)\ket{jm}
               = \sum_{m'} D^{j}_{mm'}\ket{jm'}
\end{displaymath}

\begin{definicija}[Ireducibilni sferični tenzorski operator]
Ireducibilni sferični tenzorski operator $T^{(k)}$ ranga $k$, sa
$2k+1$ komponenata $T^{(k)}_q$, $q=-k,-k+1,\ldots,k$ je operator
koji zadovoljava
\begin{equation}
 D(R) T^{(k)}_q D(R)^{-1} = \sum_{q'=-k}^{k} D^{(k)}_{q'q}(R)
      T^{(k)}_{q'}
\end{equation}
\end{definicija}


Ekvivalentno, sferične tenzore možemo definirati zahtjevima
\begin{align}
[J_z, T^{(k)}_q ] &= \hbar q T^{(k)}_q \\
[J_\pm, T^{(k)}_q] &= \hbar \sqrt{(k\mp q)(k\pm q +1)} T^{(k)}_{q\pm 1}
\end{align}
(D.Z. pokažite ekvivalentnost ovih definicija. Naputak: Napišite originalnu
definiciju u infinitezimalnom obliku.)

Npr. $J_z$ i $\,\mp (1/\sqrt{2})J_\pm$ formiraju sferični tenzor ranga 1.
(D.Z. pokažite to!)


N.B. "Komponente" sferičnog tenzora su također operatori. Ne dolaziti u
zabunu s komponentama matrica --- brojevima. Ove "komponente" sferičnog
tenzora se naravno također mogu napisati u matričnom obliku i onda one
imaju svoje komponente tj. matrične elemente koji jesu brojevi.

N.B. $D^{(j)}$ nije tenzorski operator ranga $j$.

\begin{teorem}[Wigner-Eckart]
Matrični elementi tenzorskih operatora između svojstvenih stanja momenata
impulsa zadovoljavaju relaciju
\begin{equation}
 \bra{j'm'}T^{(k)}_q \ket{jm} = C^{j'm'}_{jmkq}
\frac{\bra{j'}|T^{(k)}|\ket{j}}{\sqrt{2j+1}} \;,
\end{equation}
gdje je $\bra{j'}|T^{(k)}|\ket{j}$ tzv. \emph{reducirani matrični element}
koji ne ovisi o $m, q$ i $m'$.
\end{teorem}
Za dokaz vidi literaturu.

Posljedica ovog teorema je da izračunavanje jednog jedinog matričnog elementa
(npr. za slučaj $m'=q=m=0$) onda omogućuje trivijalno određivanje svih ostalih
pomoću tablica Clebsch-Gordanovih koeficijenata.

\subsubsection{Primjer: skalarni operator}
\begin{displaymath}
S=T^{(0)}_0 \imp \bra{j'm'}|T^{(0)}_0\ket{jm} = C^{j'm'}_{jm00}
\frac{\bra{j'}|T^{(0)}|\ket{j}}{\sqrt{2j+1}}
\end{displaymath}
pa svojstva Clebsch-Gordanovih koeficijenata automatski daju rezultate
iz uvoda ovog odjeljka:
\begin{itemize}
\item $m+0 = m' \imp m=m'$
\item $|j-0|\leq j' \leq j+0 \imp j=j'$
\end{itemize}

\subsubsection{Primjer: Izborna pravila za dipolno zračenje}

\begin{displaymath}
\text{Amplituda}\propto\bra{n'l'm'}\vec{E}\cdot\vec{x}\ket{nlm} =
\vec{E}\cdot\bra{n'l'm'}\vec{x}\ket{nlm}
\end{displaymath}
$\vec{E}$ --- vanjsko polje (nije operator). $\vec{x}$ --- tenzor ranga
1 (vektor); $x^{(1)}_q$, $x_z \leftrightarrow 
x^{(1)}_0$, $x_{x,y}\leftrightarrow x^{(1)}_{\pm 1}$
Teorem $\imp$
\begin{displaymath}
 \bra{n'l'm'}x^{(1)}_q \ket{nlm} = C^{l'm'}_{lm1q}
\frac{\bra{n'l'}|x^{(1)}|\ket{nl}}{\sqrt{2l+1}} \;,
\end{displaymath}
\begin{itemize}
\item $|l-1|\leq l' \leq l+1 \imp \Delta l = \pm 1,0$
\item $m+q = m' \imp$
  \begin{itemize}
  \item \text{za} $\vec{E} = E\hat{\vec{z}} \qquad m'=m$
  \item \text{za} $\vec{E} = E\hat{\vec{x}},E\hat{\vec{y}} \qquad m'=m\pm 1$
   \end{itemize}
$\imp \Delta m = \pm 1, 0$
\end{itemize}
(Usput, dodatna simetrija koju ima ovaj sustav, simetrija pariteta, zabranjuje
$\delta l = 0$ prijelaze.)

\secret{- Vidi se veza ovog sa zakonima očuvanja momenta impulsa koji
također impliciraju ovo gore.}

\subsection{Veza kartezijevih i sferičnih tenzora}
\label{sect:sfericniVSkartezijevi}

Tenzore smo (definicija \ref{def:tenzor}) definirali kao objekte koji se pri rotaciji
transformiraju kao npr.:
\begin{displaymath}
   T_{ij} \longrightarrow  R_{ii'}R_{jj'}T_{i'j'} \qquad \text{(tenzor ranga 2)}
\end{displaymath}
To su \emph{kartezijevi} tenzori.

\emph{Kartezijevi} tenzori ranga 1 tj. uobičajeni kartezijevi vektori
se također mogu definirati kao trojke operatora koje zadovoljavaju
komutacijske relacije
\begin{equation}
 [J_i, V_j] = i\hbar \epsilon_{ijk} V_k
\end{equation}

Problem kartezijevih tenzora je da nisu ireducibilni. To se ne
vidi na tenzorima ranga 0 ili 1 koji jesu ireducibilni i jednaki
su odgovarajućim sferičnim tenzorima do na drugačije kombinacije
komponenata. Konkretno:
\begin{equation}
T^{(0)}_0 \: \text{\small (sferični tenzor ranga 0 tj. skalar)}
 = T      \: \text{\small (kartezijev tenzor ranga 0 tj. skalar)}
\end{equation}
Za operatore ranga 1 (vektore) imamo već spomenute relacije
\begin{equation}
T^{(1)}_0 = T_z \;, \quad T^{(1)}_{\pm 1} = \mp \frac{1}{\sqrt{2}}
(T_x \pm i T_y) \;,
\label{eq:kartvektor}
\end{equation}
i rotacije očito miješaju sve tri komponente kartezijevog tenzora
$\vec{T}$. Međutim, kartezijev tenzor ranga 2, $T_{ij}$, više nije
ireducibilan; njegovih 9 komponenti se može organizirati u
kombinacije koje se međusobno ne miješaju pri rotacijama. 
Kao prvo, trag tenzora $\mathrm{Tr} T = T_{ii}$ je invarijantan
na rotacije:
\begin{align}
 T_{ii} = \delta_{ij} T_{ij} \longrightarrow
\delta_{ij} R_{i i'} R_{j j'} T_{i' j'}& = (R_{j i'} R_{j j'}) T_{i' j'}
= (R^{\mathrm{T}} R)_{i' j'} T_{i' j'} \\
& = \delta_{i' j'} T_{i' j'} = T_{i' i'} \;,
\end{align}
gdje smo u predzadnjem koraku upotrijebili svojstvo ortogonalnosti
matrica rotacije. Dakle, trag 
 $T_{xx} + T_{yy} + T_{zz}$ je skalar tj. tenzor ranga 0.
Nadalje, kartezijev tenzor $T_{ij}$, baš kao i svaku matricu, možemo
rastaviti na antisimetrični i simetrični dio:
\begin{equation}
  T_{ij} = \fhalf (T_{ij} - T_{ji}) + \fhalf (T_{ij} + T_{ji}) \;.
\label{eq:SplA}
\end{equation}
Lako je uočiti da rotacije ne miješaju ove dvije komponente:
rotacija (anti)si\-met\-ri\-čne matrice daje (anti)simetričnu matricu.
(Vidi zadatak \thechapter.\ref{zad:antisim}).
Tako tri nezavisne antisimetrične kombinacije, $(T_{xy}-T_{yx})/2$,
$(T_{yz}-T_{zy})/2$ i $(T_{zx} - T_{xz})/2$, čine sferični tenzor
prvog ranga tj. vektor, što se vidi po broju komponenata i po
činjenici da je ovaj tročlani skup ireducibilan tj. rotacije
pretvaraju jedan element u drugi.
Simetrični dio ima šest komponenata, $T_{xx}$, $T_{yy}$, $T_{zz}$,
$(T_{xy}+T_{yx})/2$, $(T_{yz}+T_{zy})/2$ i $(T_{zx} + T_{xz})/2$, 
ali treba uočiti da je zbroj prvih triju jednak tragu za kojeg
znamo da je posebno ireducibilan 
pa ga treba eliminirati da bi se dobilo pet
nezavisnih komponenata koje čine sferični tenzor drugog ranga.
Dakle, konačni rastav kartezijevog tenzora drugog ranga na
ireducibilne tenzore ranga 0, 1 i 2 je
\begin{equation}
T_{ij} = \frac{1}{3}\delta_{ij}T_{kk} +
  \fhalf (T_{ij} - T_{ji}) + 
 \bigg[\fhalf (T_{ij} + T_{ji}) - \frac{1}{3}\delta_{ij}T_{kk} \bigg] \;,
\end{equation}
gdje faktor $1/3$ osigurava da zadnji član bude traga nula.
(Cf. Sakurai odjeljak 3.10)

\section{Degeneracija nivoa vodikovog atoma i SO(4) simetrija}
\label{so4}


Vidjeli smo u odjeljku \ref{degeneracija} da postojanje operatora
simetrije $\{U(g) \td g\in G \}$, $[U(g), H] = 0$ povlači degeneraciju
energijskih nivoa koji odgovaraju stanjima $\{U(g)\ket{\alpha} \td g\in
G \}$. Tamo je to bilo ilustrirano na primjeru konačnih grupa. Zanimljiv
primjer degeneracije kao posljedice kontinuirane sferne simetrije su
nivoi u vodikovom atomu. Hamiltonijan (u CGS sustavu jedinica)
\begin{equation}
    H = \frac{p^2}{2m} - \frac{e^2}{r} \;.
\label{hatom}
\end{equation}
je rotacijski simetričan i kao posljedica toga komutira s operatorom
rotacija
\begin{equation}
     [H, D(\vec{n},\phi)] = 0 \;.
\end{equation}
To povlači da svih $2l+1$ stanja date ireducibilne reprezentacije
grupe rotacija s kvantnim brojem $l$
\begin{equation}
   \big\{\ket{nlm} \td m\in\{-l, -l+1, ..., l \}\big\} = 
   \big\{ D(g)\ket{nlm} \td g \in SO(3) \big\}
\end{equation}
ima istu energiju, kako smo već spominjali u odjeljku \ref{reprezentacije}.

Međutim, poznato je da energije stanja vodikovog atoma ne
ovise ni o kvantnom broju $l$ i da $n^2$ stanja
\begin{equation}
\big\{\ket{nlm} \td l=\{0, 1, \ldots, n-1\}; m\in\{-l, -l+1, \ldots, l \}\big\} 
\end{equation}
imaju istu energiju\footnote{Ovdje radimo s pojednostavljenim modelom vodikovog atoma
opisanim hamiltonijanom (\protect\ref{hatom}). U realnom vodikovom atomu
postoje i dodatni članovi u hamiltonijanu, poput člana interakcije spina
i orbite, koji razbijaju ovu degeneraciju i čine da energije nivoa ovise
i o orbitalnom kvantnom broju $l$ --- tzv. \emph{fina struktura} vodikovog
spektra.}
\begin{equation}
    E_n = - \frac{e^4 m}{2 \hbar^2 n^2}  \;.
\end{equation}
Dakle, umjesto $2l+1$-struke degeneracije koju očekujemo kao posljedicu
rotacijske simetrije imamo veću, $n^2$-struku degeneraciju. Ovakva
situacija obično znači da sustav ima veću simetriju nego što smo
originalno očekivali. Koju to simetriju, pored rotacijske, ima
sustav opisan hamiltonijanom (\ref{hatom})?
Da bismo istražili to pitanje vratit ćemo se u područje klasične
fizike gdje se javlja slična situacija u problemu dva tijela čiji
je hamiltonijan
\begin{equation}
    H = \frac{p^2}{2m} - \frac{e_{M}^2}{r} \quad ; \quad 
 e_{M}^2 \equiv G M m \;.
\end{equation}
matematički ekvivalentan onom vodikovog atoma.

Kod problema dva tijela rotacijska simetrija i njoj odgovarajući zakon
očuvanja momenta impulsa ($\vec{L}$=const.) manifestiraju se kroz
činjenicu da putanja sustava (elipsa) ostaje cijelo vrijeme u istoj
ravnini.  Međutim, i ovdje se javlja zanimljiva dodatna simetrija ---
putanja je zatvorena elipsa i
smjer perihela elipse je također konstantan\footnote{Ovo
vrijedi u klasičnoj Newtonovoj teoriji gravitacije. Poznato je da
Einsteinova teorija gravitacije korigira ovaj rezultat i da položaj perihela
elipse nije konstantan već vrlo polako precesira.}.
Odgovarajući očuvani vektor je tzv. \emph{Laplace-Runge-Lenzov}
vektor
\begin{equation}
  \vec{M} = \vec{v}\times\vec{L} - \frac{e_{M}^2}{r}\vec{r} \;.
\label{lrl}
\end{equation}
Njegovo očuvanje ($\vec{M}$=const.), slijedi iz drugog Newtonovog
zakona uz malo elementarne vektorske algebre.
Prije svega primijetimo da je
\begin{displaymath}
 \frac{\dd \hat{\vec{r}}}{\dd t} = \frac{\dd}{\dd t} \frac{\vec{r}}{r} =
 \frac{r \vec{v} - \vec{r}
\frac{\vec{r}\cdot\vec{v}}{r}}{r^2} \;,
\end{displaymath}
gdje smo iskoristili $\vec{r}\cdot\vec{v} = \dd (\vec{r}\cdot\vec{r})/2\dd t
= r dr/dt $.

Sad deriviramo (\ref{lrl}) po vremenu, uz korištenje 
($\dd \vec{v}/ \dd t = \vec{F}/m = - (e_{M}^2 \vec{r})/(mr^3)$):
\begin{displaymath}
\begin{split}
\frac{\dd \vec{M}}{\dd t} &=
\frac{\dd \vec{v}}{\dd t} \times \vec{L} + \vec{v}\times
\underbrace{\frac{\dd \vec{L}}{\dd t}}_{=0} - e_{M}^2 \frac{\dd \hat{\vec{r}}}
{\dd t} \\
&= -\frac{e_{M}^2}{mr^3} \underbrace{\vec{r}\times\vec{L}}_{(\vec{r}\cdot 
\vec{p})\vec{r} - r^2 \vec{p}} -
    \frac{e_{M}^2}{mr^3} \Big[ r^2\vec{p} - \vec{r}(\vec{r}\cdot\vec{p})\Big] \\
&= 0
\end{split}
\end{displaymath}
Primjetite kako je za ovaj izvod ključno da je $\vec{F}\propto 1/r^2$.

Također, uočite da su $\vec{M}$ i $\vec{L}$ okomiti
($\vec{M}\cdot\vec{L}=0$, trivijalno).

Posebno je zanimljivo kako sad nakon što smo identificirali ovaj
dodatni očuvani vektor (\ref{lrl}) možemo lako riješiti problem
dvaju tijela:

\begin{equation}
\begin{split}
 \vec{r}\cdot\vec{M} &= \underbrace{\vec{r}\cdot(\vec{v}\times\vec{L})}_{
\vec{L}\cdot(\vec{r}\cdot\vec{v})} - \frac{e_{M}^2}{r} r^2
= \frac{L^2}{m} - e_{M}^2 r \\
&= r M \cos \theta
\end{split}
\end{equation}
gdje je $\theta$ kut kojeg zatvara $\vec{r}$ prema konstantnom vektoru
$\vec{M}$. Ovu jednadžbu možemo prepisati u obliku
\begin{equation}
\frac{1}{r} = \frac{e_{M}^2 m}{L^2} \left(1+\frac{M}{e_{M}^2} \cos\theta\right)
\end{equation}
što prepoznajemo kao polarnu jednadžbu elipse ekscentriciteta $e=M/e_{M}^2$
(usp. npr. Goldstein, jedn. (3-51)). Dakle riješili smo problem i
dobili trajektoriju sustava bez ikakvog rješavanja diferencijalne
jednadžbe gibanja!

Možemo li sad ova saznanja primjeniti u kvantnoj mehanici na naš
početni problem dodatne degeneracije nivoa vodikovog atoma?
Postoji li kvantnomehanički operator koji bi bio analogon 
Laplace-Runge-Lenzovog vektora (\ref{lrl})? Naivni pokušaj
s operatorom
\begin{equation}
  \vec{M} = \frac{\vec{p}}{m}\times\vec{L} - e^2\frac{\vec{r}}{r} \;.
\end{equation}
ne prolazi jer ovaj operator, zbog 
\begin{displaymath}
    (\vec{p}\times\vec{L})^\dagger = - \vec{L}\times\vec{p} \;,
\end{displaymath}
nije hermitski. Stoga je potrebno definirati kvantni
Laplace-Runge-Lenzov vektor ovako:
\begin{equation}
  \vec{M} = \frac{1}{2m}\big(\vec{p}\times\vec{L}- \vec{L}\times\vec{p}\big) 
- e^2\frac{\vec{r}}{r} \;.
\end{equation}

Uz podosta računa (vidi Greiner\&M\"{u}ller, ex. 14.4--14.8) 
pokazuje se da vrijedi
\begin{align}
 [\vec{M}, H] &= 0 \;, \tag{A} \\
 \vec{L}\cdot\vec{M} &= 0 \;, \tag{B} \\
\vec{M}^2 &= \frac{2 H}{m}(\vec{L}^2 + \hbar^2) + e^4  \;, \tag{C} \\
 [L_i, M_j] &= i\hbar \epsilon_{ijk} M_k \;, \tag{D} \\
[M_i, M_j] &= i \hbar \epsilon_{ijk} \left(- \frac{2H}{m}\right) L_k \tag{E} \;.
\end{align}

(A) kaže da je $\vec{M}$ očuvan i u kvantnomehaničkom slučaju, a (D) da
je $M$ kartezijev vektor u smislu odjeljka \ref{sect:tenzorskioperatori}.
(Klasični analogon jednadžbe (C) je $\vec{M}^2 = 2E\vec{L}^2/m + e_{M}^4$.)

Definirajmo sada operator 
\begin{equation}
\vec{M}' \equiv \left(- \frac{2H}{m}\right)^{-\fhalf} \vec{M} \;.
\end{equation}
Uočite da je izraz u zagradi pozitivno definitan jer je spektar operatora $H$
cijeli negativan buduči da radimo s vezanim stanjima vodikovog atoma.
Ovakav reskalirani operator $\vec{M}'$ zadovoljava sada komutacijske
relacije
\begin{equation}
[M'_i, M'_j] = i \hbar \epsilon_{ijk}  L_k \tag{E'} \;.
\end{equation}
i pomoću njega možemo definirati dva nova operatora
\begin{align}
\vec{I} &= \fhalf \big(\vec{L} + \vec{M}'\big) \\
\vec{K} &= \fhalf \big(\vec{L} - \vec{M}'\big) \;.
\end{align}
(Odnosno $\vec{L} = \vec{I} + \vec{K}$ i
$\vec{M}' =  \vec{I} - \vec{K}$.)
Uporabom relacija (D) i (E') te standardnih komutacijskih
relacija za moment impulsa $\vec{L}$ lako se vidi da ova 
dva operatora zadovoljavaju komutacijske relacije
\begin{align}
[I_i, I_j] &= i \hbar \epsilon_{ijk}  I_k \\
[K_i, K_j] &= i \hbar \epsilon_{ijk}  K_k \\
[I_i, K_j] &= 0
\end{align}
što znači da i $I_i$ i $K_i$ generiraju svaki svoju
algebru grupe SO(3) tj. da je ukupna grupa simetrija vodikovog
atoma SO(3)$\times$SO(3)\footnote{Vrijedi grupni identitet
SO(3)$\times$SO(3) = SO(4). Otud naslov ovog odjeljka. 
SO(4) je grupa generirana s 6
generatora $L_{\mu\nu}=x_\mu p_\nu - x_\nu p_\mu$; $\mu, \nu = 1,2,3,4$;
$[x_\mu, p_\nu] = i\hbar \delta_{\mu\nu}$. Definiramo li
$L_i = \fhalf \epsilon_{ijk} L_{jk}$ i $M'_i = L_{4i}$, $i,j,k = 1,2,3$,
dobijemo gornje komutacijske relacije.}

Sad možemo upotrijebiti naše poznavanje svojstava reprezentacija
grupe SO(3). Činjenica da $I_i$ i $K_i$ zadovoljavaju
iste komutacijske relacije kao i moment impulsa povlači da
su svojstvene vrijednosti od $\vec{I}^2$ $i(i+1)\hbar^2$,
a od $\vec{K}^2$ $k(k+1)\hbar^2$, uz 
$i, k = 0, 1/2, 1, 3/2, ...$.

Dodatni uvjet na ove svojstvene vrijednosti je relacija (B) koja
daje:
\begin{equation}
0 = \vec{L}\cdot\vec{M}' = (\vec{I})^2 - (\vec{K})^2
  = i(i +1)\hbar^2 - k(k +1)\hbar^2 \;,
\label{j1j2}
\end{equation}
odnosno $i = k\equiv j$. (Alternativno rješenje jednadžbe (\ref{j1j2})
$i = - (k +1 )$ nije moguće jer daje negativne svojstvene vrijednosti
za $i$ ili $k$.)

Prepišimo sada jednadžbu (C) pomoću novih operatora. Množenjem (C)
s $(-2H/m)^{-1}$ dobije se

\begin{equation}
   \vec{M'}^2 = -(\vec{L}^2 + \hbar^2) - \frac{me^4}{2H}
\end{equation}

ili

\begin{align}
 -\frac{me^4}{2}H^{-1} &= \vec{M'}^{2} + \vec{L}^2 + \hbar^2 \\
&=\big[\vec{I} - \vec{K}\big]^2 + 
\big[\vec{I} + \vec{K}\big]^2  + \hbar^2\\
&=2\big[\vec{I}^2 + \vec{K}^2 \big]^2 + \hbar^2  \;.
\end{align}

To znači da za svojstvena stanja energije i momenta impulsa vrijedi
\begin{equation}
-\frac{me^4}{2}E^{-1} = 2\big[i(i +1)\hbar^2 + k(k+1)\hbar^2 
\big]^2 + \hbar^2   = \hbar^2 \big[ 4j(j+1) +1 \big] = \hbar^2 (2j+1)^2
\end{equation}
što daje spektar
\begin{equation}
   E = - \frac{m e^4}{2\hbar^2 (2j+1)^2} \;.
\end{equation}
Uvedemo li kvantni broj $n\equiv 2j+1$, onda iz $j=0,1/2, 1,\ldots$
slijedi $n=1,2,3,\ldots$ i ovo prepoznajemo kao ispravni izraz
za spektar vodikovog atoma.

Kako je ``pravi'' angularni moment $\vec{L}=\vec{I} +
\vec{K}$ njegove svojstvene vrijednosti dane su pravilima
za zbrajanje momenata impulsa tj.
\begin{equation}
    |j - j| \leq l \leq j + j
\end{equation}
tj. $l \leq 2j \leq n-1$, što je poznati rezultat. Također, $l$
automatski ispada cjelobrojan. Dakle i kvantnomehanički spektar
vodikovog atoma smo našli bez rješavanja Schr\"{o}dingerove
jednadžbe!

\secret{Usporediti algebarski i pristup preko diferencijalnih
jednadžbi. Heisenberg vs. Schr\"{o}dinger. cf. priča u Futuri.}

\textbf{Degeneracija.} Da se još jednom uvjerimo da je degeneracija
svakog nivoa $n^2$-struka  možemo brojati stanja u bazi koja
dijagonalizira operatore $H$, $\vec{L}^2$ i $L_z$, što je
uobičanjena $\ket{nlm}$ baza. Tu za dati $n$ imamo sve skupa
\begin{equation}
 \sum_{l=0}^{n-1} (2l+1) =\frac{n}{2}\big(1+2(n-1)+1)\big) = n^2
\end{equation}
degeneriranih stanja.

Alternativna baza je ona koja dijagoalizira operatore
$\vec{I}^2=\vec{K}^2$, $I_z$ i $K_z$ čije vektore možemo
označiti kao $\ket{j,m_i, m_k}$ i koja daje degeneraciju
\begin{equation}
    \sum_{m_i=-j}^{j}\sum_{m_k=-j}^{j} 1 = (2j+1)^2 = n^2 \;,
\end{equation}
u skladu s prvim računom.

\emph{Literatura}: Schiff(68), Sect. 30, Greiner\&M\"{u}ller, Jones,
\secret{L.C. Biedenhorn, The ``Sommerfeld Puzzle'' Revisited and
Resolved, Foundation of Physics, \textbf{13} (1983) 13-34}

\subsection*{Zadaci}

\begin{enumerate}[{6}.1]

\item
Odredite $\bra{j, m'}J_{i} \ket{j, m} \equiv (J_i)_{m' m}$ za 
$j=1/2$ i $j=1$

\item
Pokažite da vrijedi: 
\begin{equation}
\bra{j, m'} D^{(j)}(\phi, \theta, \psi) \ket{j, m} = e^{-im'\phi-im\psi}
\bra{j, m'} e^{-i J_y \theta /\hbar} \ket{j, m} \;,
\end{equation}
gdje su $\phi$, $\theta$ i $\psi$ tri Eulerova kuta,
te odredite eksplicitno ove matrične elemente za $j=1/2$.

\item
Izrazite stanje $\ket{\vec{J}\cdot\unitn, +}$ definirano svojstvom
\begin{displaymath}
    \vec{J}\cdot\unitn \ket{\vec{J}\cdot\unitn, +} =
  \frac{\hbar}{2} \ket{\vec{J}\cdot\unitn, +}
\end{displaymath}
preko stanja baze $\ket{j=1/2, m=\pm 1/2}$.

\item
Koja je vjerojatnost da mjerenje projekcije spina na $z$-os za stanje iz
prošlog zadatka da rezultat $\hbar/2$?

\item 
Neka je $\ket{\unitn}$ svojstveno stanje operatora usmjerenja u 3D
prostoru. Uočite da je
\begin{displaymath}
     \bra{n}lm\rangle = Y^{m}_l (\unitn) = Y^{m}_l (\theta, \phi) \;.
\end{displaymath}
Promatrajući operator $D(\phi,\theta,0)$ koji rotira $\ket{\hat{\vec{z}}}$
u $\ket{\unitn}$ pokažite da vrijedi
\begin{displaymath}
    D^{l}_{m0}(\phi,\theta,0)=\sqrt{\frac{4\pi}{2l+1}}
 Y^{m^*}_l (\theta, \phi)\;.
\end{displaymath}

\item
Promotrite stanje $\ket{\text{rot}_y(\beta)}$, dobiveno rotacijom stanja
$\ket{l=2, m=0}$ za kut $\beta$ oko $y$-osi. Pronađite vjerojatnosti da
mjerenje projekcije momenta impulsa na $z$-os da vrijednosti
$m'=0,\pm 1, \pm 2$.

\item Čestica spina $1/2$ je u $D$ stanju orbitalnog momenta impulsa
($l=2$). Koja su moguća stanja ukupnog momenta impulsa? Koje su energije
tih stanja ako je hamiltonijan 
\[ H = A + B \vec{L}\cdot\vec{S} + C \vec{L}^2 \]
gdje su $A$, $B$ i $C$ poznate konstante?

\item
Izrazite komponente sferičnog vektora $r^{(1)} = (r^{(1)}_{-1}, r^{(1)}_{0},
r^{(1)}_{1})$ preko kartezijevih komponenata $r_x, r_y, r_z$ tj. izvedite
relaciju (\ref{eq:kartvektor}).

\item Neka je poznato da za sferični vektorski operator $\sigma$ vrijedi
\[  \bra{\fhalf, \fhalf} \sigma_0 \ket{\fhalf, \fhalf} = 1 \;. \]
Izračunajte sve ostale matrične elemente ovog operatora između stanja
s $j = 1/2$.

\item Pronađite izborna pravila za zračenje u kristalu s $C_{3v}$
simetrijom za zračenje polarizirano (a) duž $z$-osi i (b) duž
$x$ ili $y$ osi. \secret{(Hammermesh)}

\item 
Tri matrice, $M_x$, $M_y$, i $M_z$, svaka $256\times 256$, zadovoljavaju
komutacijske relacije $[M_i, M_j] = i \sum_{k} \epsilon_{ijk} M_k$.
Svojstvene vrijednosti od $M_z$ su:
\begin{center}
\begin{tabular}[h]{l|ccccccccc}
\hline
svojstvena vrijednost & 2 & -2 & 3/2 & -3/2 & 1 & -1 & 1/2 & -1/2 & 0 \\ \hline
koliko puta se pojavljuje & 1 & 1 & 8 & 8 & 28 & 28 & 56 & 56 & 70 \\ \hline
\end{tabular}
\end{center}
Navedite svojstvene vrijednosti matrice $M^2 \equiv M_{x}^2 + M_{y}^2 + M_{z}^2$
i broj njihovih pojavljivanja.

\item Pokažite da se sve komponente tenzorskog operatora 
$T^{(J)}_M$ mogu dobiti iz $T^{(J)}_J$ uzastopnom primjenom operatora $J_-$:
\[ T^{(J)}_M = C(J,M) [J_-, [J_-, \dots [J_-, T^{(J)}_J]. . .]] \;. \]
Koliki je $C(J,M)$?

\item \label{zad:antisim} Pokažite da se simetrični i antisimetrični
dijelovi tenzora drugog ranga ne miješaju pri rotacijama.

\end{enumerate}
