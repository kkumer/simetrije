\fancypagestyle{plain}{%
    \fancyhf{}
    \fancyhead[RO,LE]{\bfseries \thepage}
    \fancyhead[CO]{\rightmark}
    \fancyhead[CE]{\leftmark}
    \renewcommand{\headrulewidth}{0.4pt}}

\chapter{Rotacije i moment impulsa u kvantnoj mehanici}
\label{ch:rotacije}

Teorija reprezentacija općenitih Liejevih grupa je zahtjevno
gradivo u koje se nećemo ovdje upuštati. U preostalim
poglavljima ove knjige posvetit ćemo se samo
reprezentacijama nekoliko Liejevih grupa posebno važnih
za fiziku. Prva od njih će biti grupa rotacija u trodimenzionalnom
euklidskom prostoru \SO{3}, koju ćemo  zvati
jednostavno "grupa rotacija". Grupa rotacija je od ogromne
važnosti za cijelu fiziku i radu s njenim reprezentacijama
se tipično posvećuju središnja poglavlja udžbenika kvantne mehanike,
obično pod naslovom "teorija momenta impulsa".
Tako ovo prvo poglavlje ima određeni preklop s tim udžbenicima, ali
izlaganje će naglasiti upravo grupno-teorijske aspekte.

Kasnije ćemo se na dobivene rezultate moći dosta osloniti i
prilikom razmatranja složenijih grupa poput \SU{N} i, posebice,
Lorentzove grupe \SO{1, 3} kojoj je grupa rotacija podgrupa.

\section{Ireducibilne reprezentacije grupe \SO{2}}

Promotrimo za zagrijavanje jednostavniji slučaj grupe \SO{2}
rotacija u ravnini. Kao i puna grupa rotacija, \SO{2} je
\emph{kompaktna} (njen grupni prostor je kružnica).
Teorija reprezentacija za kompaktne grupe je osjetno lakša
 od one za nekompaktne grupe.
Ključna prednost kompaktnosti je činjenica da su kontinuirane funkcije
na kompaktnom skupu integrabilne pa 
velik broj teorema o konačnim grupama vrijedi i za kompaktne Liejeve
grupe, uz zamjenu u dokazima i iskazima:
\begin{equation}
    \frac{1}{n} \sum_{g} \longrightarrow \int \rmd g \;,
\end{equation}
gdje mjeru integrala valja izabrati tako da vrijedi
\begin{equation}
  \int \rmd g f(g) = \int \rmd g f(hg) = \int \rmd g f(gh) \,,
\end{equation}
$\forall h \in G$.
Integraciju koja ima to svojstvo zovemo \emph{invarijantna}.  
Invarijantna integracija nam omogućuje zamjenu
varijabli kakva nam treba za dokaze teorema o reprezentacijama
(vidi npr. (\ref{eq:gshift})).
Za grupe koje ćemo ovdje sretati uvijek je moguće je izabrati
mjeru integracije tako da ona bude invarijantna.

Tako možemo "preuzeti" iz teorije reprezentacija konačnih grupa
teorem da su sve reprezentacije ekvivalentne unitarnima. (To nam je važno
jer je unitarnost željeno svojstvo transformacija kvantnomehaničkih stanja.)
Specijalno, sve \emph{ireducibilne} reprezentacije grupe \SO{2} su ekvivalentne unitarnima
pa se, kao što smo to radili i kod konačnih grupa, smijemo ograničiti na
unitarne reprezentacije bez gubitka općenitosti. 

Nadalje, kako je \SO{2} Abelova, sve ireducibilne reprezentacije
su jednodimenzionalne (kako smo ustanovili u zadatku \ref{zad:1Dabelrep}),
što uz unitarnost ($uu^\dagger = uu^* = |u|^2=1$) znači da su operatori koji
su elementi reprezentacija svakako oblika
\begin{equation}
D(\phi)=e ^{-i f(\phi)} \;, \qquad f(\phi) \in \mathbb{R} \;.
\end{equation}
Kompozicija dvaju rotacija u ravnini za kuteve $\phi_1$ i $\phi_2$ je rotacija za
kut $\phi_1 + \phi_2$ pa zahtjev da reprezentacija bude
homomorfna grupi povlači
\begin{equation}
   D(\phi_2) D(\phi_1) = D(\phi_1+\phi_2) = e^{-i f(\phi_1 + \phi_2)}\;.
\end{equation}
S druge strane svojstvo eksponencijalne funkcije daje
\begin{equation}
   D(\phi_2) D(\phi_1) = e^{-i f(\phi_2)} e^{-i f(\phi_1)} =  e^{-i f(\phi_1) - i f(\phi_2)}\;,
\end{equation}
pa slijedi
\begin{equation}
   f(\phi_1)+f(\phi_2) = f(\phi_1+\phi_2) \;,
\end{equation}
što znači da je $f(\phi)=m\phi$, $m\in\mathbb{R}$, odnosno
reprezentacija je skup operatora
\begin{equation}
    \Gamma^{(m)} = \{ D^{(m)}(\phi) = e^{-i m \phi} \td  \phi \in [0, 2\pi)\}\,,
\end{equation}
s fiksnim $m$.  Na kraju, zahtjev periodičnosti rotacija daje
\begin{equation}
D^{(m)}(\phi)=D^{(m)}(\phi + 2\pi) \;,
\end{equation}
iz čega odmah slijedi da $m$ mora biti cijeli broj.
Reprezentacije koje imaju različitu vrijednost $m$ su očito
neekvivalentne (transformacija sličnosti (\ref{eq:ekvivrep})
u jednodimenzionalnom slučaju ne mijenja operator).
Dakle, grupa \SO{2} ima prebrojivo beskonačno ireducibilnih
reprezentacija i pogodno
je $m \in \mathbb{Z}$ koristiti kao oznaku (labelu) reprezentacije.

Trag jednodimenzionalnog operatora je jednak njemu samome pa su
karakteri 
\begin{equation}
\chi^{(m)}(\phi)=D^{(m)}(\phi)=e^{-\rmis m \phi} \,,
\end{equation}
i oni zadovoljavaju relaciju ortogonalnosti
\begin{displaymath}
(\chi^{(m)}, \chi^{(m')})=\int_{0}^{2\pi}\frac{\rmd \phi}{2\pi}\,
 e^{\rmis m \phi}e^{-\rmis m' \phi} = \delta_{mm'} \,,
\end{displaymath}
a čitaoc će se lako uvjeriti da je ovakva integracija invarijantna u gornjem smislu.

Kao i kod konačnih grupa možemo rastaviti direktni produkt dviju
ireducibilnih reprezentacija na Clebsch-Gordanov direktni zbroj
\begin{displaymath}
  \Gamma^{(m)}\otimes\Gamma^{(n)} =
  \sum \oplus\: a_{k}\Gamma^{(k)}  \,,
\end{displaymath}
gdje su koeficijenti dani skalarnim produktom odgovarajućih karaktera,
koji je ovdje integral
\begin{eqnarray*}
a_{k} & = & (\chi^{(k)}, \chi^{(m)}\chi^{(n)}) \\
      & = & \int_{0}^{2\pi}\frac{\rmd \phi}{2\pi}\,
    e^{\rmis k \phi}e^{-\rmis m \phi}e^{-\rmis n \phi} = \delta_{k,m+n} \,,
\end{eqnarray*}
odnosno
\begin{displaymath}
  \Gamma^{(m)}\otimes\Gamma^{(n)} =
  \sum_{k} \oplus\: \delta_{k,m+n}\Gamma^{(k)} = \Gamma^{(m+n)} \;.
\end{displaymath}
To da je direktan produkt jednodimenzionalnih ireducibilnih reprezentacija svakako
ireducibilan smo ustanovili već ranije u zadatku \ref{zad:1D1D1D}, a sad vidimo i koja je
to točno rezultirajuća reprezentacija. Fizikalno, poznato je da je generator
rotacija operator momenta impulsa, tako da je $m$ iznos momenta impulsa,
u jedinicama $\hbar$, fizikalnog sustava koji se pri rotacijama transformira
u skladu s reprezentacijom $\Gamma^{(m)}$. Direktan produkt $\Gamma^{(m)}\otimes\Gamma^{(n)}$
je reprezentacija grupe rotacija na sustavu dobivenom \emph{združivanjem} dvaju sustava 
s momentima impulsa $m$ i $n$. Jasno je da će združeni sustav imati moment impulsa $m+n$.
No, kako znamo iz kvantne mehanike i kako ćemo vidjeti u sljedećem odjeljku, ova
jednostavna matematika ne vrijedi za punu grupu rotacija u trodimenzionalnom prostoru.


Kako su sve ireducibilne reprezentacije jednodimenzionalne,
dobro poznata dvodimenzionalna reprezentacija rotacija u ravnini
\begin{equation}        
 D_{\rm 2D} (\phi) = \left( 
\begin{array}{cc}        
\cos\phi & -\sin\phi \\
\sin\phi & \cos\phi
\end{array}                
\right) \,, 
\end{equation}
je svakako reducibilna. Kao i gore, redukciju provodimo
poznavajući karaktere reprezentacije
\begin{equation}
    \chi_{\rm 2D}=2\cos\phi \,.
\end{equation}
Koeficijenti Clebsch-Gordanovog razvoja su
\begin{eqnarray*}
a_k & = & (\chi^{(k)}, \chi_{\rm 2D}) \\
    & = & \int_{0}^{2\pi}\frac{\rmd \phi}{2\pi}\,
   e^{\rmis k \phi} 2\cos\phi  \\
 & = & \int_{0}^{2\pi}\frac{\rmd \phi}{2\pi}\,
  e^{\rmis k \phi} ( e^{\rmis  \phi} + e^{-\rmis  \phi}) \\
 & = & \delta_{k,-1} + \delta_{k, 1} \;.
\end{eqnarray*}
To znači da postoji $S$ (pronađite ga!) takav da je
\begin{equation}
  S D_{\rm 2D} (\phi) S^{-1} = \left(
\begin{array}{cc}        
e^{\rmis  \phi} &  0 \\
0 & e^{- \rmis  \phi}
\end{array}                
\right) \,.
\end{equation}
Obični ravninski vektori $(x, y)$ nisu svojstveni vektori
rotacija, što je očito jer ih rotacije mijenjaju. No, "cirkularno
polarizirane" funkcije u ravnini poput $x + \rmi y$ to mogu biti.


\section{Ireducibilne reprezentacije algebre \soAlg{3} = \suAlg{2}}

Puna grupa rotacija \SO{3} nije Abelova i ireducibilne
reprezentacije više nisu nužno jednodimenzionalne pa njihova konstrukcija nije
više tako laka kao za \SO{2}.
Oslonit ćemo se na spoznaje iz prošlog poglavlja i
postupak će nam biti da prvo konstruiramo
reprezentacije \soAlg{3} \emph{algebre} grupe rotacija, a onda
ćemo eksponencijacijom dobiti odgovarajuće reprezentacije grupe.
Kako smo vidjeli u primjeru \ref{pr:su2so3Alg}, grupa \SU{2}
ima istu algebru $\suAlg{2}=\soAlg{3}$, tako da ćemo zapravo
konstruirati reprezentacije algebre $\suAlg{2}$. Kako odgovarajuće
grupe \SO{3} i \SU{2} \emph{nisu} izomorfne, morat ćemo pokloniti
nešto pažnje pitanju koje točno reprezentacije dobivamo eksponencijacijom
reprezentacija algebre, ali o tom potom.

Algebru čine tri generatora, $X_1, X_2, X_3$, s komutacijskim relacijama
\begin{equation}
          [X_i, X_j] = \epsilon_{ijk} X_k \;,
\end{equation}
ali mi ćemo odsad nadalje uglavnom raditi s malo drugačijom, "fizičarskom" bazom, definiranom
kao
\begin{equation}
 J_i \equiv \rmi \hbar X_i \,.
    \label{eq:jfromX}
\end{equation}
Naime, kako smo vidjeli u (\ref{eq:Xantisim}) $X_i$ su antisimetrične
matrice pa će ovako definirani $J_i$ biti hermitski, $J_{i}^{\dagger}
= - \rmi \hbar X^\mathsf{T} = J_i$, što znači da će u kvantnomehaničkom
kontekstu odgovarati opservabli, momentu impulsa, i svojstvene vrijednosti
će mu biti realne.
Ovi generatori zadovoljavaju komutacijske relacije
\begin{equation}
          [J_i, J_j] = i \hbar \epsilon_{ijk} J_k \,,
\label{eq:SU2algebra}
\end{equation}
a da bismo dobili operatore grupe, eksponencijacija će biti oblika
\begin{equation}
e^{\phi\unitn\cdot\vec{X}}=e^{(-i/\hbar)\phi\unitn\cdot \vec{J}} \,.
    \label{eq:expJ}
\end{equation}


Kako je već diskutirano na kraju odjeljka \ref{sec:liejevealgebre},
operator $\vec{J}^2 = J_{1}^2 + J_{2}^2 + J_{3}^2$ je Casimirov
tj. komutira sa svim elementima algebre
\begin{equation}
   [\jsq, J_{i}^2]=0\;, \quad i=1,2,3 \,,
\end{equation}
pa iz druge Schurove leme slijedi da $\jsq$
mora biti proporcionalan jediničnom operatoru: $\jsq \propto \Eins$.
Tako je svojstvena vrijednost od $\jsq$ ista za sve vektore neke konkretne 
ireducibilne reprezentacije i koristit ćemo je za njeno označavanje.
Za prvotno definiranje grupe rotacija koristili smo njenu standardnu reprezentaciju
na trodimenzionalnom euklidskom prostoru, no sad nam je cilj pobrojati
\emph{sve} njene ireducibilne reprezentacije na konačno-dimenzionalnim
vektorskim prostorima. Te će reprezentacije biti kandidati za moguće
Hilbertove prostore kvantnomehaničkih stanja pa ćemo s tim u vidu vektore
označavati Diracovom "ket" oznakom $|\alpha\rangle$, vidi Dodatak \ref{sec:qm}.
Da bismo izbjegli svaku mogućnost zabune i istovremeno definirali
naše konvencije, usporedimo sad detaljno reprezentaciju grupe rotacija
na običnom euklidskom prostoru i reprezentacije, koje tek treba
pronači i proučiti, na konačno-dimenzionalnom Hilbertovom prostoru kvantnomehaničkih stanja.

Liejeva \emph{grupa} rotacija je parametrizirana s tri realna broja $(\vec{n}, \phi)$.
Elementi grupe su na trodimenzionalnom euklidskom prostoru reprezentirani $3\times 3$
matricama $D^{(\mathrm{3D})}$
čiji tipični elementi su $\cos(\phi)$ ili $\sin(\phi)$ i koje djeluju
na općeniti vektor $\vec{r}$, izražen preko svojih kartezijevih komponenata $r_i$,
matričnom operacijom
\begin{displaymath}
   D^{(3D)}_{ij}(\unitn, \phi) r_j  \;,
\end{displaymath}
a rezultat je novi vektor u tom prostoru koji je linearna kombinacija
baznih vektora $\hat{x}$, $\hat{y}$ i $\hat{z}$.
Ovdje je $3D$ iskorišteno kao oznaka ove specifične reprezentacije.

Mi smo u potrazi za svim ireducibilnim reprezentacijama iste te grupe rotacija.
Djelovanje operatora reprezentacija na Hilbertovom prostoru je definirano
djelovanjem na vektore baze
\begin{equation}
   D^{(\beta)}(\unitn,\phi) \ket{\beta, m} =
 e^{(-i/\hbar)\vec{J}\cdot\unitn\phi}\ket{\beta, m} \,,
 \label{eq:Drot}
\end{equation}
gdje je sada $\beta$ oznaka reprezentacije, koja bi u načelu trebala
stajati i na generatoru $\vec{J} = \vec{J}^{(\beta)}$, ali je tamo obično samo implicitna.
$ m $ je indeks koji poprima onoliko vrijednosti $m_1$, $m_2$, \ldots, kolika je dimenzionalnost
Hilbertovog prostora. Tako je njegova baza
$\ket{\beta, m_1}$, $\ket{\beta, m_2}$, \ldots i rezultat gornje rotacije će
biti neka linearna kombinacija ovih vektora baze. Kako rekosmo, za oznaku 
reprezentacija $\beta$ koristit ćemo svojstvenu vrijednost operatora $\jsq$,
konkretno
\begin{equation}
  \jsq \ket{\beta, m} = \beta \hbar^2 \ket{\beta, m} \;.
  \label{eq:defbeta}
\end{equation}
Kao Casimirov operator $\jsq$ komutira sa sva tri $J_i$, ali oni \emph{ne}
komutiraju međusobno, pa je izborom baze moguće najviše jedan od njih
dijagonalizirati i standardno se bira $J_{z} = J_{3}$. Tako izabrana baza
će se sastojati od svojstvenih vektora $\jsq$ i $J_z$ pa označimo 
njene vektore tako da vrijedi i
\begin{equation}
  J_z \ket{\beta, m} = m \hbar \ket{\beta, m} \;.
  \label{eq:defm}
\end{equation}
Kako je iz (\ref{eq:expJ}) vidljivo da je dimenzija $J_i$ jednaka
dimenziji $\hbar$ (a time i dimenziji momenta impulsa), izbori
(\ref{eq:defbeta})  i (\ref{eq:defm}) znače da su $\beta$ i $m$
bezdimenzionalni. 

Sada ćemo eksplicitno konstruirati ireducibilne reprezentacije \suAlg{2}, koristeći
isključivo komutacijske relacije (\ref{eq:SU2algebra}).
Prvo definiramo tzv. operatore \emph{podizanja i spuštanja}
\begin{equation}
    J_{\pm} \equiv J_x \pm i J_y \,, 
\end{equation}
koji naravno isto komutiraju s $\jsq$
\begin{equation}
    [\jsq, J_\pm] =0 \,,
\end{equation}
dok su komutacijske relacije s $J_z$, 
\begin{eqnarray*}
 [J_z, J_\pm]& = & [J_z, J_x] \pm i [J_z, J_y] = i\hbar J_y \pm i (-i\hbar J_x)
 \\ & = & \pm \hbar (J_x \pm i J_y) = \pm \hbar J_\pm \,.
\end{eqnarray*}
Slično (pokažite),
\begin{displaymath}
 [J_+, J_-] = 2 \hbar J_z \,.
\end{displaymath}
Djelovanjem ovih operatora dizanja i spuštanja na vektore neke ireducibilne
reprezentacije, ostajemo naravno u toj reprezentaciji
\begin{displaymath}
   \jsq(J_{\pm}\ket{\beta, m})=J_{\pm}\jsq\ket{\beta, m}=\beta\hbar^2
    (J_{\pm}\ket{\beta, m}) \;,
\end{displaymath}
ali svojstvena vrijednost od $J_z$ se mijenja i to točno za $\pm 1$
\begin{equation}
J_z(J_{\pm}\ket{\beta, m})=([J_z,J_{\pm}]+J_{\pm}J_z)\ket{\beta, m}=
\hbar(m\pm 1)J_{\pm}\ket{\beta, m} \;,
\label{jzjpm}
\end{equation}
tj.
\begin{equation}
    J_{\pm}\ket{\beta, m}\propto \ket{\beta, m\pm 1} \;.
    \label{eq:jpmpropto}
\end{equation}


Do kuda  može ići to dizanje i spuštanje tj. koliko različitih vrijednosti
može poprimiti $m$?
Pokazat ćemo da vrijedi $m^2 \leq \beta$.
Uočimo prvo da je $J_{\pm}^{\dagger}=J_{\mp}$, jer $J^{\dagger}_{x,y,z}=J_{x,y,z}$.
Iz toga slijedi
\begin{eqnarray*}
 \frac{1}{2}(J_+ J_{+}^{\dagger} + J_{+}^{\dagger}J_+)& = &
 \frac{1}{2}(J_+ J_{-}+ J_- J_+) \\
& = & \frac{1}{2} \left[ (J_x + i J_y)(J_x - i J_y) + \textrm{h. c.} \right]
\\ & = & \frac{1}{2} \left[ J_{x}^2 + i J_y J_x - i J_x J_y + J_{y}^2
  + J_{x}^2 - i J_y J_x + i J_x J_y + J_{x}^2 \right] \\
& = & J_{x}^2 + J_{y}^2 = \jsq - J_{z}^2
\end{eqnarray*}
Matrični elementi operatora na lijevoj strani ove jednakosti su očito
pozitivni jer je
\begin{displaymath}
\bra{\beta, m}J_{+}^{\dagger}J_+ \ket{\beta, m} =
\bra{J_{+}\beta, m}J_{+}\beta, m\rangle  \geq 0 \,,
\end{displaymath}
i isto za $J_+ J_{+}^{\dagger}$.
Slijedi da je
\begin{displaymath}
\bra{\beta, m}(\jsq - J_{z}^2)\ket{\beta, m}=
\bra{\beta, m}(\beta\hbar^2 - m^2\hbar^2)\ket{\beta, m} =
(\beta - m^2)\hbar^2 \geq 0 \,,
\end{displaymath}
što je i trebalo pokazati.

Dakle, za svaki $\beta$ postoji $m_{\textrm{\scriptsize max}}$ tako da 
\begin{displaymath}
 J_+ \ket{\beta, m_{\textrm{\scriptsize max}}} = 0 \,,
\end{displaymath}
jer to je jedini način da (\ref{jzjpm}) ostane vrijediti.
Odredimo sada vrijednost $m_{\textrm{\scriptsize max}}$ za datu
ireducibilnu reprezentaciju $\beta$. Vrijedi
\begin{eqnarray*}
J_- J_+ =(J_x - i J_y)(J_x - i J_y)
   & = &J_{x}^2 + J_{y}^2 +i (J_x J_y - J_y J_x) \\
 & = & J_{x}^2 + J_{y}^2 - \hbar J_z \\
 & = & \jsq - J_{z}^2 - \hbar J_z \,.
\end{eqnarray*}
No kako je 
\begin{displaymath}
J_- J_+ \ket{\beta, m_{\textrm{\scriptsize max}}} = 0 \,,
\end{displaymath}
onda je i
\begin{displaymath}
 (\jsq - J_{z}^2 - \hbar J_z)\ket{\beta, m_{\textrm{\scriptsize max}}}
 = (\beta\hbar^2 - m_{\textrm{\scriptsize max}}^2 \hbar^2 
                 - m_{\textrm{\scriptsize max}}   \hbar^2 )
\ket{\beta, m_{\textrm{\scriptsize max}}} = 0 \;,
\end{displaymath}
pa kako $\ket{\beta, m_{\textrm{\scriptsize max}}}$ nije nul-vektor
slijedi da je
\begin{displaymath}
    \beta = m_{\textrm{\scriptsize max}} ( m_{\textrm{\scriptsize max}} +1)\;.
\end{displaymath}
Nadalje, iz maločas dokazane činjenice da je $m^2 \le \beta$ slijedi također
da ni spuštanje vrijednosti $m$ operatorom spuštanja ne može ići
unedogled već da mora postojati $m_{\textrm{\scriptsize min}}$
sa svojstvom
\begin{displaymath}
    J_- \ket{\beta, m_{\textrm{\scriptsize min}}} = 0 \;,
\end{displaymath}
iz čega postupkom analognim ovom gore dolazimo do
\begin{displaymath}
    \beta = m_{\textrm{\scriptsize min}} ( m_{\textrm{\scriptsize min}} -1)\;.
\end{displaymath}
Izjednačivši ove dvije vrijednosti za $\beta$ 
\begin{displaymath}
            m_{\textrm{\scriptsize max}} ( m_{\textrm{\scriptsize max}} +1)
= m_{\textrm{\scriptsize min}} ( m_{\textrm{\scriptsize min}} -1)
\end{displaymath}
dobijemo kvadratnu jednadžbu za $m_{\textrm{\scriptsize min}}$ od čija
dva rješenja $m_{\textrm{\scriptsize min}}=m_{\textrm{\scriptsize max}}+1$
i $m_{\textrm{\scriptsize min}} = - m_{\textrm{\scriptsize max}}$ ovo
prvo ne dolazi u obzir jer je $m_{\textrm{\scriptsize max}}$ po pretpostavci
najveća moguća vrijednost za $m$. Tako imamo, uvodeći oznaku
$j\equiv m_{\textrm{\scriptsize max}}$,
\begin{displaymath}
               -j = m_{\textrm{\scriptsize min}} \le m \le 
 m_{\textrm{\scriptsize max}} = j   \;.
\end{displaymath}

Nadalje, kako operatori $J_\pm$ dižu ili spuštaju $m$ točno za 1, mora
biti $m_{\textrm{\scriptsize max}} = m_{\textrm{\scriptsize min}} +n $,
gdje je $n\in \mathbb{N}_{0}$, tj. $j = -j +n$, odnosno $j=n/2$ pa
zaključujemo da su jedine moguće vrijednosti za $j$
\begin{displaymath}
    j \in \{ 0, \frac{1}{2}, 1, \frac{3}{2}, 2, \ldots \}\;.
\end{displaymath}
Kako imamo identitet $\beta = j(j+1)$ možemo umjesto $\beta$ koristiti
i vrijednost $j$ za označavanje ireducibilnih reprezentacija i to ćemo
odsad i činiti. U kontekstu teorije reprezentacija grupa, $j$ se naziva
\emph{spin} i govorimo o \emph{reprezentaciji spina $j$} čak i ukoliko
razmatramo sustav za koji bi orbitalni moment impulsa
bio fizikalno primjereniji pojam od spina.
Dakle mijenjamo oznaku
\begin{displaymath}
  \ket{\beta, m} \longrightarrow \ket{j, m} \;,
\end{displaymath}
gdje $m$ poprima vrijednosti iz skupa
\begin{displaymath}
    m \in \{ -j, -j+1, \ldots, j-1, j\} \,,
\end{displaymath}
što je niz od $2j+1$ vrijednosti, pa zaključujemo da je
reprezentacija spina $j$  $2j+1$-dimenzionalna.
Kako $J_{\pm}$ povezuju svih $2j+1$ vektora $\ket{j, m}$ 
reprezentacija je stvarno ireducibilna.
Tako vidimo da grupa rotacija ima beskonačno ireducibilnih
reprezentacija, po jednu za svaku vrijednost spina $j$.

Vektori baze IRREPa $\ket{j, m}$ zadovoljavaju
\begin{eqnarray}
 \jsq \ket{j, m}& = & \hbar^2 j(j+1)\ket{j, m} \quad \mbox{i} \\
 J_z \ket{j, m} & =  &\hbar m \ket{j, m} \;, \label{eq:jz}
\end{eqnarray}
a sada možemo i kompletirati matrične elemente cijele algebre tako
da točno ustanovimo djelovanje operatora $J_\pm$ na vektore baze
tj. da odredimo konstante proporcionalnosti u(\ref{eq:jpmpropto}):
\begin{equation}
     J_+ \ket{j, m} = c_{jm} \ket{j, m+1} \;.
\end{equation}
Zahvaljujući ortonormiranosti baze imamo
\begin{eqnarray*}
 |c_{jm}|^2 & = & \bra{j, m} J_- J_+ \ket{j, m} \\
            & = & \bra{j, m}(\jsq - J_{z}^2 - \hbar J_z \ket{j, m} \\
& = & \bra{j, m}(\hbar^2 j(j+1) - \hbar^2 m^2 - \hbar^2 m \ket{j, m}  \\
            & = & \hbar^2[j(j+1)-m(m+1)] \\
            & = & \hbar^2 (j-m)(j+m+1) \,.
\end{eqnarray*}
Sama faza od $c_{jm}$ je općenito neodređena i prema tzv. Condon-Shortley konvenciji
odabire se da je $c_{jm}$ realan i pozitivan pa je
\begin{equation}
  J_+ \ket{j, m} = \hbar \sqrt{(j-m)(j+m+1)} \ket{j, m+1} \;.
\label{eq:jplus}
\end{equation}
Slično se dobije
\begin{equation}
  J_- \ket{j, m} = \hbar \sqrt{(j+m)(j-m+1)} \ket{j, m-1} \;.
\label{eq:jminus}
\end{equation}
Relacije (\ref{eq:jz}), (\ref{eq:jplus}) i (\ref{eq:jminus}) 
nam omogućuju da odredimo matrične elemente operatora
momenta impulsa između proizvoljnih stanja i tako kompletno definiraju
te operatore, a time i cijelu $2j+1$-dimenionalnu ireducibilnu reprezentaciju
\suAlg{2} algebre. Tako smo (korištenjem samo komutacijskih relacija!) eksplicitno
konstruirali sve ireducibilne reprezentacije ove algebre. Eksplicitni 
matrični oblici za niže dimenzije mogu se naći u knjigama iz kvantne
mehanike.

Baza vektorskog prostora 
\[\ket{j, m}, m\in\{-j, -j+1, \ldots, j-1, j\}\,,\] 
se obično naziva \emph{multiplet}, a neki autori tako zovu i cijeli
vektorski prostor kojeg ta baza razapinje. Tako imamo
\begin{itemize}
\item  $j=0$ : \emph{singlet}
\item  $j=\frac{1}{2}$ : \emph{dublet}
\item  $j=1$ : \emph{triplet}
\item  i. t. d.
\end{itemize}
gdje ime odražava dimenzionalnost reprezentacije.

Važna je značajka svakog sustava u prirodi 
njegovo ponašanje pri rotacijama. Kvantni sustavi se redovito
transformiraju prema nekoj konkretnoj ireducibilnoj reprezentaciji grupe
rotacija\footnote{Izolirani ili elementarni sustavi praktički uvijek, no
    općenito su moguće i kvantne superpozicije stanja s različitim $j$.}, i
 tada govorimo da je riječ o sustavu \emph{spina $j$}.


\section{Projektivne reprezentacije grupe \SO{3}}
 \label{sec:projektivnerep}

Nakon što smo konstruirali sve ireducibilne reprezentacije \emph{algebre} 
\suAlg{2}=\soAlg{3}, promotrimo sada detaljnije u kojoj mjeri eksponencijacijom
odgovarajućih operatora dobivamo reprezentacije \emph{grupa} \SO{3} i \SU{2}.
Matrični elementi ekponenciranih operatora između standardnih
$\ket{j, m}$ stanja
\begin{equation}
    D^{(j)}_{m'm}(\hat{\vec{z}},\phi)  \equiv  \bra{j,m'} D^{(j)}(\hat{\vec{n}},\phi) \ket{j, m} 
    =\bra{j,m'} e^{(-i/\hbar)\vec{J}\cdot\unitn\phi}\ket{j, m} \,.
    \label{eq:WignerD}
\end{equation}
se nazivaju \emph{Wignerove D-funkcije} i formiraju tzv. \emph{Wignerovu D-matricu}.
Mogu se pronaći tabelirani u specijaliziranoj literaturi, a postoje
i zatvoreni analitički izrazi. No razmotrimo u kojoj mjeri oni
reprezentiraju grupe \SO{3} i \SU{2}. Kako te grupe nisu izomorfne,
reprezentacije koje čine $D^{(j)}$ sigurno ne mogu općenito biti vjerne.

Slučaj $j=0$ je trivijalan, jer je odgovarajuća reprezentacija $D^{(0)}=\Eins$
trivijalna i objekte koji se tako transformiraju (zapravo, ne transformiraju)
nazivamo \emph{skalari}.

Slučajeve $j=1/2$ i $j=1$ smo već sreli. Naime, same grupe nismo definirali
apstraktno već kao matrične grupe. Te matrice možemo interpretirati kao operatore
na vektorskim prostorima i tako one istovremeno čine i reprezentaciju grupe.
$j=1$ je trodimenzionalna reprezentacija i eksplicitnom konstrukcijom
operatora $J_z$, $J_\pm$ putem formula iz prošlog odjeljka, pa onda linearnim
kombiniranjem da bi dobili i $J_x$ i $J_y$ (zadatak \ref{zad:matj}), 
dobivamo upravo generatore iz (\ref{eq:SO3generators}), do na faktor $\rmi\hbar$.
Tako je $j=1$ \emph{definiciona} reprezentacija grupe \SO{3} i njena najmanja
vjerna reprezentacija.
No, ona nije i vjerna reprezentacija grupe \SU{2}, što znamo jer znamo da
se po dva elementa $U$ i $-U$ iz  \SU{2} preslikavaju u isti element 
\SO{3}. To je samo još jedna manifestacija činjenice da su te dvije grupe
iste lokalno (imaju istu algebru), ali ne i globalno.


Zanimljiviji je slučaj $j=1/2$ reprezentacije. Ona je dvodimenzionalna i
istim postupkom kao gore lako se uvjerimo da su $J_i$ do na faktor
dani Paulijevim matricama i eksponencijacijom dobivamo definicionu
vjernu reprezentaciju grupe \SU{2} (vidi zadatak  \ref{zad:svojstvapaulijevih})
\begin{equation}
 e^{-\frac{i}{2}\vec{\sigma}\cdot\unitn \theta} =
\cos \frac{\theta}{2}-i\vec{\sigma}\cdot\unitn \sin\frac{\theta}{2} \,.
\end{equation}
No, je li $j=1/2$ također i reprezentacija grupe rotacija \SO{3}?
Da bismo odgovorili na to promotrimo rotaciju 
stanja spina $j=1/2$ i pozitivne projekcije spina $m=1/2$ za 
kut $2\pi$ oko $z$-osi
\begin{equation}
    e^{(-i/\hbar)\vec{J}\cdot\unitn\phi}\ket{\frac{1}{2}, +\frac{1}{2}} =
    e^{-i\frac{\sigma_3}{2} 2 \pi} \begin{pmatrix} 1 \\ 0 \end{pmatrix} =
    e^{-i \pi}\ket{\frac{1}{2}, +\frac{1}{2}} =
    -\ket{\frac{1}{2}, +\frac{1}{2}} \;.
    \label{eq:ferm2pifaza}
\end{equation}
Vidimo da pri toj rotaciji vektor stanja mijenja predznak tj. ne vraća se u početno stanje.
Tek je operator rotacije za $4\pi$ jednak identiteti.  Čitatelj
se lako može uvjeriti da će takvo ponašanje imati sve polucjelobrojne
reprezentacije.
No, po definiciji, reprezentacija grupe  mora zadovoljavati
općeniti uvjet homomorfizma s grupom, što znači da operatori
reprezentacije grupe \SO{3} moraju zadovoljavati
\begin{equation}
    D(\pi)D(\pi)=D(2\pi)=\Eins \,,
\end{equation}
no kako za operatore $j=1/2$ reprezentacije vrijedi
\begin{equation}
    D^{(1/2)}(\pi)D^{(1/2)}(\pi)=D^{(1/2)}(2\pi)=-\Eins \,,
\end{equation}
zaključujemo da $D^{(1/2)}$
\emph{ne} čine reprezentaciju grupe \SO{3}! Drugi način gledanja
na ovu situaciju je da preslikavanje s grupe \SO{3} na
skup operatora  $D^{(1/2)}$ nije jednoznačno jer svakom elementu
grupe pripadaju dva operatora $D^{(1/2)}(\phi)$ i 
$D^{(1/2)}(\phi + 2\pi) = -D^{(1/2)}(\phi)$.

Premda dakle strogo matematički gledano polucjelobrojne reprezentacije
grupe \SU{2} nisu reprezentacije grupe rotacija \SO{3}, u prirodi
postoje važni sustavi koji se transformiraju prema polucjelobrojnim
reprezentacijama (nazivamo ih fermionski\footnote{%
Jedan od centralnih rezultata kvantne teorije polja je tzv.
\emph{teorem  veze spina i statistike} koji kaže da se sustavi
cjelobrojnog spina ponašaju u skladu s Bose-Einsteinovom
statistikom (i zovemo ih bozonski), a oni polucjelobrojnog spina
u skladu s Fermi-Diracovom statistikom (i zovemo ih fermionski).
Npr. foton je spina 1 i time bozon, a elektron je spina 1/2 i time fermion.})
 i promjena faze vidljiva u (\ref{eq:ferm2pifaza})
se lijepo može vidjeti u eksperimentima.

Tu formalno problematičnu situaciju možemo formalno razriješiti na dva
načina. Prvi je da iskoristimo (fizikalnu, a ne matematičku!)
činjenicu da u kvantnoj mehanici vektori Hilbertovog prostora koji
se razlikuju za fazu i dalje opisuju isto fizikalno stanje.
Tamo onda možemo dopustiti reprezentiranje grupe simetrija i tzv.
\emph{projektivnim reprezentacijama} kod kojih je standardni
uvjet homomorfizma "olabavljen" dopuštanjem razlike u fazi
\begin{displaymath}
   D(g_1)D(g_2)=e^{i \phi(g_1,g_2)}D(g_1 g_2) \;.
   \label{eq:projektivnarep}
\end{displaymath}
Ovdje se nećemo upuštati u općenitu teoriju projektivnih reprezentacija 
(za diskusiju u kontekstu kvantne fizike vidi \cite{Weinberg:1995mt}),
nego ćemo samo izreći nekoliko nama relevantnih rezultata:

\begin{itemize}
    \item O strukturi grupe ovisi hoće li ona imati i projektivnih 
        reprezentacija\footnote{Ovdje se misli na projektivne reprezentacije
        koje su netrivijalne u smislu da se faza u (\ref{eq:projektivnarep})
        ne može eleminirati jednostavnom redefinicijom operatora $D(g) \to
        e^{i\phi(g)}D(g)$.}, a o fizikalnom sustavu hoće li se transformirati 
        obzirom na projektivne reprezentacije ili ne.

    \item  Jedan od načina da grupa ima i projektivne reprezentacije je da
           ne bude jednostavno povezana. Konkretna topologija grupnog prostora
           određuje mogućnosti za fazu $\phi(g_1,g_2)$.

    \item \SO{3} je dvostruko povezana i dopušta dvije faze: $\pm 1$. Fermionski
          sustavi se transformiraju projektivno. U literaturi se nekad kaže
          da se fermioni transformiraju u skladu s \emph{dvovrijednom} (engl.
          \emph{double-valued}) reprezentacijom grupe rotacija gdje se misli
          da se svaki pojedini element grupe rotacija preslikava
          u skup $\{U, -U\}$ dvaju operatora iz \SU{2}.
\end{itemize}

Grupa \SU{2} je jednostavno povezana i sve njene reprezentacije su "obične".
To nam onda omogućuje drugi pristup ovom formalnom problemu, a to je da
prihvatimo da je zapravo \SU{2} "prava" grupa rotacija i prava simetrija prirode
i onda ne moramo razmatrati projektivne reprezentacije.
To može na prvi pogled biti protivno intuiciji, ali postoje i klasične
situacije kod kojih rotacija za 360 stupnjeva nije identična
rotaciji za nula stupnjeva, a ona za 720 stupnjeva jest (poznati trikovi
s pojasom ili konobarevim pladnjem). Također, najjednostavniji odgovor na
pitanje kako komponirati dvije prostorne rotacije tj. kako odrediti
rezultantni kut i os rotacije, dobije se upravo korištenjem
kvaterniona (koji su ekvivalentni grupi \SU{2}) što isto sugerira svojvrsnu
fundamentalnost grupe \SU{2} i izvan kvantnomehaničkog konteksta.

\section{Orbitalni moment impulsa}

Među raznim operatorima u kvantnoj mehanici koji zadovoljavaju
komutacijske relacije \suAlg{2} algebre nailazimo i na važni
operator \emph{orbitalnog momenta impulsa} oblika 
\begin{equation}
\vec{L} = \vec{r} \times \vec{p} \;,
\label{eq:Lorb}
\end{equation}
gdje \suAlg{2} komutacijske relacije 
\begin{equation}
    [L_i, L_j] = \rmi \hbar \epsilon_{ijk} L_k \,,
\end{equation}
slijede izravno iz
temeljnih kvantnomehaničkih komutacijskih relacija
\begin{equation}
    [r_i, p_j] = i\hbar \delta_{ij} \Eins \,.
    \label{eq:rpkomutacija}
\end{equation}

Iz kvantne mehanike je poznato da orbitalni moment impulsa
poprima samo cjelobrojne vrijednosti 
\begin{equation}
     \vec{L}^2 \ket{l, m} = \hbar^2 l(l+1) \ket{l,m} \qquad l=0,1,2,\ldots\,,
\end{equation}
i odgovarajući momenti impulsa tj. ireducibilne
reprezentacije se tradicionalno označavaju s $l$, a ne $j$.
Iz povijesnih razloga vezanih uz atomsku fiziku, stanja koja
pripadaju $l=0,1,2,\ldots$ reprezentacijama
se često nazivaju $S, P, D,\ldots$ stanja, a $m$ se ponekad naziva
\emph{magnetski} kvantni broj (zbog svoje uloge u Zeemanovom efektu).

Zanimljivo je pitanje zašto za orbitalni moment impulsa otpadaju
polucjelobrojne reprezentacije $l \neq 1/2, 3/2, \ldots$?
    Sama algebra, kako smo vidjeli u prošlom odjeljku, implicira
postojanje i takvih reprezentacija, dakle stvar mora biti u konkretnom
obliku operatora (\ref{eq:Lorb}) odnosno svojstvima operatora
položaja i impulsa koji dodatno zadovoljavaju i relacije
(\ref{eq:rpkomutacija}).

Pozabavimo se prvo posebnim problemom dimenzionalnosti Hilbertovog prostora
na kojem operator $\vec{L}$ iz (\ref{eq:Lorb}) djeluje. Naime, taj
je prostor beskonačnodimenzionalan. Već smo to u sličnoj situaciji
sreli u primjeru \ref{pr:repC3} u kontekstu rotacije valnih funkcija
vodikovog atoma. No beskonačna dimenzionalnost Hilbertovog prostora
na koji djeluju operatori položaja i impulsa (pa time i $\vec{L}$)
vrijedi sasvim općenito u kvantnoj mehanici, u što se možemo uvjeriti
djelovanjem operacije traga na (\ref{eq:rpkomutacija}). Tu lijeva
strana zbog cikličnosti traga iščezava, a desna ne, pa imamo nekonzistenciju
za konačno-dimenzionalne operatore. U beskonačno-dimenzionalnim prostorima
trag operatora općenito nije definiran pa je samo tamo relacija
(\ref{eq:rpkomutacija}) matematički konzistentna.
Beskonačnodimenzionalni Hilbertovi prostori su u mnogome različiti
od konačnodimenzionalnih i velika većina rezultata u ovoj knjizi tamo ne vrijedi
onako kako su iskazani bez dodatnog posvećivanja velike pažnje
domeni pojedinih operatora i drugim suptilnostima.
Srećom, operator (\ref{eq:Lorb}) na cijelom Hilbertovom prostoru
je \emph{reducibilan}. Iz osnova kvantne mehanike znamo da se valne
funkcije faktoriziraju na radijalni i kutni dio $\psi(\vec{r})
= R(r) Y^{(l)}_{m}(\theta, \phi)$. Radijalni dio je onaj koji
je odgovoran za beskonačnu dimenzionalnost prostora,
no za naše potrebe nam on nije važan jer na njega rotacije djeluju
trivijalno. One djeluju netrivijalno na kutni dio $Y^{(l)}_{m}(\theta, \phi)$
i tu, analizom svojstvenih funkcija Laplaceove diferencijalne
jednadžbe (kutnog dijela Schr\"{o}dingerove jednadžbe), se u
standardnim knjigama iz kvantne mehanike pokazuje da su rotacije
ireduciblne na \emph{konačno-dimenzionalnim} prostorima razapetim
funkcijama $Y^{(l)}_{m}(\theta, \phi)$ za $m \in \{-l, -l+1, \ldots, l\}$.
Dakle potpuno smo u okviru razmatranja i rezultata prošlog odjeljka i ostaje
nam samo odgovoriti na pitanje što je s polucjelobrojnim reprezentacijama.

U literaturi se do rezultata da $l$ ne može biti polucjelobrojan
često dolazi nezadovoljavajuće. Autori se uglavnom oslanjaju na postavljanje rubnih
uvjeta na rješenja Laplaceove jednadžbe; konkretno, zahtijevaju
da valne funkcije moraju zadovoljavati $\psi(\phi + 2\pi) = \psi(\phi)$.
No sve fizikalne posljedice su, kako smo baš diskutirali
u prošlom odjeljku, neosjetljive na fazu valne funkcije
i nema nikakvog fizikalnog razloga zahtijevati takve rubne uvjete,
pogotovu kad smo svjesni da u prirodi postoje sustavi (fermioni)
koji takve zahtjeve \emph{ne} zadovoljavaju.
Izložit ćemo sada argument koji objašnjava zašto $l$ ne može biti
polucjelobrojan, a koji se ne oslanja na rubne uvjete nego samo
na temeljne komutacijske relacije \cite{Ballentine:1998}.

Radi jasnoće, umjesto standardnih operatora položaja $\vec{r}$
i impulsa $\vec{p}$, definirajmo njihove bezdimezionalne inačice
\begin{equation}
    \vec{Q} \equiv \sqrt{\frac{m \omega}{\hbar}} \,\vec{r} \,, \qquad
    \vec{P} \equiv \sqrt{\frac{1}{\hbar m \omega}} \, \vec{p} \,,
\end{equation}
gdje su $m$ i $\omega$ neke konstante s dimenzijama mase i frekvencije
koje nam neće biti od važnosti. Pomoću ovih operatora definiramo 
još četiri bezdimenzionalna operatora
\begin{align}
    q_{\pm}& \equiv \frac{1}{\sqrt{2}} (Q_x \pm P_y) \,, \\
    p_{\pm}& \equiv \frac{1}{\sqrt{2}} (P_x \mp Q_y) \,.
\end{align}
Ovi operatori zadovoljavaju komutacijske relacije
\begin{align}
    [Q_i, P_j]& =  [r_i, p_j] / \hbar = i \delta_{ij} \,, \\
    [q_{\pm}, p_{\pm}]& = i \,, \label{eq:qpkomutacija}\\
    [q_{\pm}, p_{\mp}]& = 0 \,,
\end{align}
i svi ostali komutatori isto iščezavaju. Uz ovako definirane
operatore imamo sada za bezdimenzionalni operator momenta impulsa
oko $z$-osi
\begin{equation}
    \frac{1}{\hbar} L_z = Q_x P_y - Q_y P_x = \frac{1}{2}(p_{+}^2 + q_{+}^2)
                   - \frac{1}{2}(p_{-}^2 + q_{-}^2) \;.
\end{equation}
Ovdje treba uočiti prvo da iz (\ref{eq:qpkomutacija}) vidimo da
$q_{\pm}$ i $p_{\pm}$ zadovoljavaju kvantne komutacijske relacije položaja
i impulsa, a drugo da je $(p^2 + q^2)/2$, ako je $p$ impuls, a $q$ položaj,
upravo bezdimenzionalna varijanta Hamiltonijana jednodimenzionalnog kvantnog
harmoničkog oscilatora. Dakle, operator $L_z$ je matematički ekvivalentan
\emph{razlici} Hamiltonijana dvaju \emph{nezavisnih} harmoničkih oscilatora
\begin{equation}
\frac{1}{\hbar} L_z = H^{(+)} - H^{(-)} \;.
\label{eq:Hpm}
\end{equation}
U svakoj knjizi iz kvantne mehanike je pokazano
da su svojstvene vrijednosti ovakvih Hamiltonijana, dakle energije harmoničkog
oscilatora, dane s 
\begin{equation}
    E_n = \hbar \omega \left(n + \frac{1}{2}\right)\,, 
    \quad n \in \{0, 1, 2, \ldots\} \,,
\end{equation}
pa će odgovarajuće svojstvene vrijednosti bezdimenzionalnih Hamiltonijana
iz (\ref{eq:Hpm}) biti upravo $n_{\pm} + 1/2$ uz nenegativne
cjelobrojne $n_{\pm}$. Tako slijedi da su moguće svojstvene vrijednosti
$L_z$ dane s
\begin{equation}
    \frac{1}{\hbar} L_z = \left(n_{+} + \frac{1}{2}\right) -
    \left(n_{-} + \frac{1}{2}\right) = n_{+} - n_{-} \,,
\end{equation}
tj. da su svakako cjelobrojne, što je i trebalo pokazati.

\section{Zbrajanje momenata impulsa i Clebsch-Gordanovi koeficijenti}
\label{sec:zbrajanjeJ}

Kod konačnih grupa smo razmatrali pitanje direktnog produkta
ireducibilnih reprezentacija i njegovog rastavljanja na tzv.
Clebsch-Gordanov direktni zbroj. U kontekstu ireducibilnih
reprezentacija grupe rotacija to je pitanje ekvivalentno
poznatom problemu "zbrajanja momenata impulsa" u kvantnoj
mehanici. Čitatelj koji poznaje to gradivo neće u ovom
odjeljku naći ništa novo, no edukativno je razumjeti te
rezultate izrečene jezikom teorije reprezentacija grupe rotacija.

Promotrimo fizikalni sustav izgrađen od dvaju podsustava poznatih
spinova $j_1$ i $j_2$. 
Ako su stanja pojedinog podsustava opisana vektorima u
prostorima $V_1$ i $V_2$
\begin{align}
\ket{j_1, m_1} &\in V_1\\
\ket{j_2, m_2} &\in V_2\,,
\end{align}
onda će združeni sustav biti opisan vektorima
\begin{equation}
    \ket{j_1, m_1; j_2, m_2} \equiv\ket{j_1, m_1}\ket{j_2, m_2} \,,
\end{equation}
koji su elementi produktnog vektorskog prostora $V_1\otimes V_2$.
Na $V_1$ i $V_2$ su definirani operatori momenta impulsa
$\vec{J}_1$ i $\vec{J}_2$ sa svojstvima
\begin{align}
\vec{J}_{1}^2 \ket{j_1, m_1; j_2, m_2} &= j_1(j_1+1)\hbar^2 
  \ket{j_1, m_1; j_2, m_2}\,, \\
\vec{J}_{2}^2 \ket{j_1, m_1; j_2, m_2} &= j_2(j_2+1)\hbar^2 
  \ket{j_1, m_1; j_2, m_2}\,, \\
J_{1z}\ket{j_1, m_1; j_2, m_2} &= m_1 \hbar \ket{j_1, m_1; j_2, m_2}\,,\\
J_{2z}\ket{j_1, m_1; j_2, m_2} &= m_2 \hbar \ket{j_1, m_1; j_2, m_2}\,.
\end{align}
Na vektorskom
prostoru $V_1$ grupa rotacija reprezentirana\footnote{Odsad ćemo i projektivne reprezentacije
    grupe \SO{3} zvati samo reprezentacije, ili ćemo \SU{2} smatrati grupom rotacija,
štogod je čitatelju draže, vidi odjeljak \ref{sec:projektivnerep}.}
je operatorima $D^{(j_1)}=
\exp(-i\vec{J}_1\cdot\hat{\vec{n}}\theta/\hbar)$ (i slično za $V_2$),
a na $V_1\otimes V_2$ operatorima
$D^{(j_1)}\otimes D^{(j_2)}$.
Međutim, $D^{(j_1)}\otimes D^{(j_2)}$ je općenito reducibilna reprezentacija
odnosno združeni sustav nema nužno definirani spin. 
$\ket{j_1, m_1; j_2, m_2}$ nije nužno svojstveno
stanje operatora ukupnog spina združenog sustava $\vec{J}^2 = (\vec{J}_1 + \vec{J}_2)^2$.

Da bismo ustanovili moguće spinove združenog sustava trebamo
provesti Clebsch-Gordanov rastav $D^{(j_1)}\otimes D^{(j_2)}$ na direktni
zbroj ireducibilnih reprezentacija
\begin{displaymath}
  D^{(j_1)}\otimes D^{(j_2)} =
  \sum_J \oplus\: a_{J}D^{(J)} \,,
\end{displaymath}
gdje su koeficijenti kao i inače dani s
\begin{displaymath}
a_J = \left( \chi^{(J)}, \chi^{(j_1)} \chi^{(j_2)} \right) \,.
\end{displaymath}
Za općeniti račun skalarnih umnožaka karaktera trebali bismo
znati invarijantno integrirati u grupnom prostoru grupe rotacija (cf. Jones,
Appendix C ili Hamermesh 9-2), ali ovdje ćemo potrebni rezulatat
dobiti i bez toga, uz par matematičkih trikova.
Kako su karakteri konstante klasa konjugacije, za reprezentante 
klasa biramo rotacije oko $z$-osi za koje su odgovarajući operatori
rotacije dijagonalni
\begin{equation}
\begin{split}
 \chi^{(j)}(\phi) &= \Tr D^{(j)}(\phi) = \Tr e^{(-\rmi/\hbar)J_3 \phi} \\
 &= \Tr \textrm{diagonal}\left( e^{-\rmi j\phi }, e^{-\rmi (j-1)\phi}, \ldots,
e^{-\rmi(-j)\phi}\right)\\
&= e^{-\rmi j\phi } + e^{-\rmi (j-1)\phi} + \cdots + e^{-\rmi(-j)\phi}\\
&= \text{geometrijski red s omjerom članova $e^{\rmi j\phi}$} \\
&= e^{-\rmi j\phi} \frac{1- (e^{\rmi j\phi})^{2j+1}}{1-e^{\rmi\phi}}
= \frac{e^{-\rmi(j+1/2)\phi} - e^{\rmi(j+1/2)\phi}}{e^{-\rmi\phi/2}-e^{\rmi\phi/2}}\\
&= \frac{\sin(j+1/2)\phi}{\sin\phi/2}
\end{split}
\end{equation}
Odredimo sada koeficijente $a_J$ Clebsch-Gordanovog razvoja. Pretpostavimo
prvo da je $j_1 \ge j_2$. Tada vrijedi
\begin{equation}
\begin{split}
\chi^{(j_1)}(\phi)\chi^{(j_2)}(\phi)&=
 \frac{e^{+\rmi(j_1+1/2)\phi} - e^{-\rmi(j_1+1/2)\phi}}{2\rmi\sin\phi/2}
\sum_{m=-j_2}^{j_2}e^{\rmi m \phi}\\
&= \frac{1}{2 \rmi \sin\phi/2} \sum_{m=-j_2}^{j_2}
\left[ e^{\rmi(j_1+m+1/2)\phi} - e^{-\rmi(j_1-m+1/2)\phi}\right]\\
&= (\text{zamjena $m\to -m$ u drugom članu}) \\
&=\sum_{m=-j_2}^{j_2} \frac{\sin(j_1+m+1/2)\phi}{\sin\phi/2}\\
&=\sum_{m=-j_2}^{j_2} \chi^{(j_1 + m)}(\phi) \qquad (J\equiv j_1 +m) \\
&=\sum_{J=j_1-j_2}^{J=j_1+j_2} \chi^{(J)}(\phi) \;.
\end{split}
\end{equation}
Ovdje $J$ mora biti pozitivan da bi bio labela neke ireducibilne
reprezentacije. Zato smo se ograničili na $j_1\ge j_2$.
Za slučaj $j_2 \ge j_1$ imali bi sve isto kao gore, samo uz zamjenu $j_1 \leftrightarrow
j_2$. Slijedi da općenito možemo pisati:
\begin{equation}
 \chi^{(j_1)}(\phi)\chi^{(j_2)}(\phi)=
\sum_{J=|j_1-j_2|}^{J=j_1+j_2} \chi^{(J)}(\phi)
\end{equation}
Dakle,
\begin{equation}
 a_J = \left(\chi^{(J)}, \chi^{(j_1)}\chi^{(j_2)}\right) =
 \sum_{J'=|j_1-j_2|}^{J'=j_1+j_2} 
\underbrace{\left(\chi^{(J)}, \chi^{(J')}\right)}_{\delta_{JJ'}}
\end{equation}
odnosno,
\begin{equation}
D^{(j_1)} \otimes D^{(j_2)} = \sum_{J=|j_1-j_2|}^{J=j_1+j_2} 
\oplus D^{(J)} \,. \label{eq:CGrot}
\end{equation}
Kao primjer, vrijedi $D^{(1/2)}\otimes D^{(1)} = D^{(1/2)} \oplus D^{(3/2)}$.
Treba primijetiti da se svaka ireducibilna reprezentacija pojavljuje najviše jednom
u  Clebsch-Gordanovom razvoju. To je veliko pojednostavljenje svojstveno grupi rotacija.
Za čitaoca bi bilo dobro da se uvjeri da su dimenzionalnosti reprezentacija
u (\ref{eq:CGrot}) konzistentne tj. da je
\begin{displaymath}
    \sum_{J=|j_1-j_2|}^{J=j_1+j_2} (2J+1) = (2j_1+1)(2j_2+1) \,.
\end{displaymath}

Osim gore navedene baze prostora $V_1 \otimes V_2$
$\ket{j_1, m_1; j_2, m_2}$ koju čine
svojstveni vektori operatora $\{\jsqj, \jsqd, J_{1z}, J_{2z}\}$ u kvantnomehaničkim
računima je važna i baza $\ket{j_1, j_2; J, M}\equiv\ket{J, M}$
tog istog prostora koju čine svojstveni vektori operatora
$\{\jsqj, \jsqd, \jsq, J_z\}$.
Ove dvije baze su naravno povezane
\begin{equation}
\ket{J, M} = \underbrace{\sum_{m_1,m_2}\ket{j_1, m_1; j_2, m_2}
\bra{j_1, m_1; j_2, m_2}}_{=1}
\ket{J, M} \,,
\end{equation}
i koeficijenti razvoja jedne baze po drugoj
\begin{equation}
 \bra{j_1, m_1; j_2, m_2} J, M\rangle \equiv C^{JM}_{j_1m_1j_2m_2}
 \label{eq:defCG}
\end{equation}
nazivaju se \emph{Clebsch-Gordanovi koeficijenti} i mogu se naći tabelirani
    u literaturi. U Dodatku \ref{sec:clebsch} mogu se naći neka svojstva
ovih koeficijenata i metoda njihovog određivanja.


\section{Tenzorski operatori i Wigner-Eckartov teorem}
\label{sec:tenzorskioperatori}

U prošla tri odjeljka smo naučili kako se kvantnomehanička \emph{stanja}
ponašaju pri rotacijama.
No, da bismo potpuno razumjeli posljedice rotacijske simetrije u
kvantnomehaničkim sustavima
potrebno je znati i kako se pri rotacijama ponašaju kvantnomehanički
\emph{operatori}. Mnogi važni operatori su obzirom na rotacije \emph{vektori},
što znači da se transformiraju analogno klasičnim vektorima tj. množenjem
sa standardnom matricom rotacije $R(\unitn, \theta)$ iz odjeljka
\ref{sec:tenzori}. Na primjer, transformacija operatora položaja je
\begin{equation}
    r_i \longrightarrow U(R) r_i U(R)^{-1} = R_{ij} r_j \;.
\end{equation}
Slično, tenzori višeg ranga, poput npr. $r_i p_j$ će se transformirati
množenjem s brojem matrica $R$ koji odgovara njihovom rangu, kao
u (\ref{eq:tenzor}). Problem je međutim da ti tenzori općenito nisu
\emph{ireducibilni} tj. mogu se rastaviti na zbroj tenzora
koji se ne miješaju pri rotacijama i neki od kojih su
nižeg ranga. Na primjer, općeniti tenzor drugog ranga $T_{ij}$ ima
devet komponenata, ali one se mogu organizirati u kombinacije
koje se međusobno ne miješaju pri rotacijama. 
Kao prvo, trag tenzora $\mathrm{Tr} T = T_{ii}$ je invarijantan
na rotacije jer imamo
\begin{align*}
 T_{ii} = \delta_{ij} T_{ij} \longrightarrow
\delta_{ij} R_{i i'} R_{j j'} T_{i' j'}& = (R_{j i'} R_{j j'}) T_{i' j'}
= (R^{\mathsf{T}} R)_{i' j'} T_{i' j'} \\
& = \delta_{i' j'} T_{i' j'} = T_{i' i'} \;,
\end{align*}
gdje smo u predzadnjem koraku upotrijebili svojstvo ortogonalnosti
matrica rotacije. Dakle, trag 
 $T_{xx} + T_{yy} + T_{zz}$ je skalar tj. tenzor ranga 0.
Nadalje, $T_{ij}$, baš kao i svaku matricu, možemo
rastaviti na antisimetrični i simetrični dio
\begin{equation}
  T_{ij} = \fhalf (T_{ij} - T_{ji}) + \fhalf (T_{ij} + T_{ji}) \;.
\label{eq:SplA}
\end{equation}
Lako je uočiti da rotacije ne miješaju ove dvije komponente:
rotacija (anti)si\-met\-ri\-čne matrice daje (anti)simetričnu matricu,
kako će čitatelj pokazati u zadatku \ref{zad:antisim}.
Tako tri nezavisne antisimetrične kombinacije, $(T_{xy}-T_{yx})/2$,
$(T_{yz}-T_{zy})/2$ i $(T_{zx} - T_{xz})/2$, čine tenzor
prvog ranga tj. vektor.
Ovaj tročlani skup nije dalje reducibilan jer rotacije međusobno
miješaju sva tri elementa. Njegovu vektorsku prirodu možemo
lako uočiti na primjeru $T_{ij} = 2 r_i p_j$ gdje ove tri
kombinacije upravo daju tri komponente momenta impulsa 
$L_k = \epsilon_{ijk} r_i p_k$.
Simetrični dio ima šest komponenata, $T_{xx}$, $T_{yy}$, $T_{zz}$,
$(T_{xy}+T_{yx})/2$, $(T_{yz}+T_{zy})/2$ i $(T_{zx} + T_{xz})/2$, 
ali treba uočiti da je zbroj prvih triju jednak tragu za kojeg
smo upravo vidjeli da je sam za sebe ireducibilan 
pa ga treba eliminirati da bi se dobilo pet
nezavisnih komponenata koje čine tenzor drugog ranga.
Dakle, konačni rastav tenzora drugog ranga na
ireducibilne tenzore ranga 0, 1 i 2 je
\begin{equation}
T_{ij} = \frac{1}{3}\delta_{ij}T_{kk} +
  \fhalf (T_{ij} - T_{ji}) + 
 \bigg[\fhalf (T_{ij} + T_{ji}) - \frac{1}{3}\delta_{ij}T_{kk} \bigg] \;,
 \label{eq:Tijrastav}
\end{equation}
gdje faktor $1/3$ osigurava da zadnji član bude traga nula.

Kao posljedica reducibilnosti djelovanje ovakvih tenzorskih operatora na 
stanja $\ket{j,m}$ ireducibilnih reprezentacija rezultirat će stanjima
koja nemaju jednostavna transformacijska svojstva jer
su superpozicija komponenata iz različitih ireducibilnih reprezentacija. To će
otežavati i računanje s takvim stanjima i njihovu fizikalnu interpretaciju.
Stoga uvodimo drugačije tenzorske operatore s 
transformacijskim svojstvima srodnima onima stanja $\ket{j,m}$.
Iz (\ref{eq:Drot}) i (\ref{eq:WignerD}) slijedi da se  $\ket{j,m}$
stanja transformiraju množenjem (transponiranim) Wignerovim D-matricama
\begin{equation}
 D(R) \ket{j,m} = \sum_{m'} \ket{j,m'}\bra{j,m'}D(R)\ket{j,m}
               = \sum_{m'} D^{(j)}_{m'm}\ket{j,m'} \;.
\end{equation}
\begin{definicija}[Ireducibilni sferični tenzorski operator]
Ireducibilni sferični tenzorski operator $T^{(k)}$ ranga $k$, sa
$2k+1$ komponenata $T^{(k)}_q$, $q=-k,-k+1,\ldots,k$ je operator
koji zadovoljava
\begin{equation}
 D(R) T^{(k)}_q D(R)^{-1} = \sum_{q'=-k}^{k} D^{(k)}_{q'q}(R)
      T^{(k)}_{q'} \;.
      \label{eq:defTk}
\end{equation}
\end{definicija}
Pogledajmo prvo infinitezimalnu verziju ove definicione relacije.
Do prvog reda razvoja u malom kutu rotacije $\theta \ll 1$, 
(\ref{eq:defTk}) postaje
\begin{equation}
    \left(1-\frac{i}{\hbar}\vec{J}\cdot\unitn \theta\right)
       T^{(k)}_q
    \left(1+\frac{i}{\hbar}\vec{J}\cdot\unitn \theta\right)
    = \sum_{q'=-k}^{k} \bra{k, q'} 
    \left(1-\frac{i}{\hbar}\vec{J}\cdot\unitn \theta\right)
    \ket{k, q}
      T^{(k)}_{q'} \;,
\end{equation}
odnosno
\begin{equation}
    [\vec{J}\cdot\unitn, T^{(k)}_q] 
    = \sum_{q'=-k}^{k} \bra{k, q'} \vec{J}\cdot\unitn \ket{k, q}
      T^{(k)}_{q'} \;.
\end{equation}
Sada izborom $\unitn = \vec{\hat{z}}$ i
$\unitn = \vec{\hat{x}} \pm \rmi \vec{\hat{y}}$
te korištenjem (\ref{eq:jz}), (\ref{eq:jplus}) i (\ref{eq:jminus}) 
dobijamo komutacijske relacije
\begin{align}
[J_z, T^{(k)}_q ] &= \hbar q T^{(k)}_q \,, \\
[J_\pm, T^{(k)}_q] &= \hbar \sqrt{(k\mp q)(k\pm q +1)} T^{(k)}_{q\pm 1} \,,
\end{align}
koje se mogu smatrati ekvivalentnom
definicijom ireducibilnih sferičnih tenzora. 
Obične (općenito reducibilne) tenzore s početka odjeljka nekad
nazivamo \emph{kartezijevi} jer su im komponente definirane pomoću kartezijeve
baze $\{\vec{\hat{x}}, \vec{\hat{y}}, \vec{\hat{z}}\}$.
Obični kartezijevi tenzori ranga 1, dakle vektori, zadovoljavaju komutacijske
relacije
\begin{equation}
 [J_i, V_j] = \rmi \hbar \epsilon_{ijk} V_k \;.
 \label{eq:jvkomutacija}
\end{equation}
\begin{primjer}[Tenzori ranga 0]
Tenzori ranga nula tj. skalari imaju samo jednu komponentu
i nema razlike između sferične varijante $T^{(0)}_0$ i kartezijeve varijante $T$.
Ovakvi operatori komutiraju sa sve tri komponente operatora momenta impulsa
\begin{equation}
    [J_i, T] = 0 \;,
\end{equation}
i tako su invarijantni na rotacije, baš kao što su operatori
koji komutiraju s Hamiltonijanom invarijantni na vremenske translacije.
\end{primjer}
\begin{primjer}[Tenzori ranga 1]
    Standardni vektorski operatori, poput $\vec{r}$, $\vec{p}$ i
    \vec{J}, su kartezijevi operatori ranga 1. Za razliku od
    kartezijevih operatora ranga 2, oni \emph{jesu} ireducibilni.
    Rotacije naravno miješaju sve tri kartezijeve komponente
    $T_x$, $T_y$, $T_z$ vektorskog operatora $\vec{T}$. 
    \emph{Sferični} tenzor ranga $k=1$ ima $2k+1=3$ komponente i
    one su povezane s kartezijevim komponentama relacijama
    \begin{equation}
    T^{(1)}_0 = T_z \;, \quad T^{(1)}_{\pm 1} = \mp \frac{1}{\sqrt{2}}
    (T_x \pm i T_y) \;,
    \label{eq:kartvektor}
    \end{equation}
    koje čitatelj treba provjeriti.
    Tako na primjer $J_z$ i $\,\mp (1/\sqrt{2})J_\pm$ formiraju sferični tenzor 
    ranga 1\footnote{Treba spomenuti da $D^{(j)}$ \emph{nije} 
    sferični tenzorski operator ranga $j$.}.
\end{primjer}
\begin{primjer}[Tenzori ranga 2]
    Kartezijev tenzor ranga 2 je reducibilan i u (\ref{eq:Tijrastav}) je ekspliciran
    rastav na sferične tenzore ranga 0, 1 i 2, s ukupno 1+3+5=9 komponenata,
    gdje pojedine komponente $T^{(k)}_q$ nisu izravno vidljive u (\ref{eq:Tijrastav}) 
    nego ih se može odrediti računom.
\end{primjer}


Glavni razlog upotrebe sferičnih tenzorskih operatora je da njihovo
djelovanje na $\ket{j,m}$ stanja rezultira stanjima s jednostavnim
transformacijskim svojstvima. Rotacija stanja
$T^{(k)}_q\ket{j,m}$ dana je s
\begin{align}
    U(R) \Big( T^{(k)}_q\ket{j,m} \Big)& =
        U(R)  T^{(k)}_q U(R)^{-1} U(R) \ket{j,m}  \\
   &= \sum_{q'=-k}^{k} \sum_{m=-j}^{j}
   D^{(k)}_{q'q}(R) D^{(j)}_{m'm}(R) T^{(k)}_{q'} \ket{j, m'} \;,
\end{align}
a to je identično transformaciji kakvu bi imalo stanje
\begin{equation}
    \ket{k, q} \otimes \ket{j, m} = \ket{k, q; j, m} \;.
\end{equation}
Dakle djelovanje sferičnog operatora ranga $k$ na sustav spina $j$ 
je matematički ekvivalentno zbrajanju dvaju sustava spinova $k$ i $j$.
Stoga se često kaže da je $k$ "spin" takvog operatora.
Amplituda vjerojatnosti da stanje $\ket{k, q; j, m}$ ima
neki konkretni spin $J$ i projekciju spina $M$ dana je s Clebsch-Gordanovim
koeficijentom $C^{J M}_{jmkq}$, vidi (\ref{eq:defCG}).
Po bliskoj analogiji s tim, za amplitudu vjerojatnosti
pronalaženja stanja $T^{(k)}_q\ket{j,m}$ u stanju $\ket{j',m'}$
stanje  imamo sljedeći teorem, koji je jedan od centralnih
rezultata primjene simetrija u kvantnoj mehanici.
\begin{teorem}[Wigner-Eckart]
Matrični elementi sferičnih tenzorskih operatora između svojstvenih stanja momenta
impulsa zadovoljavaju relaciju
\begin{equation}
 \bra{j'm'}T^{(k)}_q \ket{jm} = C^{j'm'}_{jmkq}
\frac{\bra{j'}|T^{(k)}|\ket{j}}{\sqrt{2j+1}} \;,
\end{equation}
gdje je $\bra{j'}|T^{(k)}|\ket{j}$ tzv. \emph{reducirani matrični element}
koji ne ovisi o $m, q$ i $m'$.
\end{teorem}
Posljedica ovog teorema je da izračunavanje jednog jedinog matričnog elementa
(npr. za slučaj $m'=q=m=0$) onda omogućuje trivijalno određivanje svih ostalih
$(2j'+1)\times(2k+1)\times(2j+1)$ matričnih elemenata
pomoću tablica Clebsch-Gordanovih koeficijenata.
Amplituda se faktorizira na reducirani matrični element koji je netrivijalni 
"dinamički" dio i Clebsch-Gordanove koeficijente koji sadrže obično
manje važnu informaciju o prostornim orijentacijama stanja i operatora.
Teorem omogućuje i određivanje tzv. \emph{izbornih pravila} za prijelaze
atomskih i nuklearnih kvantnih sustava pod vanjskim utjecajima.

\begin{primjer}[Skalarni operator]
\begin{displaymath}
\bra{j'm'}|T^{(0)}_0\ket{jm} = C^{j'm'}_{jm00}
\frac{\bra{j'}|T^{(0)}|\ket{j}}{\sqrt{2j+1}}
\end{displaymath}
pa svojstva Clebsch-Gordanovih koeficijenata, vidi dodatak \ref{sec:clebsch},
odmah daju da matrični element iščezava osim ako je $m=m'$ i
$|j-0|\leq j' \leq j+0$ odnosno $j=j'$. To je naprosto odraz
činjenice da skalarni operator nikako ne rotira stanje.
\end{primjer}


\begin{primjer}[Izborna pravila za dipolno zračenje]
Interakcija atoma s elektromagnetskim poljem opisanim vektorskim
potencijalom $\vec{A}$ dana je hamiltonijanom
\begin{equation}
    \frac{1}{2m}\left(\vec{p} - \frac{e}{c}\vec{A}\right)^2
    = \frac{e}{mc} \vec{p}\cdot\vec{A} + \cdots
    = \frac{e}{c} \dot{\vec{r}} \cdot \vec{A}
    = e \vec{r} \cdot \vec{E} + \cdots \,.
\end{equation}
Električno polje $\vec{E} = \vec{\epsilon}\, e^{i \vec{k}\cdot\vec{r}}$ se 
u dipolnoj aproksimaciji, gdje se uzima da je valna duljina zračenja puno
veća od atoma, aproksimira samo polarizacijskim vektorom 
$\vec{E}\approx\vec{\epsilon}(1+\cdots)$
pa je amplituda prijelaza između dvaju nivoa atoma, specificiranih
radijalnim kvantnim brojevima $n$ i $n'$, momentima impulsa
$l$ i $l'$ i magnetnim kvantnim brojevima $m$ i $m'$ proporcionalna
matričnom elementu operatora položaja
\begin{equation}
    \vec{\epsilon} \cdot \bra{n'l'm'}\vec{r}\ket{nlm}  \,.
\end{equation}
Operator položaja $\vec{r}$ je kartezijev vektor (tenzor
ranga 1), a istovremeno i sferični tenzor $r^{(1)}$ ranga 1, gdje
je korespondencija komponenata navedena u (\ref{eq:kartvektor}).
Wigner-Eckartov teorem kaže da su matrični elementi prijelaza
proporcionalni Clebsch-Gordanovim koeficijentima za zbrajanje
momenta impulsa $l$ s momentom impulsa 1 (rang operatora položaja)
\begin{displaymath}
 \bra{n'l'm'}x^{(1)}_q \ket{nlm} = C^{l'm'}_{lm1q}
\frac{\bra{n'l'}|x^{(1)}|\ket{nl}}{\sqrt{2l+1}} \;,
\end{displaymath}
To odmah povlači da su prijelazi mogući samo ako
je $|l-1|\leq l' \leq l+1$ odnosno da je moguća promjena
momenta impulsa atoma u dipolnom zračenju $\Delta l = \pm 1,0$\footnote{Usput,
    dodatna simetrija koju ima ovaj sustav, simetrija na prostornu
    inverziju $\vec{x} \to -\vec{x}$, dodatno zabranjuje $\Delta l =0$
prijelaze.}. Ovakvi uvjeti na prijelaze između kvantnih nivoa
nazivaju se \emph{izborna pravila}.
Nadalje, kako Clebsch-Gordanovi koeficijenti iščezavaju ako
nije zadovoljeno $m+q = m'$, ukoliko znamo da je električno
polje linearno polarizirano 
$\vec{\epsilon} = \hat{\vec{z}}$ imamo i dodatno izborno
pravilo da je $m'=m$ tj. $\Delta m = 0$. Za transverzalno
ili cirkularno polarizirano polje u $x-y$ ravnini imat ćemo
pak $m'= m\pm 1$ tj. $\Delta m = \pm 1$.
Treba uočiti da ovakva izborna pravila vrijede i za kompleksnije
atome za koje ne znamo izračunati radijalni dio valnih funkcija.
Ona su naprosto posljedica sferne simetrije odnosno zakona očuvanja
momenta impulsa: foton ima spin 1 i njegovom emisijom ili apsorpcijom
moment impulsa atoma se može promijeniti najviše za 1.
\end{primjer}



\section{Spektar vodikovog atoma i \SO{4} simetrija}
\label{sec:so4}

U udžbenicima kvantne mehanike se spektar vodikovog atoma obično
dobiva rješavanjem Schr\"{o}dingerove diferencijalne jednadžbe.
No povijesno prvi izračun spektra dao je Pauli 1926.
koristeći Heisenbergovu matričnu mehaniku uz
ingenioznu upotrebom načela simetrije, što ćemo izložiti u ovom odjeljku.


Vidjeli smo u odjeljku \ref{sec:degeneracija} da postojanje operatora
$\{U(g) \td g\in G \}$ koji reprezentiraju grupu $G$, te koji
komutiraju s hamiltonijanom $[U(g), H] = 0$, povlači degeneraciju
energijskih nivoa koji odgovaraju skupu stanja $\{U(g)\ket{\alpha} \td g\in
G \}$. Tamo je to bilo ilustrirano na primjeru konačnih grupa. Zanimljiv
primjer degeneracije kao posljedice kontinuirane sferne simetrije su
nivoi u vodikovom atomu. Hamiltonijan (u CGS sustavu jedinica)
\begin{equation}
    H = \frac{p^2}{2m} - \frac{e^2}{r} \;.
\label{hatom}
\end{equation}
je rotacijski simetričan i kao posljedica toga komutira s operatorom
rotacija
\begin{equation}
     [H, D(\vec{n},\phi)] = 0 \;.
\end{equation}
To povlači da svih $2l+1$ stanja date ireducibilne reprezentacije
grupe rotacija s kvantnim brojem $l$
\begin{equation}
   \big\{\ket{nlm} \td m\in\{-l, -l+1, ..., l \}\big\} = 
   \big\{ D(g)\ket{nlm} \td g \in \SO{3} \big\}
\end{equation}
ima istu energiju, kako smo već spominjali u odjeljku \ref{sec:reprezentacije}.
Sferna simetrija implicira da energija ne ovisi o kvantnom broju $m$
tj. o prostornoj orijentaciji sustava.
Međutim, poznato je da energije stanja vodikovog atoma ne
ovise ni o kvantnom broju $l$ i da $n^2$ stanja
\begin{equation}
\big\{\ket{nlm} \td l=\{0, 1, \ldots, n-1\}; m\in\{-l, -l+1, \ldots, l \}\big\} 
\end{equation}
imaju istu energiju\footnote{Ovdje radimo s pojednostavljenim modelom vodikovog atoma
opisanim hamiltonijanom (\protect\ref{hatom}). U realnom vodikovom atomu
postoje i dodatni članovi u hamiltonijanu, poput člana interakcije spina
i orbite, koji razbijaju ovu degeneraciju i čine da energije nivoa ovise
i o orbitalnom kvantnom broju $l$ --- tzv. \emph{fina struktura} vodikovog
spektra.}
\begin{equation}
    E_n = - \frac{e^4 m}{2 \hbar^2 n^2}  \;.
\end{equation}
Dakle, umjesto $2l+1$-struke degeneracije koju očekujemo kao posljedicu
rotacijske simetrije imamo veću, $n^2$-struku degeneraciju. Ovakva
situacija obično znači da sustav ima veću simetriju nego što smo
originalno očekivali. Koju to simetriju, pored rotacijske, ima
sustav opisan hamiltonijanom (\ref{hatom})?
Da bismo istražili to pitanje vratit ćemo se u područje klasične
fizike gdje se javlja slična situacija u problemu dva tijela čiji
je hamiltonijan
\begin{equation}
    H = \frac{p^2}{2m} - \frac{e_{M}^2}{r} \quad ; \quad 
 e_{M}^2 \equiv G M m \;.
\end{equation}
matematički ekvivalentan onom vodikovog atoma.
Kod problema dva tijela rotacijska simetrija i njoj odgovarajući zakon
očuvanja momenta impulsa ($\vec{L}$=const.) manifestiraju se kroz
činjenicu da putanja sustava (elipsa) ostaje cijelo vrijeme u istoj
ravnini.  Međutim, ovdje se javlja i zanimljiva dodatna simetrija ---
putanja je zatvorena elipsa i
smjer perihela elipse je također konstantan\footnote{Ovo
vrijedi u klasičnoj Newtonovoj teoriji gravitacije. Poznato je da
Einsteinova teorija gravitacije korigira ovaj rezultat i da smjer perihela
elipse nije konstantan već vrlo polako precesira.}.
Odgovarajući očuvani vektor je tzv. \emph{Laplace-Runge-Lenzov}
vektor
\begin{equation}
  \vec{M} = \vec{v}\times\vec{L} - \frac{e_{M}^2}{r}\vec{r} \;.
\label{lrl}
\end{equation}
Njegovo očuvanje ($\vec{M}$=const.) slijedi iz drugog Newtonovog
zakona uz malo elementarne vektorske algebre.
Prije svega primijetimo da je
\begin{displaymath}
 \frac{\dd \hat{\vec{r}}}{\dd t} = \frac{\dd}{\dd t} \frac{\vec{r}}{r} =
 \frac{r \vec{v} - \vec{r}
\frac{\vec{r}\cdot\vec{v}}{r}}{r^2} \;,
\end{displaymath}
gdje smo iskoristili $\vec{r}\cdot\vec{v} = \dd (\vec{r}\cdot\vec{r})/2\dd t
= r dr/dt $.
Sad deriviramo (\ref{lrl}) po vremenu pa uz korištenje 
\[ \dd \vec{v}/ \dd t = \vec{F}/m = - (e_{M}^2 \vec{r})/(mr^3) \]
dobivamo
\begin{displaymath}
\begin{split}
\frac{\dd \vec{M}}{\dd t} &=
\frac{\dd \vec{v}}{\dd t} \times \vec{L} + \vec{v}\times
\underbrace{\frac{\dd \vec{L}}{\dd t}}_{=0} - e_{M}^2 \frac{\dd \hat{\vec{r}}}
{\dd t} \\
&= -\frac{e_{M}^2}{mr^3} \underbrace{\vec{r}\times\vec{L}}_{(\vec{r}\cdot 
\vec{p})\vec{r} - r^2 \vec{p}} -
    \frac{e_{M}^2}{mr^3} \Big[ r^2\vec{p} - \vec{r}(\vec{r}\cdot\vec{p})\Big] \\
&= 0 \,.
\end{split}
\end{displaymath}
Dobro je uočiti kako je za ovaj izvod ključno da je $\vec{F}\propto 1/r^2$.
Također, uočite da su $\vec{M}$ i $\vec{L}$ okomiti jer je očito da je
$\vec{M}\cdot\vec{L}=0$.
Sad nakon što smo identificirali ovaj
dodatni očuvani vektor (\ref{lrl}) možemo lako riješiti problem
dvaju tijela. Vrijedi
\begin{equation}
\begin{split}
 \vec{r}\cdot\vec{M} &= \underbrace{\vec{r}\cdot(\vec{v}\times\vec{L})}_{
\vec{L}\cdot(\vec{r}\cdot\vec{v})} - \frac{e_{M}^2}{r} r^2
= \frac{L^2}{m} - e_{M}^2 r \\
&= r M \cos \theta
\end{split}
\end{equation}
gdje je $\theta$ kut kojeg zatvara $\vec{r}$ prema konstantnom vektoru
$\vec{M}$. Ovu jednadžbu možemo prepisati u obliku
\begin{equation}
\frac{1}{r} = \frac{e_{M}^2 m}{L^2} \left(1+\frac{M}{e_{M}^2} \cos\theta\right) \,,
\end{equation}
što prepoznajemo kao polarnu jednadžbu elipse ekscentriciteta $e=M/e_{M}^2$
(vidi npr. \cite{Goldstein:1980}, jednadžba (3-51)). Dakle riješili smo problem i
dobili trajektoriju sustava bez ikakvog rješavanja diferencijalne
jednadžbe gibanja!

Možemo li sad ova saznanja primjeniti u kvantnoj mehanici na naš
početni problem dodatne degeneracije nivoa vodikovog atoma?
Postoji li kvantnomehanički operator koji bi bio analogon 
Laplace-Runge-Lenzovog vektora (\ref{lrl})? Naivni pokušaj
s operatorom
\begin{equation}
  \vec{M} = \frac{\vec{p}}{m}\times\vec{L} - e^2\frac{\vec{r}}{r} \;,
\end{equation}
ne prolazi jer ovaj operator, zbog 
\begin{displaymath}
    (\vec{p}\times\vec{L})^\dagger = - \vec{L}\times\vec{p} \;,
\end{displaymath}
nije hermitski. Stoga je potrebno definirati kvantni
Laplace-Runge-Lenzov vektor ovako:
\begin{equation}
  \vec{M} = \frac{1}{2m}\big(\vec{p}\times\vec{L}- \vec{L}\times\vec{p}\big) 
- e^2\frac{\vec{r}}{r} \;.
\end{equation}
Uz podosta računa (vidi npr. \cite{Greiner:1989}, zadaci 14.4--14.8) 
pokazuje se da vrijede sljedeći identiteti
\begin{align}
 [\vec{M}, H] &= 0 \;, \tag{A} \\
 \vec{L}\cdot\vec{M} &= 0 \;, \tag{B} \\
\vec{M}^2 &= \frac{2 H}{m}(\vec{L}^2 + \hbar^2) + e^4  \;, \tag{C} \\
 [L_i, M_j] &= i\hbar \epsilon_{ijk} M_k \;, \tag{D} \\
[M_i, M_j] &= i \hbar \epsilon_{ijk} \left(- \frac{2H}{m}\right) L_k \tag{E} \;.
\end{align}
(A) kaže da je $\vec{M}$ očuvan i u kvantnomehaničkom slučaju, a (D) da
je $M$ kartezijev vektor u smislu odjeljka \ref{sec:tenzorskioperatori}.
(Klasični analogon jednadžbe (C) je $\vec{M}^2 = 2E\vec{L}^2/m + e_{M}^4$.)
Definirajmo sada operator 
\begin{equation}
\vec{M}' \equiv \left(- \frac{2H}{m}\right)^{-\fhalf} \vec{M} \;.
\end{equation}
Uočite da je izraz u zagradi pozitivno definitan jer je cijeli spektar operatora $H$
negativan buduči da radimo s vezanim stanjima vodikovog atoma.
Ovakav reskalirani operator $\vec{M}'$ zadovoljava sada komutacijske
relacije
\begin{equation}
[M'_i, M'_j] = i \hbar \epsilon_{ijk}  L_k \tag{E'} \;,
\end{equation}
i pomoću njega možemo definirati dva nova operatora
\begin{align}
\vec{I} &= \fhalf \big(\vec{L} + \vec{M}'\big) \,, \\
\vec{K} &= \fhalf \big(\vec{L} - \vec{M}'\big) \;.
\end{align}
(Odnosno $\vec{L} = \vec{I} + \vec{K}$ i
$\vec{M}' =  \vec{I} - \vec{K}$.)
Uporabom relacija (D) i (E') te standardnih komutacijskih
relacija za moment impulsa $\vec{L}$ lako se vidi da ova 
dva operatora zadovoljavaju komutacijske relacije
\begin{align}
[I_i, I_j] &= i \hbar \epsilon_{ijk}  I_k \,, \\
[K_i, K_j] &= i \hbar \epsilon_{ijk}  K_k \,,\\
[I_i, K_j] &= 0 \,,
\end{align}
što znači da i $I_i$ i $K_i$ generiraju dvije nezavisne
algebre grupe \SO{3} tj. da je ukupna grupa simetrija vodikovog
atoma \SO{3}$\times$\SO{3}\footnote{Vrijedi grupni identitet
\SO{3}$\times$\SO{3} = \SO{4} pa otud naslov ovog odjeljka. 
\SO{4} je grupa generirana s 6
generatora $L_{\mu\nu}=x_\mu p_\nu - x_\nu p_\mu$; $\mu, \nu = 1,2,3,4$;
$[x_\mu, p_\nu] = i\hbar \delta_{\mu\nu}$. Definiramo li
$L_i = \fhalf \epsilon_{ijk} L_{jk}$ i $M'_i = L_{4i}$, $i,j,k = 1,2,3$,
dobijemo gornje komutacijske relacije.}.

Sad možemo upotrijebiti naše poznavanje svojstava reprezentacija
grupe \SO{3}. Činjenica da $I_i$ i $K_i$ zadovoljavaju
iste komutacijske relacije kao i moment impulsa povlači da
su svojstvene vrijednosti od $\vec{I}^2$ dane kao $i(i+1)\hbar^2$,
a od $\vec{K}^2$ kao $k(k+1)\hbar^2$, uz $i, k \in \{ 0, 1/2, 1, 3/2, \ldots\}$.
Dodatni uvjet na ove svojstvene vrijednosti je relacija (B) koja
daje:
\begin{equation}
0 = \vec{L}\cdot\vec{M}' = (\vec{I})^2 - (\vec{K})^2
  = i(i +1)\hbar^2 - k(k +1)\hbar^2 \;,
\label{j1j2}
\end{equation}
odnosno $i = k\equiv j$. (Alternativno rješenje jednadžbe (\ref{j1j2})
$i = - (k +1 )$ nije moguće jer daje negativne vrijednosti
za $i$ ili $k$.)
Prepišimo sada jednadžbu (C) pomoću novih operatora. Množenjem (C)
s $(-2H/m)^{-1}$ dobije se
\begin{equation}
   \vec{M'}^2 = -(\vec{L}^2 + \hbar^2) - \frac{me^4}{2H} \,,
\end{equation}
ili
\begin{align}
 -\frac{me^4}{2}H^{-1} &= \vec{M'}^{2} + \vec{L}^2 + \hbar^2 \\
&=\big[\vec{I} - \vec{K}\big]^2 + 
\big[\vec{I} + \vec{K}\big]^2  + \hbar^2\\
&=2\big[\vec{I}^2 + \vec{K}^2 \big]^2 + \hbar^2  \;.
\end{align}
To znači da za svojstvena stanja energije i momenta impulsa vrijedi
\begin{equation}
-\frac{me^4}{2}E^{-1} = 2\big[i(i +1)\hbar^2 + k(k+1)\hbar^2 
\big]^2 + \hbar^2   = \hbar^2 \big[ 4j(j+1) +1 \big] = \hbar^2 (2j+1)^2
\end{equation}
što daje spektar
\begin{equation}
   E = - \frac{m e^4}{2\hbar^2 (2j+1)^2} \;.
\end{equation}
Uvedemo li kvantni broj $n\equiv 2j+1$, onda iz $j=0,1/2, 1,\ldots$
slijedi $n=1,2,3,\ldots$ i ovo prepoznajemo kao ispravni izraz
za spektar vodikovog atoma.
Kako je "pravi" angularni moment $\vec{L}=\vec{I} +
\vec{K}$ njegove svojstvene vrijednosti dane su pravilima
za zbrajanje momenata impulsa tj.
\begin{equation}
    |j - j| \leq l \leq j + j \,,
\end{equation}
tj. $l \leq 2j \leq n-1$, što je poznati rezultat. Također, $l$
automatski ispada cjelobrojan. Dakle odredili smo kvantnomehanički spektar
vodikovog atoma bez rješavanja Schr\"{o}dingerove jednadžbe!

Da se još jednom uvjerimo da je degeneracija
svakog nivoa $n^2$-struka  možemo prebrojati stanja u bazi koja
dijagonalizira operatore $H$, $\vec{L}^2$ i $L_z$, što je
uobičanjena $\ket{nlm}$ baza. Tu za dati $n$ imamo sve skupa
\begin{equation}
 \sum_{l=0}^{n-1} (2l+1) =\frac{n}{2}\big(1+2(n-1)+1)\big) = n^2
\end{equation}
degeneriranih stanja.
Alternativna baza je ona koja dijagoalizira operatore
$\vec{I}^2=\vec{K}^2$, $I_z$ i $K_z$ čije vektore možemo
označiti kao $\ket{j,m_i, m_k}$ i koja daje degeneraciju
\begin{equation}
    \sum_{m_i=-j}^{j}\sum_{m_k=-j}^{j} 1 = (2j+1)^2 = n^2 \;,
\end{equation}
u skladu s prvim računom.


\subsection*{Zadaci}

\begin{enumerate}[label=\arabic{chapter}.\arabic*.]

    \item \label{zad:matj}
Odredite $\bra{j, m'}J_{i} \ket{j, m} \equiv (J_i)_{m' m}$ za 
$j=1/2$ i $j=1$

\item
Pokažite da vrijedi: 
\begin{equation}
\bra{j, m'} D^{(j)}(\phi, \theta, \psi) \ket{j, m} = e^{-im'\phi-im\psi}
\bra{j, m'} e^{-i J_y \theta /\hbar} \ket{j, m} \;,
\end{equation}
gdje su $\phi$, $\theta$ i $\psi$ tri Eulerova kuta,
te odredite eksplicitno ove matrične elemente za $j=1/2$.

\item
Izrazite stanje $\ket{\vec{J}\cdot\unitn, +}$ definirano svojstvom
\begin{displaymath}
    \vec{J}\cdot\unitn \ket{\vec{J}\cdot\unitn, +} =
  \frac{\hbar}{2} \ket{\vec{J}\cdot\unitn, +}
\end{displaymath}
preko stanja baze $\ket{j=1/2, m=\pm 1/2}$.

\item
Koja je vjerojatnost da mjerenje projekcije spina na $z$-os za stanje iz
prošlog zadatka da rezultat $\hbar/2$?

\item 
Neka je $\ket{\unitn}$ svojstveno stanje operatora usmjerenja u 3D
prostoru. Uočite da je
\begin{displaymath}
     \bra{n}lm\rangle = Y^{m}_l (\unitn) = Y^{m}_l (\theta, \phi) \;.
\end{displaymath}
Promatrajući operator $D(\phi,\theta,0)$ koji rotira $\ket{\hat{\vec{z}}}$
u $\ket{\unitn}$ pokažite da vrijedi
\begin{displaymath}
    D^{l}_{m0}(\phi,\theta,0)=\sqrt{\frac{4\pi}{2l+1}}
 Y^{m^*}_l (\theta, \phi)\;.
\end{displaymath}

\item
Promotrite stanje $\ket{\text{rot}_y(\beta)}$, dobiveno rotacijom stanja
$\ket{l=2, m=0}$ za kut $\beta$ oko $y$-osi. Pronađite vjerojatnosti da
mjerenje projekcije momenta impulsa na $z$-os da vrijednosti
$m'=0,\pm 1, \pm 2$.

\item Čestica spina $1/2$ je u $D$ stanju orbitalnog momenta impulsa
($l=2$). Koja su moguća stanja ukupnog momenta impulsa? Koje su energije
tih stanja ako je hamiltonijan 
\[ H = A + B \vec{L}\cdot\vec{S} + C \vec{L}^2 \]
gdje su $A$, $B$ i $C$ poznate konstante?

\item
Izrazite komponente sferičnog vektora $r^{(1)} = (r^{(1)}_{-1}, r^{(1)}_{0},
r^{(1)}_{1})$ preko kartezijevih komponenata $r_x, r_y, r_z$ tj. izvedite
relaciju (\ref{eq:kartvektor}).

\item Neka je poznato da za sferični vektorski operator $\sigma$ vrijedi
\[  \bra{\fhalf, \fhalf} \sigma_0 \ket{\fhalf, \fhalf} = 1 \;. \]
Izračunajte sve ostale matrične elemente ovog operatora između stanja
s $j = 1/2$.

\item Pronađite izborna pravila za zračenje u kristalu s $C_{3v}$
simetrijom za zračenje polarizirano (a) duž $z$-osi i (b) duž
$x$ ili $y$ osi. \secret{(Hammermesh)}

\item 
Tri matrice, $M_x$, $M_y$, i $M_z$, svaka $256\times 256$, zadovoljavaju
komutacijske relacije $[M_i, M_j] = i \sum_{k} \epsilon_{ijk} M_k$.
Svojstvene vrijednosti od $M_z$ su:
\begin{center}
\begin{tabular}[h]{l|ccccccccc}
\hline
svojstvena vrijednost & 2 & -2 & 3/2 & -3/2 & 1 & -1 & 1/2 & -1/2 & 0 \\ \hline
koliko puta se pojavljuje & 1 & 1 & 8 & 8 & 28 & 28 & 56 & 56 & 70 \\ \hline
\end{tabular}
\end{center}
Navedite svojstvene vrijednosti matrice $M^2 \equiv M_{x}^2 + M_{y}^2 + M_{z}^2$
i broj njihovih pojavljivanja.

\item Pokažite da se sve komponente tenzorskog operatora 
$T^{(J)}_M$ mogu dobiti iz $T^{(J)}_J$ uzastopnom primjenom operatora $J_-$:
\[ T^{(J)}_M = C(J,M) [J_-, [J_-, \dots [J_-, T^{(J)}_J]. . .]] \;. \]
Koliki je $C(J,M)$?

\item \label{zad:antisim} Pokažite da se simetrični i antisimetrični
dijelovi tenzora drugog ranga ne miješaju pri rotacijama.

\item Komutator kvantnomehaničkih generatora rotacija oko različitih osi
    je proporconalan $\hbar$. Na primjer, $[L_x, L_y] = i\hbar L_z$. To
    bi sugeriralo da u klasičnom limitu, gdje $\hbar\to 0$, rotacije
    oko različitih osi komutiraju. No znamo da to nije točno.
    U čemu je stvar?

\end{enumerate}

