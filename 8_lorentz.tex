%Fancy RCS footer:
     \fancyfoot[C]{\texttt{8\_lorentz.tex}}
     \fancyfoot[RO,LE]{\mbox{$$Revision: 1.4 $$}}
     \fancyfoot[LO,RE]{\mbox{$$Date: 2012-05-07$$}}
% Correcting the title chapter page
\fancypagestyle{plain}{%
    \fancyhf{}
    \fancyhead[RO,LE]{\bfseries \thepage}
    \fancyhead[CO]{\rightmark}
    \fancyhead[CE]{\leftmark}
     \fancyfoot[C]{\texttt{8\_lorentz.tex}}
     \fancyfoot[RO,LE]{\mbox{$$Revision: 1.4 $$}}
     \fancyfoot[LO,RE]{\mbox{$$Date: 2012-05-07$$}}
    \renewcommand{\headrulewidth}{0.4pt}
    \renewcommand{\footrulewidth}{0.4pt}}

\chapter{Lorentzova i Poincar\'{e}ova simetrija}

\section{Lorentzova grupa}
\label{sec:lorentz}

\textbf{Princip relativnosti:} Postoji skup ekvivalentnih koordinatnih
sustava u međusobnom jednolikom pravocrtnom gibanju (tzv. \emph{skup
inercijalnih sustava}) takvih da fizikalni zakoni i pojave u svima njima
izgledaju isto. (Promatrač ne može eksperimentalno detektirati da 
se giba, ako se giba jednoliko. Mirovanje nije apsolutno.)

\secret{- toga su na ovaj ili onaj način bili svjesni Kopernik, Galilei, Newton}

- Princip relativnosti je princip simetrije $\to$ 3-parametarski skup
transformacija simetrije. Transformacije preslikavaju inercijalne
sustave jedan u drugi.

- Da bismo znali o kojoj se grupi radi (i da li se uopće radi o grupi),
moramo znati kako se kombiniraju transformacije među inercijalnim sustavima.

Kako je poznato, iz principa relativnosti slijedi da
transformacije iz sustava $S=\{t,x,y,x\}$ u sustav $S'=\{t',x',y',x'\}$, 
tzv. \emph{Lorentzovi potisci}, imaju oblik\footnote{U
standardnoj literaturi se pri izvodu Lorentzovih potisaka osim principa
relativnosti obično koristi i postulat o konstantnosti brzine svjetlosti
u svim sustavima.  No, ovaj drugi postulat je zapravo suvišan, vidi
Mermin, Am. J. Phys. \textbf{52}(2) (1984) 119 ili J. D. Jackson,
\emph{Classical Electrodynamics}, 3rd ed.(!), exercises 11.1 i 11.2}
\begin{align}
t' &= \gamma \big(t-\frac{\beta}{c}z\big) \\
z' &= \gamma \big(z-\beta ct \big) \\
x' &= x \\
y' &= y  \;,
\end{align}
gdje je uzeto da se je brzina relativnog gibanja dvaju sustava duž
$z$-osi, $\vec{v}=v\hat{\vec{z}}$, te su uvedene standardne pokrate
\begin{displaymath}
 \beta \equiv \frac{v}{c} \;; \qquad \qquad \gamma\equiv\frac{1}{
\sqrt{1-\beta^2}} \;.
\end{displaymath}
Ove transformacije djeluju na 4-dim vektorskom prostoru koji
se zove \emph{prostor Minkowskog} i kojeg sačinjavaju tzv.
\emph{Lorentzovi vektori}: 
\begin{align}
x^\mu \qquad  \mu&=0,1,2,3  \qquad x^0=ct, x^1=x, x^2=y, x^3=z \\
x^\mu &= (ct, \vec{x}) \\
x_\mu &= (ct,-\vec{x}) = g_{\mu\nu} x^\nu
\qquad x_0=ct, x_1=-x, x_2=-y, x_3=-z \\
g &= 
\begin{pmatrix}
1 & 0 & 0 & 0 \\
0 &-1 & 0 & 0 \\
0 & 0 &-1 & 0 \\
0 & 0 & 0 &-1
\end{pmatrix}
\end{align}
Izraženo preko ovih Lorentzovih 4-vektora, Lorentzovi potisci
poprimaju oblik 
\begin{align}
x^{0'} &= \gamma (x^0 - \beta x^3 ) \\
x^{1'} &= x^1 \\
x^{2'} &= x^2  \\
x^{3'} &= \gamma (x^3 -\beta x^0 ) \;,
\end{align}
ili u kompaktnom matričnom obliku
\begin{equation}
 x^{\mu'} = \Lambda^{\mu}_{\;\nu} x^\nu \;, \qquad
\Lambda^{\mu}_{\;\nu} =
\begin{pmatrix}
\gamma & 0 & 0 & -\beta\gamma \\
0 & 1 & 0 & 0 \\
0 & 0 & 1 & 0 \\
-\beta\gamma & 0 & 0 & \gamma
\end{pmatrix} \;.
\end{equation}
4-vektori poput $x^\mu$ imaju dakle jednostavna transformacijska
svojstva pri Lorentzovim potiscima i očekujemo da će se jednadžbe
relativističke fizike ``izgrađivati'' od takvih vektora te odgovarajućih
skalara i tenzora, baš kao što se u nerelativističkoj fizici jednadžbe
izgrađuju od 3-vektora i drugih tenzora s dobrim transformacijskim 
svojstvima pri prostornim rotacijama.

Matrice $\Lambda$ ovise o parametrima Lorentzovog
potiska kojeg je prirodno parametrizirati vektorom brzine $\vec{v}$ pojedinog
inercijalnog sustava u odnosu na neki referentni sustav.
 Vidimo da Lorentzovi potisci $\Lambda(\vec{v})$ čine 3-parametarski
skup. Da li je on grupa? Kako ćemo pokazati u slijedećem
odjeljku odgovor je ne. Kompozicija dva Lorentzova potiska nije
nužno također Lorentzov potisak već može biti i kompozicija
Lorentzovog potiska s prostornom rotacijom:
\begin{equation}
 \Lambda(\vec{v}_2) \circ \Lambda(\vec{v}_1)
  = R(\vec{v}_2,\vec{v}_1) \circ \Lambda(\vec{v}_2,\vec{v}_1) \;.
\end{equation}

Lako se vidi da i $\Lambda$ i $R$ ostavljaju invarijantnim
skalarni produkt $(x,y)\equiv x^{0}y^0 - \vec{x}\cdot\vec{y}$. Skup \emph{svih}
transformacija koje čuvaju ovaj skalarni produkt u prostoru Minkowskog
čini grupu O(1,3) koja se obično naziva \emph{opća Lorentzova grupa}.
Grupa prostornih rotacija SO(3)=$\{R(\unitn, \theta)\}$ 
je podgrupa, a skup Lorentzovih
potisaka $\{\Lambda(\vec{v})\}$ je podskup ove grupe.

Specifično razlikovanje "gornjih" i "donjih" indeksa 4-vektora u
specijalnoj teoriji relativnosti služi samo jednostavnom zapisu
skalarnog produkta u prostoru Minkowskog kod kojeg su članovi
prostornih koordinata suprotnog predznaka od člana vremenske koordinate:
\begin{equation}
 (x,y) = x^\mu y_\mu = x^0 y^0 - \vec{x}\cdot\vec{y} \;.
\end{equation}
U \emph{općoj} teoriji relativnosti to razlikovanje dviju vrsta
koordinata prelazi u razlikovanje dviju vrsta vektora 
(\emph{kovarijantni} i \emph{kontravarijantni} vektori), odnosno,
još preciznije jezikom diferencijalne geometrije razlikujemo 
\emph{vektore} i \emph{1-forme}, ali ovdje nam te finese ne igraju
nikakvu ulogu.

\subsubsection*{Struktura grupe O(1,3)}

Za transformirani 4-vektor $x' = \Lambda x$ vrijedi
\begin{align}
 {x'}^2 &= g_{\mu\nu} {x'}^\mu {x'}^\nu \\
        &= ({x^0}', {x^1}', {x^2}', {x^3}')
\begin{pmatrix}
1 & 0 & 0 & 0 \\
0 &-1 & 0 & 0 \\
0 & 0 &-1 & 0 \\
0 & 0 & 0 &-1
\end{pmatrix}
\begin{pmatrix}
{x^0}' \\ {x^1}' \\ {x^2}' \\ {x^3}'
\end{pmatrix} \\
  &= {x'}^{\top} g x' = x^\top \Lambda^\top g \Lambda x \\
  &= x^2 = x^\top g x
\end{align}
pa usporedbom dobivamo alternativnu definiciju opće Lorentzove grupe
kao grupe svih matrica $\Lambda$ sa svojstvom
\begin{equation}
   \Lambda^\top g \Lambda = g
\label{deflambda}
\end{equation}
Uzimanjem determinante ove matrične jednadžbe, uz svojstva
da je $\det A^\top = \det A$ i $\det g = -1$ dobijemo 
\begin{equation}
   (\det \Lambda)^2 = 1 \imp  \det\Lambda = \pm 1 \;.
\label{detL}
\end{equation}
Nadalje, raspišimo tu matričnu jednadžbu po komponentama
\begin{equation}
 \underbrace{(\Lambda^\top)_{\mu}^{\;\nu}}_{\Lambda^{\nu}_{\:\mu}}
g_{\nu\rho}\Lambda^{\rho}_{\:\sigma} = g_{\mu\sigma}
\end{equation}
i pogledajmo komponentu $\mu=\sigma=0$:
\begin{align}
1 &= g_{\nu\rho} \Lambda^{\nu}_{\:0} \Lambda^{\rho}_{\:0}  \\
  &= (\Lambda^{0}_{\:0})^2 - \sum_{i=1}^{3}(\Lambda^{i}_{\:0})^2 \;.
\end{align}
Slijedi da je $(\Lambda^{0}_{\:0})^2 = 1 + \sum_{i=1}^{3}(\Lambda^{i}_{\:0})^2$
odnosno da je $(\Lambda^{0}_{\:0})^2 \geq 1$ što daje dvije mogućnosti:
\begin{equation}
    \Lambda^{0}_{\:0}\geq 1  \qquad\text{ili}\qquad \Lambda^{0}_{\:0}\leq -1 \;.
\end{equation}
Zajedno s dvije iz (\ref{detL}) imamo dakle četiri mogućnosti koje vode
na četiri odvojene komponente povezanosti od O(1,3): \\

\begin{center}
\begin{tabular}{cccc}
\hline
$\det\Lambda$ & $\Lambda^{0}_{\:0}$ & oznaka & napomena  \\ \hline
    1         &  $\geq 1$   & $\mathcal{L}^{\uparrow}_{+}$ & sadrži 1 \\
 -1           &  $\geq 1$   & $\mathcal{L}^{\uparrow}_{-}$ & 
$= P \mathcal{L}^{\uparrow}_{+} $ \\
 1           &  $\leq -1$   & $\mathcal{L}^{\downarrow}_{+}$ & 
$= - \mathcal{L}^{\uparrow}_{+} = PT \mathcal{L}^{\uparrow}_{+} $\\
 -1           &  $\leq -1$   & $\mathcal{L}^{\downarrow}_{-}$ & 
$= T \mathcal{L}^{\uparrow}_{+} $  \\ \hline
\end{tabular}
\end{center}
gdje je
\begin{align}
 P=g:  (t\to t, 
\vec{x}\to -\vec{x}) \qquad &\quad \text{(paritet)} \label{eq:paritet} \\
 T=-g:  (t\to -t, 
\vec{x}\to \vec{x}) \qquad  &\quad \text{(vremenska inverzija)} \;.
\end{align}

- $\mathcal{L}^{\uparrow}_{+} \cup \mathcal{L}^{\downarrow}_{+}$
 --- (prava) Lorentzova grupa --- SO(1,3)

- $\mathcal{L}^{\uparrow}_{+} \cup\mathcal{L}^{\uparrow}_{-}$
 --- ortokrona Lorentzova grupa

- $\mathcal{L}^{\uparrow}_{+}$ --- prava ortokrona Lorentzova grupa

\secret{Topološki, $\mathcal{L}^{\uparrow}_{+} = \mathbb{R}^3 \times SO(3)$
i riječ je o nekompaktnoj grupi tako da su joj (izuzev trivijalne)
konačnodimenzionalne reprezentacije neunitarne, a unitarne
reprezentacije beskonačnodimenzionalne.}


\section{Generatori i reprezentacije Lorentzove grupe}

\label{id:5}
Kao i kod rotacija, sustave u prirodi treba klasificirati prema
njihovim transformacijskim svojstvima pri Lorentzovim transformacijama
tj. prema pripadnosti reprezentacijama Lorentzove
grupe. Kao i kod rotacija, poželjno je usredotočiti se na \emph{algebru}
grupe s generatorima $L$:
\begin{equation}
 \Lambda \in SO(1,3) \qquad \Lambda= e^L
\end{equation}
Iz definicionog svojstva (\ref{deflambda}) dobijemo $L^\top g = -g L$,
što uz činjenicu da je $g^\top = g$ daje $(gL)^\top = -gL$ odnosno
vidimo da je $gL$ antisimetrična matrica. To znači da ako $L$
parametriziramo na slijedeći način:
\begin{align}
 gL &= 
\begin{pmatrix}
1 & 0 & 0 & 0 \\
0 &-1 & 0 & 0 \\
0 & 0 &-1 & 0 \\
0 & 0 & 0 &-1
\end{pmatrix}
\begin{pmatrix}
L_{00} & L_{01} & L_{02} & L_{03}\\
L_{10} &\hdotsfor{3} \\
\hdotsfor{4} \\
\hdotsfor{3} & L_{33}
\end{pmatrix} \\[1ex]
&= 
\begin{pmatrix}
L_{00} & L_{01} & L_{02} & L_{03}\\
-L_{10} & -L_{11} & -L_{12} &\hdotsfor{1} \\
-L_{20} & -L_{21} & \hdotsfor{2} \\
\hdotsfor{3} & -L_{33}
\end{pmatrix}
\end{align}
svojstvo antisimetrije $gL$ traži
\begin{align}
L_{0i} &= L_{i0} \\
L_{ij} &= - L_{ji} \;.
\end{align}
tj.
\begin{equation}
 L = \begin{pmatrix}
0 & L_{01} & L_{02} & L_{03}\\
L_{01} & 0 & L_{12} & L_{13}\\
L_{02} & -L_{12} & 0& L_{23} \\
L_{03} & -L_{13} &-L_{23} & 0
\end{pmatrix} \;,
\end{equation}
gdje tri parametra $L_{01}$, $L_{02}$ i $L_{03}$ opisuju Lorentzove
potiske, a tri parametra $L_{12}$, $L_{13}$ i $L_{23}$ prostorne rotacije.
Umjesto ovih šest parametara pogodno je koristiti parametre
$\theta_i$ i $\zeta_i$,  definirane na slijedeći način:
\begin{equation}
 L = -i (\theta_i J_i + \zeta_i K_i)  \qquad i=1,2,3  \;,
\label{eq:LJK}
\end{equation}
gdje su $J_i$ već dobro poznati generatori rotacija, samo prošireni
na četverodimenzionalni prostor Minkowskog:
\begin{equation}
 J_1 =
\begin{pmatrix}
0 & 0 & 0 & 0 \\
0 & 0 & 0 & 0 \\
0 & 0 & 0 & -i \\
0 & 0 & i & 0
\end{pmatrix} \quad
 J_2 =
\begin{pmatrix}
0 & 0 & 0 & 0 \\
0 & 0 & 0 & i \\
0 & 0 & 0 & 0 \\
0 & -i & 0 & 0
\end{pmatrix} \quad
J_3 =
\begin{pmatrix}
0 & 0 & 0 & 0 \\
0 & 0 & -i & 0 \\
0 & i & 0 & 0 \\
0 & 0 & 0 & 0
\label{eq:defJi}
\end{pmatrix}
\end{equation}
dok su $K_i$ generatori Lorentzovih potisaka
\begin{equation}
K_1 =
\begin{pmatrix}
0 & -i & 0 & 0 \\
-i & 0 & 0 & 0 \\
0 & 0 & 0 & 0 \\
0 & 0 & 0 & 0
\end{pmatrix} \quad
K_2=
\begin{pmatrix}
0 & 0 & -i & 0 \\
0 & 0 & 0 & 0 \\
-i & 0 & 0 & 0 \\
0 & 0 & 0 & 0
\end{pmatrix} \quad
K_3 =
\begin{pmatrix}
0 & 0 & 0 & -i \\
0 & 0 & 0 & 0 \\
0 & 0 & 0 & 0 \\
-i & 0 & 0 & 0
\end{pmatrix} \;.
\label{eq:defKi}
\end{equation}

Treba primijetiti kako operatori $K_i$ nisu hermitski tako da odgovarajuće
transformacije $\exp(-i\zeta_i K_i)$ neće biti unitarne. To je posljedica
nekompaktnosti Lorentzove grupe --- parametri potiska poprimaju
vrijednosti iz nekompaktnog intervala $[0,c)$.

Pogledajmo sada algebru grupe SO(1,3). Lako se vidi da su
komutacijske relacije:
\begin{align}
[J_i, J_j] &= i\epsilon_{ijk} J_k  \label{eq:LK1}\\
[K_i, K_j] &= -i\epsilon_{ijk} J_k \\
[J_i, K_j] &= i\epsilon_{ijk} K_k \;. \label{eq:LK3}
\end{align}
Prva relacija je dobro poznata algebra grupe prostornih rotacija SO(3).
Druga relacija govori da podskup Lorentzovih potisaka nije zatvoren
i ne čini grupu, kako smo najavili u prošlom odjeljku. Treća
relacija govori da tri generatora potisaka $K_i$ čine vektor obzirom
na rotacije.

Ove komutacijske relacije su vrlo slične onima iz odjeljka \ref{so4} gdje smo 
rastavili grupu SO(4) na direktan produkt SU(2)$\otimes$SU(2) identificirajući
kombinacije generatora koji zatvaraju dvije neovisne podgrupe.
Slično ćemo postupiti i ovdje te definirati
\begin{equation}
  \vec{J}^{(\pm)} \equiv \fhalf \big(\vec{J}\pm i \vec{K} \big) \;,
\end{equation}
odnosno $\vec{J}=\vec{J}^{(+)}+\vec{J}^{(-)}$, $\vec{K} = -i
(\vec{J}^{(+)}-\vec{J}^{(-)})$. Primijetite dodatni ``$i$'' obzirom
na situaciju u odjeljku \ref{so4}.
Uz ovakve definicije imamo dvije odvojene algebre
\begin{align}
[J_{i}^{(+)}, J_{j}^{(+)}] &= i \epsilon_{ijk} J_{k}^{(+)} \\
[J_{i}^{(-)}, J_{j}^{(-)}] &= i \epsilon_{ijk} J_{k}^{(-)} \\
[J_{i}^{(+)}, J_{j}^{(-)}] &= 0  \;.
\end{align}
Ovo ipak ne znači  da O(1,3) ima istu algebru kao i SU(2)$\otimes$SU(2),
jer gore nismo radili \emph{realne} linearne kombinacije generatora.
Svejedno, za klasifikaciju ireducibilnih reprezentacija Lorentzove
grupe možemo kao i u odjeljku \ref{so4} koristiti parove
\begin{equation}
(j^{(+)}, j^{(-)})  \qquad j^{(+)}, j^{(-)} = 0, \fhalf, 1, \dotsc .
\end{equation}
Tako imamo trivijalnu (0,0) reprezentaciju i objekte koji se
transformiraju prema njoj zovemo Lorentzovi skalari. Slijedeće
dvije su tzv. \emph{Weylove} reprezentacije ($\fhalf$, 0) i
(0, $\fhalf$) prema kojima bi se transformirali bezmaseni
fermioni ako takvi postoje .
Obični 4-vektori poput $x^\mu$ pripadaju ireducibilnoj
reprezenaciji $(\fhalf, \fhalf)$.

Za masivne fermione pogodno je proširiti SO(1,3) Lorentzovu
grupu s operacijom pariteta i onda promatrati IRREPse obzirom
na ovu veću grupu\footnote{Masivno stanje impulsa $\vec{p}$
moguće je ``prestići'' Lorentzovim potiskom dovoljno velikog
parametra brzine $\vec{v}$, što rezultira stanjem koje izgleda
kao stanje impulsa $-\vec{p}$; ekvivalentno djelovanju pariteta
na originalno stanje.}. Operator pariteta (\ref{eq:paritet}) 
matrično izgleda kao
\begin{equation}
 P = P^{-1} =
\begin{pmatrix}
1 & 0 & 0 & 0 \\
0 & -1 & 0 & 0 \\
0 & 0 & -1 & 0 \\
0 & 0 & 0 & -1
\end{pmatrix} \;,
\end{equation}
i eksplicitnim djelovanjem na (\ref{eq:defJi}) i (\ref{eq:defKi}) 
je vidljivo da se pri paritetu generatori rotacije
transformiraju kao pseudovektori (vidi odjeljak \ref{sec:pseudovektori}),
\begin{equation}
   P^{-1} J_i P = J_i \;,
\end{equation}
a generatori Lorentzovih potisaka kao pravi polarni vektori,
\begin{equation}
   P^{-1} K_i P = - K_i \;.
\end{equation}
Slijedi da je 
\begin{equation}
 P^{-1} J^{(\pm)}_i P = J^{(\mp)} \;.
\end{equation}
Promotrimo sada neko konkretno stanje $\ket{(\fhalf,0)}$, koje je 
dublet obzirom na transformacije iz ($\fhalf$, 0), 
a singlet obzirom na (0, $\fhalf$):
\begin{align}
\vec{J}^{(+)^2} \ket{(\fhalf,0)}& = \fhalf(\fhalf+1) \ket{(\fhalf,0)} \\
\vec{J}^{(-)^2} \ket{(\fhalf,0)}& = 0 \;.
\end{align}
Kojoj reprezentaciji pripada paritetom transformirano stanje
$P \ket{(\fhalf,0)}$? To ustanovimo djelujući na to stanje
s $\vec{J}^{(\pm)^2}$:
\begin{align}
\vec{J}^{(-)^2} P \ket{(\fhalf,0)}& = 
P P^{-1} J^{(-)}_i P P^{-1} J^{(-)}_i P \ket{(\fhalf,0)}
= P \vec{J}^{(+)^2} \ket{(\fhalf,0)}  \nonumber \\
  & = \fhalf(\fhalf+1) P \ket{(\fhalf,0)} \\
\vec{J}^{(+)^2} P \ket{(\fhalf,0)}& = 0 \;.
\end{align}
Zaključujemo da stanje $P \ket{(\fhalf,0)}$ pripada IRREP (0, $\fhalf$).
Dakle, proširivanjem SO(1,3) Lorentzove grupe s paritetom,
(0, $\fhalf$) i ($\fhalf$,0) više nisu svaka zasebno IRREPs, već
IRREP postaje reprezentacija $(\fhalf, 0) \oplus (0, \fhalf)$, poznata
kao Diracova reprezentacija.


Slično, tenzor elektromagnetskog polja $F^{\mu\nu}$ pripada
6-dimenzionalnoj reducibilnoj reprezentaciji (1,0)$\oplus$(0,1).


Kao što smo u poglavlju \ref{rotacije} i od samih operatora tražili
dobro definirana tenzorska svojstva obzirom na rotacije
(npr. tri generatora $J_i$ čine vektor obzirom na rotacije), tako
je i u kontekstu Lorentzove simetrije moguće generatore i
same elemente grupe definirati na način koji manifestno pokazuje
kovarijantnost obzirom na Lorentzove transformacije.
Dakle, želimo relacije poput (\ref{eq:LJK}) i (\ref{eq:LK1})--(\ref{eq:LK3}) 
zapisati u 4-komponentnoj notaciji, putem Lorentzovih tenzora.
To postižemo definiranjem antisimetričnih generatora $J^{\mu\nu}$ kao
\begin{align}
J^{mn}& = \epsilon^{mni} J^i \\
J^{i0}& = K^i \;.
\label{eq:defJmn}
\end{align}
(Podsjetimo se da grčki indeksi idu $\alpha,\beta = 0,1,2,3$, a 
latinski $i,m = 1,2,3$.)  
Sada je element Lorentzove grupe dan kao
\begin{equation}
 \Lambda = \exp\left(-\frac{i}{2} \omega_{\rho\sigma} J^{\rho\sigma}\right)
\label{eq:4DLorentz}
\end{equation}
a komutacijske relacije (\ref{eq:LK1})--(\ref{eq:LK3}) se ujedinjuju u
\begin{equation}
i [ J^{\mu\nu}, J^{\rho\sigma}] = g^{\mu\rho} J^{\nu\sigma}
- g^{\nu\rho} J^{\mu\sigma} + g^{\sigma\mu} J^{\rho\nu}
- g^{\sigma\nu} J^{\rho\mu} \;.
\label{eq:4DLorKom}
\end{equation}

Na kraju, matrični elementi generatora Lorentzove grupe u
4-dim definicionoj ($\fhalf$, $\fhalf$) reprezentaciji dani su kao
\begin{equation}
\tensor{(J^{\mu\nu})}{^\alpha_\beta} = i \big(
g^{\mu\alpha}\tensor{\delta}{^\nu_\beta} - \tensor{\delta}{^\mu_\beta}
g^{\nu\alpha} \big) \;.
\label{eq:4DJmn}
\end{equation}

S druge strane matrični elementi Lorentzove grupe u
Diracovoj reprezentaciji
$(\fhalf, 0) \oplus (0, \fhalf)$
(koja je isto 4-dim) dani su kao
\begin{equation}
 J^{\mu\nu} = \frac{i}{4} [\gamma^\mu, \gamma^\nu] \;,
\label{eq:DiracJ}
\end{equation}
Gdje su $\gamma^\mu$ tzv. Diracove gama matrice definirane
antikomutacijskim relacijama
\begin{equation}
 \gamma^\mu\gamma^\nu + \gamma^\nu \gamma^\mu
\equiv \{\gamma^\mu, \gamma^\nu\} = 2 g^{\mu\nu}
\label{eq:DiracGamma}
\end{equation}
Jedan mogući izbor za ove matrice je
\begin{equation}
\gamma^0 = \left(\begin{array}{rr} 0 & -1 \\
                        -1 & 0 \end{array}\right)
\;, \quad
\gamma^i = \left(\begin{array}{rr} 0 & \sigma^i \\
                        -\sigma^i & 0 \end{array}\right) \;,
\end{equation}
gdje su $\sigma^i$ Paulijeve matrice (\ref{eq:PaulijeveMatrice}).

\section{Poincar\'{e}ova grupa$^*$}


Poincar\'{e}ova (poznata i kao nehomogena Lorentzova) grupa je
direktni produkt Lorentzove grupe i grupe translacija u prostoru
i vremenu. Elementi te grupe djeluju na vektore u 4-dim prostoru
Minkowskog kao
\begin{equation}
  x^\mu \longrightarrow \tensor{\Lambda}{^\mu_\nu} x^\nu + a^\mu
\label{eq:defPoin}
\end{equation}
gdje su $a^\mu$ četiri parametra translacija u prostor-vremenu.
Riječ je očito o deset-parametarskoj grupi gdje povrh generatora
Lorentzove grupe ($J^i$ i $K^i$) imamo i impuls $P^i$ kao
generator translacija u prostoru i Hamiltonijan $H$ kao
generator translacija u vremenu.
Algebra ove grupe je algebra Lorentzove grupe SO(1,3) (\ref{eq:4DLorKom}) 
i dodatno
\begin{align}
[H, H]& = [H, P^i] = [P^i, P^j] = 0 \label{eq:Poi1} \\
[H, J_i]& = 0  \label{eq:Poi2}\\
[K^i, H]& = i P^i  \label{eq:PoiPuzzle} \\
[J^i, P^j]& = i \epsilon^{ijk} P^k \\
[K^i, P^j]& = i H \delta^{ij} \;,
\label{eq:PoinKom}
\end{align}
što je moguće ujedinjeno napisati u 4-dim notaciji kao
\begin{equation}
i [P^\mu, J^{\rho\sigma}] = g^{\mu\sigma}P^\rho - g^{\mu\rho}P^{\sigma} \;.
\label{eq:4DPoinKom}
\end{equation}
Npr.
\begin{displaymath}
[H, K^i] = [P^0, J^{i0}] = -i(g^{0 0}P^i - g^{0i}P^0) =  -i P^i \;.
\end{displaymath}
Relacije  (\ref{eq:Poi1}) i (\ref{eq:Poi2}) govore da su impuls i
moment impulsa očuvani pri translacijama. Zanimljivo je da relacija
(\ref{eq:PoiPuzzle}) govori kako generatori Lorentzovih potisaka
$K^i$ ne komutiraju s Hamiltonijanom što sugerira da ne odgovaraju
očuvanim veličinama. To je pomalo zbunjujuće jer Noetherin teorem
nam govori kako bi simetrija obzirom na 10-parametarsku Poincar\'{e}ovu grupu
trebala rezultirati s deset očuvanih veličina, a ovako imamo samo
7: energija, tri komponente impulsa i tri komponente momenta impulsa.
Razrješenje je u tome da Noetherini očuvani naboji
(pokušajte ih eksplicitno odrediti) koji
odgovaraju Lorentzovim potiscima jesu formalno očuvani (vremenska
derivacija iščezava), ali eksplicitno ovise o vremenskoj 
koordinati\footnote{Slično kao što npr. $J_z = x P_y - y P_z$ 
eksplicitno ovisi o prostornim $x$ i $y$ koordinatama.}
pa nisu od praktične koristi. 

\subsection*{Zadaci}

\begin{enumerate}[{\thechapter}.1]
\item Uvjerite se eksplicitno da (\ref{eq:4DLorKom}) sadrži npr. (\ref{eq:LK3}).

\item Uvjerite se eksplicitno da (\ref{eq:4DJmn}) daje npr. (\ref{eq:defKi}).

\item Uvjerite se da generatori $J^{\mu\nu}$ definirani putem
(\ref{eq:DiracJ}) i (\ref{eq:DiracGamma}) zadovoljavaju komutacijske
relacije Lorentzove grupe (\ref{eq:4DLorKom}).

\end{enumerate}
