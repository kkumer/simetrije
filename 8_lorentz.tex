\fancypagestyle{plain}{%
    \fancyhf{}
    \fancyhead[RO,LE]{\bfseries \thepage}
    \fancyhead[CO]{\rightmark}
    \fancyhead[CE]{\leftmark}
    \renewcommand{\headrulewidth}{0.4pt}}

\chapter{Lorentzova i Poincar\'{e}ova simetrija}

Iz dosadašnjih poglavlja trebalo bi biti jasno da je moć
neke simetrije to veća što je odgovarajuća grupa veća, što
će za kontinuirane (Liejeve) grupe obično značiti da je broj generatora
grupe veći. Stanja fizikalnih sustava i jednadžbe ili lagranžijani
koji tim sustavima upravljaju
će u tom slučaju biti više ograničeni zahtjevima kovarijantnosti
na transformacije simetrije i predviđanja koja su posljedica
primjene teorije grupa će biti netrivijalnija i općenitija.
Dodatnu moć simetrije dobivaju od netrivijalnih komutacijskih
relacija između generatora simetrija.
Tako već i razmjerno jednostavana simetrija na rotacije ima izuzetno
netrivijalne posljedice, kako smo vidjeli u poglavlju \ref{ch:rotacije}.
Prirodno je onda postaviti pitanje koja je maksimalna
simetrija koju teorije prirode moraju poštivati. Osim rotacija,
maksimalna grupa simetrija svakako treba sadržavati i translacije
u prostoru i vremenu, koje smo sreli u odjeljcima
\ref{sec:prostornetranslacije} i \ref{sec:vremensketranslacije}.
Već i kombiniranje ovih transformacija u istu grupu simetrija
nije trivijalno jer translacije i rotacije ne komutiraju. To je
vidljivo i iz običnih razmatranja geometrije u prostoru, a i iz 
komutacijskih relacija odgovarajućih
kvantnomehaničkih operatora impulsa i momenta impulsa,
vidi (\ref{eq:jmkomutacija}) 

\begin{equation}
    [J_{i}, p_{j}] = \rmi \hbar \epsilon_{ijk} p_{k} \;.
    \label{eq:jpkomutacijske}
\end{equation}

Pokazuje se da osim upravo navedenih transformacija, maksimalna
grupa simetrija koja upravlja najuniverzalnijim poznatim zakonima
prirode, onima kvantne teorije polja, od kontinuiranih prostornovremenskih
transformacija sadrži još samo Lorentzove transformacije među inercijalnim sustavima,
poznate kao Lorentzovi \emph{potisci} (engl. \emph{boost}).
Ta maksimalna grupa se obično naziva Poincar\'{e}ova\footnote{Priroda poštuje
    i općenitije trasformacije prostorvremena poznate kao difeomorfizmi. 
    Odgovarajuće simetrije poštuje klasična Einsteinova teorija gravitacije, ali još
    ih nismo uspjeli ugraditi u univerzalne kvantne teorije ostalih sila i
    tim simetrijama se ne bavimo u ovoj knjizi. Ne bavimo se ni \emph{konformnom}
    grupom simetrija koja Poincar\'{e}ovim transformacijama dodaje i
    dilatacije $x^{\mu}\to\lambda x^{\mu}$ što je simetrija koja prema
    trenutnim spoznajama nije egzaktna simetrija prirode,
    ali je svejedno od velikog teorijskog interesa.}.
Prije nego se pozabavimo Poincar\'{e}ovom grupom i njenim reprezentacijama,
pogledat ćemo prvo u sljedeća dva odjeljka grupno-teorijska svojstva samih
Lorentzovih potisaka koji, kako ćemo vidjeti, ne tvore grupu samostalno
nego tek u kombinaciji s rotacijama.


\section{Lorentzova grupa}
\label{sec:lorentz}

Einsteinova specijalna teorija relativnosti počiva na tzv.
principu relativnosti. Prema njemu, postoji skup ekvivalentnih koordinatnih
sustava (zvanih \emph{inercijalni sustavi})
u međusobnom jednolikom pravocrtnom gibanju, a u kojima fizikalni zakoni i pojave
izgledaju isto. Promatrač ne može eksperimentalno detektirati da 
se giba, ako se giba jednoliko. Mirovanje nije apsolutno,
čega su na ovaj ili onaj način bili svjesni već Kopernik, Galilei i Newton, ali je
tek Einstein uočio druge radikalne posljedice ovog načela, poput toga
da ni simultanost događaja nije apsolutna.
Princip relativnosti je princip simetrije na troparametarski skup transformacija
koje preslikavaju inercijalne sustave jedan u drugi.
Da bismo vidjeli o kojoj se grupi simetrija radi (i da li se uopće radi o grupi),
moramo ustanoviti kako se kombiniraju transformacije među inercijalnim sustavima.

Kako je poznato, iz principa relativnosti slijedi da
transformacije iz sustava $S=\{t,x,y,x\}$ u sustav $S'=\{t',x',y',x'\}$, 
tzv. \emph{Lorentzovi potisci}, imaju oblik\footnote{U
standardnoj literaturi se pri izvodu Lorentzovih potisaka osim principa
relativnosti obično koristi i postulat o konstantnosti brzine svjetlosti
u svim sustavima.  No, ovaj drugi postulat je zapravo suvišan, vidi
Mermin, Am. J. Phys. \textbf{52}(2) (1984) 119 ili J. D. Jackson,
\emph{Classical Electrodynamics}, 3rd ed.(!), zadaci 11.1 i 11.2.}
\begin{align}
t' &= \gamma \big(t-\frac{\beta}{c}z\big) \,, \\
x' &= x \,, \\
y' &= y \,,  \\
z' &= \gamma \big(z-\beta ct \big) \,,
\end{align}
gdje je $c$ brzina svjetlosti i
gdje je radi jednostavnosti uzeto da se je brzina $\vec{v}$ relativnog gibanja
dvaju sustava duž $z$-osi, $\vec{v}=v\hat{\vec{z}}$, te su uvedene standardne pokrate
\begin{displaymath}
 \beta \equiv \frac{v}{c} \,, \qquad \qquad \gamma\equiv\frac{1}{
\sqrt{1-\beta^2}} \,.
\end{displaymath}
Ove transformacije djeluju na četverodimenzionalnom vektorskom prostoru koji
se zove \emph{prostor Minkowskog} i kojeg sačinjavaju tzv.
\emph{Lorentzovi vektori}
$x^\mu$,  $\mu=0,1,2,3$, čije su komponente  $x^0=ct, x^1=x, x^2=y$ i $x^3=z$.
U prostoru Minkowskog je vektorski produkt dvaju vektora definiran kao
\begin{equation}
    (x, y) \equiv x \cdot y \equiv x^0 y^0 - x^1 y^1 - x^2 y^2 - x^3 y^3 \,,
    \label{eq:produktminkowskog}
\end{equation}
pa je  pogodno definirati i tzv. \emph{kovarijantne}
komponente istog vektora s donjim indeksima
$x_0=ct, x_1=-x, x_2=-y, x_3=-z$, tako da je skalarni produkt jednostavno
\begin{equation}
x \cdot y = x^{\mu} y_{\mu} \;.
\end{equation}
\emph{Metrički tenzor}
\begin{equation}
g = 
\begin{pmatrix}
1 & 0 & 0 & 0 \\
0 &-1 & 0 & 0 \\
0 & 0 &-1 & 0 \\
0 & 0 & 0 &-1
\end{pmatrix} \,,
    \label{eq:metrickitenzor}
\end{equation}
omogućuje pretvorbu donjih kovarijantnih u gornje (\emph{kontravarijantne})
indekse tako da je
\begin{align}
x^\mu &= (ct, \vec{x}) \\
x_\mu &= (ct,-\vec{x}) = g_{\mu\nu} x^\nu
\end{align}
te skalarni produkt zapisujemo i kao\footnote{Specifično 
razlikovanje "gornjih" i "donjih" indeksa vektora u
specijalnoj teoriji relativnosti služi samo jednostavnom zapisu
skalarnog produkta u prostoru Minkowskog.
U \emph{općoj} teoriji relativnosti to razlikovanje dviju vrsta
koordinata prelazi u razlikovanje dviju vrsta vektora 
(\emph{kovarijantni} i \emph{kontravarijantni} vektori), odnosno,
još preciznije, jezikom diferencijalne geometrije razlikujemo 
\emph{vektore} i \emph{1-forme}, ali ovdje nam te finese ne igraju
nikakvu ulogu.}
\begin{equation}
     x \cdot y = g_{\mu\nu} x^{\mu}x^{\nu} \;.
\end{equation}
Izraženo preko ovih Lorentzovih vektora (zvanih i \emph{četverovektori}),
Lorentzovi potisci poprimaju oblik 
\begin{align}
x'^{0} &= \gamma (x^0 - \beta x^3 ) \\
x'^{1} &= x^1 \,, \\
x'^{2} &= x^2 \,, \\
x'^{3} &= \gamma (x^3 -\beta x^0 ) \;,
\end{align}
ili u kompaktnom obliku
\begin{equation}
    x'^{\mu} = \tensor{\Lambda}{^{\mu}_{\nu}} x^\nu \;, \qquad
\tensor{\Lambda}{^{\mu}_{\nu}} =
\begin{pmatrix}
\gamma & 0 & 0 & -\beta\gamma \\
0 & 1 & 0 & 0 \\
0 & 0 & 1 & 0 \\
-\beta\gamma & 0 & 0 & \gamma
\end{pmatrix} \;.
\label{eq:deflambda}
\end{equation}
Očekujemo da će se jednadžbe
relativističke fizike izgrađivati od vektora koji se transformiraju
kao i $x^{\mu}$, te odgovarajućih
skalara i tenzora, koji se transformiraju množenjem s brojem $\Lambda$ matrica
koji odgovara njihovom rangu, baš kao što se u nerelativističkoj fizici jednadžbe
izgrađuju od tenzora s dobrim transformacijskim svojstvima pri prostornim rotacijama,
definiranim u (\ref{eq:tenzor}).

Matrice $\Lambda$ ovise o parametrima Lorentzovog
potiska kojeg je prirodno parametrizirati vektorom brzine $\vec{v}$ pojedinog
inercijalnog sustava u odnosu na neki referentni sustav.
 Vidimo da Lorentzovi potisci $\Lambda(\vec{v})$ čine 3-parametarski
skup. Da li je on grupa? Kako ćemo eksplicitno pokazati u slijedećem
odjeljku odgovor je \emph{ne}. Kompozicija dva Lorentzova potiska, ako
isti nisu kolinearni, nije
samo Lorentzov potisak već kompozicija
Lorentzovog potiska i prostorne rotacije
\begin{equation}
 \Lambda(\vec{v}_2) \circ \Lambda(\vec{v}_1)
  = R(\vec{v}_2,\vec{v}_1) \circ \Lambda(\vec{v}_2,\vec{v}_1) \;,
\end{equation}
čija je  poznata posljedica pojava tzv. \emph{Thomasove precesije}. 
Tek skup svih Lorentzovih potisaka
i prostornih rotacija čini grupu, za čiju je identifikaciju najlakše
osloniti se na definiciju grupe kao skupa transformacija koji
ostavljaju skalarni produkt invarijantan, kako smo radili
u odjeljku \ref{sec:primjeriLie}, gdje smo definirali
pseudo-ortogonalnu grupu \O{1, 3} upravo kao grupu transformacija
koja ostavlja invarijantnim kvadratnu formu koja
odgovara skalarnom produktu (\ref{eq:produktminkowskog}).
Stoga se \O{1,3} često naziva \emph{opća Lorentzova grupa}.
Grupa prostornih rotacija \SO{3}
je podgrupa ove grupe koja čuva produkt (\ref{eq:produtkminkowskog}) tako
da čuva njen prostorni dio, a ne dira vremenski dio. Skup Lorentzovih
potisaka $\{\Lambda(\vec{v})\}$ je podskup ove grupe, ali ne i podgrupa.

Za transformirani 4-vektor $x' = \Lambda x$ vrijedi
\begin{align}
 {x'}^2 &= g_{\mu\nu} {x'}^\mu {x'}^\nu \\
        &= ({x^0}', {x^1}', {x^2}', {x^3}')
\begin{pmatrix}
1 & 0 & 0 & 0 \\
0 &-1 & 0 & 0 \\
0 & 0 &-1 & 0 \\
0 & 0 & 0 &-1
\end{pmatrix}
\begin{pmatrix}
{x^0}' \\ {x^1}' \\ {x^2}' \\ {x^3}'
\end{pmatrix} \\
  &= {x'}^{\top} g x' = x^\top \Lambda^\top g \Lambda x \\
  &= x^2 = x^\top g x
\end{align}
pa usporedbom dobivamo definiciju opće Lorentzove grupe
\O{1, 3}
kao grupe svih matrica $\Lambda$ sa svojstvom
\begin{equation}
   \Lambda^\top g \Lambda = g
\label{deflambda}
\end{equation}
što je definicija koju smo upoznali već u (\ref{eq:MgM}).
Uzimanjem determinante ove matrične jednadžbe, uz svojstva
da je za svaku matricu $\det A^\top = \det A$ te da je za
metrički tenzor $\det g = -1$ dobijemo 
\begin{equation}
   (\det \Lambda)^2 = 1 \,,
\end{equation}
iz čega slijedi da je
\begin{equation}
   \det\Lambda = \pm 1 \;.
\label{detL}
\end{equation}
Ova situacija je analogna situaciji u grupi \O{3} čiji elementi
također imaju determinantu $\pm 1$ pa (vidi argumentaciju na
stranici \pageref{eq:povezanostO3}) elementi koji imaju
determinantu $+1$ i tako čine podgrupu \SO{1, 3} nisu topološki
u grupnoj mnogostrukosti povezani sa elementima koji imaju
determinantu $-1$. Za razliku od \O{3}, koja ima točno te
dvije komponente povezanosti (vidi (\ref{eq:so3Iso3})), 
vidjet ćemo da ih \O{1, 3} ima četiri.
Naime, raspišimo matričnu jednadžbu (\ref{eq:deflambda})  po komponentama
\begin{equation}
    \underbrace{\tensor{(\Lambda^\top)}{_{\mu}^{\nu}}}_{\tensor{\Lambda}{^{\nu}_{\mu}}}
    g_{\nu\rho}\tensor{\Lambda}{^{\rho}_{\sigma}} = g_{\mu\sigma}
\end{equation}
i pogledajmo komponentu $\mu=\sigma=0$. Kako je $g_{00}=1$ imamo
\begin{align}
    1 &= g_{\nu\rho} \tensor{\Lambda}{^{\nu}_{0}} \tensor{\Lambda}{^{\rho}_{0}}  \\
      &= (\tensor{\Lambda}{^{0}_{0}})^2 - \sum_{i=1}^{3}(\tensor{\Lambda}{^{i}_{0}})^2 \;.
\end{align}
Slijedi da je 
\begin{equation}
    (\tensor{\Lambda}{^{0}_{0}})^2 = 1 + \sum_{i=1}^{3}(\tensor{\Lambda}{^{i}_{0}})^2 \,,
\end{equation}
odnosno da je $(\tensor{\Lambda}{^{0}_{0}})^2 \geq 1$ što daje dvije mogućnosti:
\begin{equation}
    \tensor{\Lambda}{^{0}_{0}}\geq 1  \qquad\text{ili}\qquad \tensor{\Lambda}{^{0}_{0}}\leq -1 \;.
\end{equation}
Zajedno s dvije mogućnosti za $\det \Lambda$ imamo dakle četiri mogućnosti koje vode
na četiri odvojene komponente povezanosti od O(1,3)
\begin{center}
\renewcommand{\arraystretch}{1.3}
\begin{tabular}{cccc}
\hline
$\det\Lambda$ & $\tensor{\Lambda}{^{0}_{0}}$ & oznaka & napomena  \\ \hline
    1         &  $\geq 1$   & $\mathcal{L}^{\uparrow}_{+}$ & sadrži 1 \\
 -1           &  $\geq 1$   & $\mathcal{L}^{\uparrow}_{-}$ & 
$= P \mathcal{L}^{\uparrow}_{+} $ \\
 1           &  $\leq -1$   & $\mathcal{L}^{\downarrow}_{+}$ & 
$= - \mathcal{L}^{\uparrow}_{+} = PT \mathcal{L}^{\uparrow}_{+} $\\
 -1           &  $\leq -1$   & $\mathcal{L}^{\downarrow}_{-}$ & 
$= T \mathcal{L}^{\uparrow}_{+} $  \\ \hline
\end{tabular}
\renewcommand{\arraystretch}{1.0}
\end{center}
gdje smo izdvojili specijalne elemente grupe \O{1, 3} 
prostornu inverziju (\emph{paritet}) $P$
i vremensku inverziju $T$, definirane kao
\begin{align}
    P& =g :  (t\to t, \vec{x}\to -\vec{x}) \,, \label{eq:paritet} \\
    T& =-g :  (t\to -t, \vec{x}\to \vec{x})  \;.
\end{align}
Komponenta povezanosti jedinice
$\mathcal{L}^{\uparrow}_{+}$ je od najvećeg interesa i
nekad se naziva \emph{prava ortokrona Lorentzova grupa}.
Podgrupu $\mathcal{L}^{\uparrow}_{+} \cup \mathcal{L}^{\downarrow}_{+}$ = \SO{1, 3},
 zovemo i prava Lorentzova grupa ili samo Lorentzova grupa,
dok podgrupu $\mathcal{L}^{\uparrow}_{+} \cup\mathcal{L}^{\uparrow}_{-}$
zovemo ortokrona Lorentzova grupa.


\section{Generatori i reprezentacije Lorentzove grupe}

\label{id:5}
Kao i kod rotacija, sustave u prirodi treba klasificirati prema
njihovim transformacijskim svojstvima pri Lorentzovim transformacijama
tj. prema pripadnosti reprezentacijama Lorentzove
grupe. Kao i kod rotacija, poželjno je usredotočiti se na Lievu algebru
grupe koju čine generatorima $L$:
\begin{equation}
 \Lambda \in \mathcal{L}^{\uparrow}_{+}\,, \qquad \Lambda = e^L \,.
\end{equation}
Iz definicionog svojstva (\ref{deflambda}) dobijemo $L^\top g = -g L$,
što uz činjenicu da je $g^\top = g$ daje $(gL)^\top = -gL$ odnosno
vidimo da je $gL$ antisimetrična matrica. To znači da ako $L$
parametriziramo na slijedeći način
\begin{align}
 gL &= 
\begin{pmatrix}
1 & 0 & 0 & 0 \\
0 &-1 & 0 & 0 \\
0 & 0 &-1 & 0 \\
0 & 0 & 0 &-1
\end{pmatrix}
\begin{pmatrix}
L_{00} & L_{01} & L_{02} & L_{03}\\
L_{10} &\hdotsfor{3} \\
\hdotsfor{4} \\
\hdotsfor{3} & L_{33}
\end{pmatrix} \\[1ex]
&= 
\begin{pmatrix}
L_{00} & L_{01} & L_{02} & L_{03}\\
-L_{10} & -L_{11} & -L_{12} &\hdotsfor{1} \\
-L_{20} & -L_{21} & \hdotsfor{2} \\
\hdotsfor{3} & -L_{33}
\end{pmatrix} \,,
\end{align}
svojstvo antisimetrije $gL$ traži
\begin{align}
L_{0i} &= L_{i0} \,, \\
L_{ij} &= - L_{ji} \;.
\end{align}
tj.
\begin{equation}
 L = \begin{pmatrix}
0 & L_{01} & L_{02} & L_{03}\\
L_{01} & 0 & L_{12} & L_{13}\\
L_{02} & -L_{12} & 0& L_{23} \\
L_{03} & -L_{13} &-L_{23} & 0
\end{pmatrix} \;,
\end{equation}
gdje tri parametra $L_{01}$, $L_{02}$ i $L_{03}$ opisuju Lorentzove
potiske, a tri parametra $L_{12}$, $L_{13}$ i $L_{23}$ prostorne rotacije.
Umjesto ovih šest parametara pogodno je koristiti parametre
$\theta_i$ i $\zeta_i$,  definirane na slijedeći način:
\begin{equation}
 L = -i (\theta_i J_i + \zeta_i K_i)  \qquad i=1,2,3  \;,
\label{eq:LJK}
\end{equation}
gdje su $J_i$ već dobro poznati generatori rotacija, samo prošireni
na četverodimenzionalni prostor Minkowskog
\begin{equation}
 J_1 =
\begin{pmatrix}
0 & 0 & 0 & 0 \\
0 & 0 & 0 & 0 \\
0 & 0 & 0 & -i \\
0 & 0 & i & 0
\end{pmatrix} \quad
 J_2 =
\begin{pmatrix}
0 & 0 & 0 & 0 \\
0 & 0 & 0 & i \\
0 & 0 & 0 & 0 \\
0 & -i & 0 & 0
\end{pmatrix} \quad
J_3 =
\begin{pmatrix}
0 & 0 & 0 & 0 \\
0 & 0 & -i & 0 \\
0 & i & 0 & 0 \\
0 & 0 & 0 & 0
\label{eq:defJi}
\end{pmatrix}
\end{equation}
dok su $K_i$ generatori Lorentzovih potisaka
\begin{equation}
K_1 =
\begin{pmatrix}
0 & -i & 0 & 0 \\
-i & 0 & 0 & 0 \\
0 & 0 & 0 & 0 \\
0 & 0 & 0 & 0
\end{pmatrix} \quad
K_2=
\begin{pmatrix}
0 & 0 & -i & 0 \\
0 & 0 & 0 & 0 \\
-i & 0 & 0 & 0 \\
0 & 0 & 0 & 0
\end{pmatrix} \quad
K_3 =
\begin{pmatrix}
0 & 0 & 0 & -i \\
0 & 0 & 0 & 0 \\
0 & 0 & 0 & 0 \\
-i & 0 & 0 & 0
\end{pmatrix} \;.
\label{eq:defKi}
\end{equation}

Treba primijetiti kako operatori $K_i$ nisu hermitski tako da odgovarajuće
transformacije $\exp(-i\zeta_i K_i)$ neće biti unitarne. To je posljedica
nekompaktnosti Lorentzove grupe --- parametri potiska poprimaju
vrijednosti iz nekompaktnog intervala $[0,c)$.
Univerzalno je svojstvo nekompaktnih Lievih grupa da njihove konačnodimenzionalne
reprezentacije ne mogu biti unitarne.

Pogledajmo sada komutacijske relacije Lieve algebre \soAlg{1,3}. Eskplicitnim
množenjem vidimo da vrijedi
\begin{align}
[J_i, J_j] &= i\epsilon_{ijk} J_k  \,, \label{eq:LK1}\\
[K_i, K_j] &= -i\epsilon_{ijk} J_k \,, \label{eq:LK2} \\
[J_i, K_j] &= i\epsilon_{ijk} K_k \;. \label{eq:LK3}
\end{align}
Prva relacija je dobro poznata algebra \soAlg{3} grupe prostornih rotacija.
Druga relacija govori da podskup Lorentzovih potisaka nije zatvoren
i ne čini grupu, kako smo najavili u prošlom odjeljku. Treća
relacija govori da tri generatora potisaka $K_i$ čine vektor obzirom
na rotacije.
Ove komutacijske relacije su vrlo slične onima iz odjeljka \ref{sec:so4} gdje smo 
rastavili grupu SO(4) na direktan produkt SU(2)$\otimes$SU(2) identificirajući
kombinacije generatora koji zatvaraju dvije neovisne podgrupe.
Slično ćemo postupiti i ovdje te definirati
\begin{equation}
  \vec{J}^{(\pm)} \equiv \fhalf \big(\vec{J}\pm i \vec{K} \big) \;,
  \label{eq:defJpm}
\end{equation}
odnosno $\vec{J}=\vec{J}^{(+)}+\vec{J}^{(-)}$, $\vec{K} = -i
(\vec{J}^{(+)}-\vec{J}^{(-)})$. Primijetite dodatni ``$i$'' obzirom
na situaciju u odjeljku \ref{sec:so4} koji je potreban zbog
minus predznaka u (\ref{eq:LK2}).
Uz ovakve definicije dobivamo dvije odvojene \suAlg{2} algebre
\begin{align}
[J_{i}^{(+)}, J_{j}^{(+)}] &= i \epsilon_{ijk} J_{k}^{(+)} \\
[J_{i}^{(-)}, J_{j}^{(-)}] &= i \epsilon_{ijk} J_{k}^{(-)} \\
[J_{i}^{(+)}, J_{j}^{(-)}] &= 0  \;.
\end{align}
Ovo ipak ne znači  da O(1,3) ima istu algebru kao i SU(2)$\otimes$SU(2),
jer gore u (\ref{eq:defJpm})  nismo radili \emph{realne} linearne kombinacije generatora.
Svejedno ispostavlja se da za klasifikaciju ireducibilnih reprezentacija Lorentzove
grupe možemo kao i u odjeljku \ref{sec:so4} koristiti parove
\begin{equation}
(j^{(+)}, j^{(-)})  \qquad j^{(+)}, j^{(-)} = 0, \fhalf, 1, \dotsc .
\end{equation}
Tako imamo trivijalnu (0,0) reprezentaciju i objekte koji se
transformiraju prema njoj zovemo Lorentzovi skalari. Slijedeće
dvije su tzv. \emph{Weylove} reprezentacije ($\fhalf$, 0) i
(0, $\fhalf$) prema kojima bi se transformirali bezmaseni
fermioni ako takvi postoje .
Obični 4-vektori poput $x^\mu$ pripadaju ireducibilnoj
reprezenaciji $(\fhalf, \fhalf)$.

Za masivne fermione pogodno je proširiti SO(1,3) Lorentzovu
grupu s operacijom pariteta i onda promatrati IRREPse obzirom
na ovu veću grupu\footnote{Masivno stanje impulsa $\vec{p}$
moguće je ``prestići'' Lorentzovim potiskom dovoljno velikog
parametra brzine $\vec{v}$, što rezultira stanjem koje izgleda
kao stanje impulsa $-\vec{p}$; ekvivalentno djelovanju pariteta
na originalno stanje.}. Operator pariteta (\ref{eq:paritet}) 
matrično izgleda kao
\begin{equation}
 P = P^{-1} =
\begin{pmatrix}
1 & 0 & 0 & 0 \\
0 & -1 & 0 & 0 \\
0 & 0 & -1 & 0 \\
0 & 0 & 0 & -1
\end{pmatrix} \;,
\end{equation}
i eksplicitnim djelovanjem na (\ref{eq:defJi}) i (\ref{eq:defKi}) 
je vidljivo da se pri paritetu generatori rotacije
transformiraju kao pseudovektori (vidi odjeljak \ref{sec:pseudovektori}),
\begin{equation}
   P^{-1} J_i P = J_i \;,
\end{equation}
a generatori Lorentzovih potisaka kao pravi polarni vektori,
\begin{equation}
   P^{-1} K_i P = - K_i \;.
\end{equation}
Slijedi da je 
\begin{equation}
 P^{-1} J^{(\pm)}_i P = J^{(\mp)} \;.
\end{equation}
Promotrimo sada neko konkretno stanje $\ket{(\fhalf,0)}$, koje je 
dublet obzirom na transformacije iz ($\fhalf$, 0), 
a singlet obzirom na (0, $\fhalf$):
\begin{align}
\vec{J}^{(+)^2} \ket{(\fhalf,0)}& = \fhalf(\fhalf+1) \ket{(\fhalf,0)} \\
\vec{J}^{(-)^2} \ket{(\fhalf,0)}& = 0 \;.
\end{align}
Kojoj reprezentaciji pripada paritetom transformirano stanje
$P \ket{(\fhalf,0)}$? To ustanovimo djelujući na to stanje
s $\vec{J}^{(\pm)^2}$:
\begin{align}
\vec{J}^{(-)^2} P \ket{(\fhalf,0)}& = 
P P^{-1} J^{(-)}_i P P^{-1} J^{(-)}_i P \ket{(\fhalf,0)}
= P \vec{J}^{(+)^2} \ket{(\fhalf,0)}  \nonumber \\
  & = \fhalf(\fhalf+1) P \ket{(\fhalf,0)} \\
\vec{J}^{(+)^2} P \ket{(\fhalf,0)}& = 0 \;.
\end{align}
Zaključujemo da stanje $P \ket{(\fhalf,0)}$ pripada IRREP (0, $\fhalf$).
Dakle, proširivanjem SO(1,3) Lorentzove grupe s paritetom,
(0, $\fhalf$) i ($\fhalf$,0) više nisu svaka zasebno IRREPs, već
IRREP postaje reprezentacija $(\fhalf, 0) \oplus (0, \fhalf)$, poznata
kao Diracova reprezentacija.


Slično, tenzor elektromagnetskog polja $F^{\mu\nu}$ pripada
6-dimenzionalnoj reducibilnoj reprezentaciji (1,0)$\oplus$(0,1).


Kao što smo u poglavlju \ref{ch:rotacije} i od samih operatora tražili
dobro definirana tenzorska svojstva obzirom na rotacije
(npr. tri generatora $J_i$ čine vektor obzirom na rotacije), tako
je i u kontekstu Lorentzove simetrije moguće generatore i
same elemente grupe definirati na način koji manifestno pokazuje
kovarijantnost obzirom na Lorentzove transformacije.
Dakle, želimo relacije poput (\ref{eq:LJK}) i (\ref{eq:LK1})--(\ref{eq:LK3}) 
zapisati u 4-komponentnoj notaciji, putem Lorentzovih tenzora.
To postižemo definiranjem antisimetričnih generatora $J^{\mu\nu}$ kao
\begin{align}
J^{mn}& = \epsilon^{mni} J^i \\
J^{i0}& = K^i \;.
\label{eq:defJmn}
\end{align}
(Podsjetimo se da grčki indeksi idu $\alpha,\beta = 0,1,2,3$, a 
latinski $i,m = 1,2,3$.)  
Sada je element Lorentzove grupe dan kao
\begin{equation}
 \Lambda = \exp\left(-\frac{i}{2} \omega_{\rho\sigma} J^{\rho\sigma}\right)
\label{eq:4DLorentz}
\end{equation}
a komutacijske relacije (\ref{eq:LK1})--(\ref{eq:LK3}) se ujedinjuju u
\begin{equation}
i [ J^{\mu\nu}, J^{\rho\sigma}] = g^{\mu\rho} J^{\nu\sigma}
- g^{\nu\rho} J^{\mu\sigma} + g^{\sigma\mu} J^{\rho\nu}
- g^{\sigma\nu} J^{\rho\mu} \;.
\label{eq:4DLorKom}
\end{equation}

Na kraju, matrični elementi generatora Lorentzove grupe u
4-dim definicionoj ($\fhalf$, $\fhalf$) reprezentaciji dani su kao
\begin{equation}
\tensor{(J^{\mu\nu})}{^\alpha_\beta} = i \big(
g^{\mu\alpha}\tensor{\delta}{^\nu_\beta} - \tensor{\delta}{^\mu_\beta}
g^{\nu\alpha} \big) \;.
\label{eq:4DJmn}
\end{equation}

S druge strane matrični elementi Lorentzove grupe u
Diracovoj reprezentaciji
$(\fhalf, 0) \oplus (0, \fhalf)$
(koja je isto 4-dim) dani su kao
\begin{equation}
 J^{\mu\nu} = \frac{i}{4} [\gamma^\mu, \gamma^\nu] \;,
\label{eq:DiracJ}
\end{equation}
Gdje su $\gamma^\mu$ tzv. Diracove gama matrice definirane
antikomutacijskim relacijama
\begin{equation}
 \gamma^\mu\gamma^\nu + \gamma^\nu \gamma^\mu
\equiv \{\gamma^\mu, \gamma^\nu\} = 2 g^{\mu\nu}
\label{eq:DiracGamma}
\end{equation}
Jedan mogući izbor za ove matrice je
\begin{equation}
\gamma^0 = \left(\begin{array}{rr} 0 & -1 \\
                        -1 & 0 \end{array}\right)
\;, \quad
\gamma^i = \left(\begin{array}{rr} 0 & \sigma^i \\
                        -\sigma^i & 0 \end{array}\right) \;,
\end{equation}
gdje su $\sigma^i$ Paulijeve matrice (\ref{eq:PaulijeveMatrice}).

\section{Poincar\'{e}ova grupa$^*$}


Poincar\'{e}ova (poznata i kao nehomogena Lorentzova) grupa je
direktni produkt Lorentzove grupe i grupe translacija u prostoru
i vremenu. Elementi te grupe djeluju na vektore u 4-dim prostoru
Minkowskog kao
\begin{equation}
  x^\mu \longrightarrow \tensor{\Lambda}{^\mu_\nu} x^\nu + a^\mu
\label{eq:defPoin}
\end{equation}
gdje su $a^\mu$ četiri parametra translacija u prostor-vremenu.
Riječ je očito o deset-parametarskoj grupi gdje povrh generatora
Lorentzove grupe ($J^i$ i $K^i$) imamo i impuls $P^i$ kao
generator translacija u prostoru i Hamiltonijan $H$ kao
generator translacija u vremenu.
Algebra ove grupe je algebra Lorentzove grupe SO(1,3) (\ref{eq:4DLorKom}) 
i dodatno
\begin{align}
[H, H]& = [H, P^i] = [P^i, P^j] = 0 \label{eq:Poi1} \\
[H, J_i]& = 0  \label{eq:Poi2}\\
[K^i, H]& = i P^i  \label{eq:PoiPuzzle} \\
[J^i, P^j]& = i \epsilon^{ijk} P^k \\
[K^i, P^j]& = i H \delta^{ij} \;,
\label{eq:PoinKom}
\end{align}
što je moguće ujedinjeno napisati u 4-dim notaciji kao
\begin{equation}
i [P^\mu, J^{\rho\sigma}] = g^{\mu\sigma}P^\rho - g^{\mu\rho}P^{\sigma} \;.
\label{eq:4DPoinKom}
\end{equation}
Npr.
\begin{displaymath}
[H, K^i] = [P^0, J^{i0}] = -i(g^{0 0}P^i - g^{0i}P^0) =  -i P^i \;.
\end{displaymath}
Relacije  (\ref{eq:Poi1}) i (\ref{eq:Poi2}) govore da su impuls i
moment impulsa očuvani pri translacijama. Zanimljivo je da relacija
(\ref{eq:PoiPuzzle}) govori kako generatori Lorentzovih potisaka
$K^i$ ne komutiraju s Hamiltonijanom što sugerira da ne odgovaraju
očuvanim veličinama. To je pomalo zbunjujuće jer Noetherin teorem
nam govori kako bi simetrija obzirom na 10-parametarsku Poincar\'{e}ovu grupu
trebala rezultirati s deset očuvanih veličina, a ovako imamo samo
7: energija, tri komponente impulsa i tri komponente momenta impulsa.
Razrješenje je u tome da Noetherini očuvani naboji
(pokušajte ih eksplicitno odrediti) koji
odgovaraju Lorentzovim potiscima jesu formalno očuvani (vremenska
derivacija iščezava), ali eksplicitno ovise o vremenskoj 
koordinati\footnote{Slično kao što npr. $J_z = x P_y - y P_z$ 
eksplicitno ovisi o prostornim $x$ i $y$ koordinatama.}
pa nisu od praktične koristi. 

\subsection*{Zadaci}

\begin{enumerate}[label=\arabic{chapter}.\arabic*.]

\item Uvjerite se eksplicitno da (\ref{eq:4DLorKom}) sadrži npr. (\ref{eq:LK3}).

\item Uvjerite se eksplicitno da (\ref{eq:4DJmn}) daje npr. (\ref{eq:defKi}).

\item Uvjerite se da generatori $J^{\mu\nu}$ definirani putem
(\ref{eq:DiracJ}) i (\ref{eq:DiracGamma}) zadovoljavaju komutacijske
relacije Lorentzove grupe (\ref{eq:4DLorKom}).

\end{enumerate}
